\section{Diskussion}
\label{sec:Diskussion}
Die durch atomare Größen und druch Messgrößen bestimmten Suszeptibilitäten und ihre jeweilige Abweichung sind:
\begin{align*}
  \chi_\text{Dy,Atomar} &= \num{24.86e-3} & \chi_\text{Dy,Messung} &= \num{23.0\pm0.8e-3} \\
    & \Rightarrow \text{Abweichung} = \SI{7.48}{\percent} \\
  \chi_\text{Nd,Atomar} &= \num{2.91e-3} & \chi_\text{Nd,Messung} &= \num{5.9\pm0.6e-3} \\
    & \Rightarrow \text{Abweichung} = \SI{102.75}{\percent} \\
  \chi_\text{Gd,Atomar} &= \num{13.56e-3} & \chi_\text{Gd,Messung} &= \num{10.2\pm0.5e-3} \\
    & \Rightarrow \text{Abweichung} = \SI{24.78}{\percent}
\end{align*}
Es ist allgemein zu beobachten, dass die Abweichungen bei kleineren Suszeptibilitäten größer werden. Dies lässt auf eine nicht ausreichende kalibrierung
der Messapparatur schlißen, welche die Messauflösung beeinträchtigt.
Des Weiteren könnte es sein, dass sich die Proben erwärmt haben und sich somit die Suszeptibilität verändert hat. Zuletzt war die Messapparatur noch sehr
anfällig für Störungen, da die eingebaute Spule locker war.
Die Theorie lässt sich so nur für das Dysprosium(III)-dioxid mit hinreichender Genauigkeit bestätigen. Zur vollständigen Bestätigung der Theorie müsste eine
erneute Messung unter Ausschluss der oben genannten Fehlerquellen durchgeführt werden. 
