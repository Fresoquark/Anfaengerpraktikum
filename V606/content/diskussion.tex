\section{Diskussion}
\label{sec:Diskussion}
Die durch atomare Größen und druch Messgrößen bestimmten Suszeptibilitäten und ihre jeweilige Abweichung sind:
\begin{align*}
  \chi_\text{Dy,Atomar} &= \num{12.43e-3} & \chi_\text{Dy,Messung} &= \num{7.9\pm0.4e-3} \\
    & \Rightarrow \text{Abweichung} = \SI{57.34}{\percent} \\
  \chi_\text{Nd,Atomar} &= \num{1.46e-3} & \chi_\text{Nd,Messung} &= \num{2.6\pm0.2e-3} \\
    & \Rightarrow \text{Abweichung} = \SI{78.08}{\percent} \\
  \chi_\text{Gd,Atomar} &= \num{6.78e-3} & \chi_\text{Gd,Messung} &= \num{4.1\pm0.2e-3} \\
    & \Rightarrow \text{Abweichung} = \SI{65.37}{\percent}
\end{align*}
Es ist zu beobachten, dass die Werte sehr stark voneinader Abweichen. Dies kann verschiedene Gründe haben.
Zum einen kann es sein, dass die Brückenschlatung nicht perfekt eingestellt war, und dadurch Fehler enstanden sind, beziehungsweise
die Messdaten nicht innerhalb der erreichten Auflösung liegen. Des weiteren könnten sich die Proben erwärmt haben und somit schlechte
Messergebnisse erzielt worden sein. Außerdem war die Anordnung sehr anfällig für Störungen, da die Spule locker war, was auch eine
Fehlerquelle darstellt. Zuletzt war das Minimum der Brückenspannung nicht an genau einem Widerstand, sonder viel mehr in einem größeren
Wertebereich für den Widerstand messbar. Dies erhöht die Messungenauigkeit auch. Schlussendlich wurden die Suszeptibilitäten nicht
mit einer hinreichenden Genauigkeit bestimmt, und zur bestätigung der Theorie müsste eine erneute Messung mit weniger Fehlerquellen
durchgeführt werden.
