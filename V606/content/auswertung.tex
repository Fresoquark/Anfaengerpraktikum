\section{Auswertung}
\label{sec:Auswertung}

\subsection{Untersuchen der Filterkurve des verwendeten Selektivverstärkers}

Die Güte des Selektivverstärkers wird nach Gleichung \eqref{eqn:Güte} berechnet.
Um an die Werte für $\nu_0$, $\nu_-$ und $\nu_+$ zu gelangen, werden die entsprechenden Daten in Abbildung \ref{fig:Filterkurve} aufgezeichnet.
Die Messwerte sind in Tabelle \ref{tab:Filterkurve} zusammengetragen.
Die Eingangsspannung beträgt $\SI{1}{\volt}$.
Die Werte in Tabelle \ref{tab:Filterkurve} für $\text{U}_\text{A}$ sind durch den Selektivverstärker noch einmal um den Faktor 10 verstärkt worden.
Unter Berücksichtigung dieses Faktors erhält man das richtige Verhältnis  von $\text{U}_\text{A}$ zu $\text{U}_\text{E}$ in Abbildung \ref{fig:Filterkurve}.

\begin{table}
  \centering
  \caption{Messwerte der untersuchten Filterkurve des Selektivverstärkers}
  \label{tab:Filterkurve}
  \begin{tabular}[t]{c@{} S [table-format=2.0] S [table-format=1.2] |}
   \toprule
    $\omega \, / \, \si{\kilo\hertz}\:\:$ & $\text{U}_\text{A}$ \, /  \, $\si{\volt}$\\
     \midrule
     \csvreader[no head,
     late after line=\\,
     late after last line=\\\bottomrule,
     filter test={\ifnumless{\thecsvinputline}{12}}]%
     {data/verstaerker.csv}{}%
     {$\SI{\csvcoli}{}$ & $\SI{\csvcolii}{}$}%
   \end{tabular}
   \begin{tabular}[t]{| c@{} S [table-format=2.0] S [table-format=1.2] |}
    \toprule
     $\omega \, / \, \si{\kilo\hertz}\:\:$ & $\text{U}_\text{A}$ \, /  \, $\si{\volt}$\\
      \midrule
      \csvreader[filter expr={ test{\ifnumgreater{\thecsvinputline}{11}}
                           and test{\ifnumless{\thecsvinputline}{23}}},
      late after line=\\,
      late after last line=\\\bottomrule]%
      {data/verstaerker.csv}{}%
      {$\SI{\csvcoli}{}$ & $\SI{\csvcolii}{}$}%
    \end{tabular}
    \begin{tabular}[t]{| c@{} S [table-format=2.0] S [table-format=1.2]}
     \toprule
      $\omega \, / \, \si{\kilo\hertz}\:\:$ & $\text{U}_\text{A}$ \, /  \, $\si{\volt}$\\
       \midrule
       \csvreader[filter test={\ifnumgreater{\thecsvinputline}{22}},
       late after line=\\,
       late after last line=\\\bottomrule]%
       {data/verstaerker.csv}{}%
       {$\SI{\csvcoli}{}$ & $\SI{\csvcolii}{}$}%
     \end{tabular}
 \end{table}

 \begin{figure}
   \centering
   \includegraphics{build/verstaerker.pdf}
   \caption{Filterkurve, die sich aus den Messwerten aus Tabelle \ref{tab:Filterkurve} ergibt.}
   \label{fig:Filterkurve}
 \end{figure}

Aus den Daten wird ersichtlich, dass $\nu_0$ bei einer Frequenz von $\SI{35.1}{\kilo\hertz}$ liegt.
Das Spannungsverhältnis beträgt dort $0.91$.
Für $\nu_-$ und $\nu_+$ ergibt sich somit ein theoretisches Verhältnis von 0.6435.
Da es keine Messwerte mit diesem Verhältnis gibt, werden die Messwerte genutz, die diesem Verhältnis am nächsten kommen.
Die sind für $\nu_- = 0.68$ bei einer Frequenz von $\SI{34.9}{\kilo\hertz}$ und für $\nu_+ = 0.72$ bei einer Frequenz von $\SI{35.2}{\kilo\hertz}$.
Die Angaben für die Frequenzen von $\nu_-$ und $\nu_+$ sind hinreichend genau, da bei Betrachtung des Kurvenverlaufs die tatsächlichen Frequenzen für das Verhältnis von 0.6435 lediglich um wenige $\SI{10}{\hertz}$ abweichen würden.
Die Güte lässt sich damit nach Gleichung \eqref{eqn:Güte} folgendermaßen bestimmen:

\begin{equation*}
  Q = \frac{\SI{35.1}{\kilo\hertz}}{\SI{35.2}{\kilo\hertz} - \SI{34.9}{\kilo\hertz}} = 117
\end{equation*}

Dieser Wert ist allerdings nicht die tatsächliche Güte, da hierfür die Auflösung der Frequenzen zu gering war.
