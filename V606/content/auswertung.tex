\section{Auswertung}
\label{sec:Auswertung}

\subsection{Untersuchen der Filterkurve des verwendeten Selektivverstärkers}

Die Güte des Selektivverstärkers wird nach Gleichung \eqref{eqn:Güte} berechnet.
Um an die Werte für $\nu_0$, $\nu_-$ und $\nu_+$ zu gelangen, werden die entsprechenden Daten in Abbildung \ref{fig:Filterkurve} aufgezeichnet.
Die Messwerte sind in Tabelle \ref{tab:Filterkurve} zusammengetragen.
Die Eingangsspannung beträgt $\SI{1}{\volt}$.
Die Werte in Tabelle \ref{tab:Filterkurve} für $\text{U}_\text{A}$ sind durch den Selektivverstärker noch einmal um den Faktor 10 verstärkt worden.
Unter Berücksichtigung dieses Faktors erhält man das richtige Verhältnis  von $\text{U}_\text{A}$ zu $\text{U}_\text{E}$ in Abbildung \ref{fig:Filterkurve}.

\begin{table}
  \centering
  \caption{Messwerte der untersuchten Filterkurve des Selektivverstärkers}
  \label{tab:Filterkurve}
  \begin{tabular}[t]{c@{} S [table-format=2.0] S [table-format=1.2] |}
   \toprule
    $\omega \, / \, \si{\kilo\hertz}\:\:$ & $\text{U}_\text{A}$ \, /  \, $\si{\volt}$\\
     \midrule
     \csvreader[no head,
     late after line=\\,
     late after last line=\\\bottomrule,
     filter test={\ifnumless{\thecsvinputline}{12}}]%
     {data/verstaerker.csv}{}%
     {$\SI{\csvcoli}{}$ & $\SI{\csvcolii}{}$}%
   \end{tabular}
   \begin{tabular}[t]{| c@{} S [table-format=2.0] S [table-format=1.2] |}
    \toprule
     $\omega \, / \, \si{\kilo\hertz}\:\:$ & $\text{U}_\text{A}$ \, /  \, $\si{\volt}$\\
      \midrule
      \csvreader[filter expr={ test{\ifnumgreater{\thecsvinputline}{11}}
                           and test{\ifnumless{\thecsvinputline}{23}}},
      late after line=\\,
      late after last line=\\\bottomrule]%
      {data/verstaerker.csv}{}%
      {$\SI{\csvcoli}{}$ & $\SI{\csvcolii}{}$}%
    \end{tabular}
    \begin{tabular}[t]{| c@{} S [table-format=2.0] S [table-format=1.2]}
     \toprule
      $\omega \, / \, \si{\kilo\hertz}\:\:$ & $\text{U}_\text{A}$ \, /  \, $\si{\volt}$\\
       \midrule
       \csvreader[filter test={\ifnumgreater{\thecsvinputline}{22}},
       late after line=\\,
       late after last line=\\\bottomrule]%
       {data/verstaerker.csv}{}%
       {$\SI{\csvcoli}{}$ & $\SI{\csvcolii}{}$}%
     \end{tabular}
 \end{table}

 \begin{figure}
   \centering
   \includegraphics{build/verstaerker.pdf}
   \caption{Filterkurve, die sich aus den Messwerten aus Tabelle \ref{tab:Filterkurve} ergibt.}
   \label{fig:Filterkurve}
 \end{figure}

Aus den Daten wird ersichtlich, dass $\nu_0$ bei einer Frequenz von $\SI{35.1}{\kilo\hertz}$ liegt.
Das Spannungsverhältnis beträgt dort $0.91$.
Für $\nu_-$ und $\nu_+$ ergibt sich somit nach Gleichung \eqref{eqn:Güte} ein theoretisches Verhältnis von $\num{0.6435}$.
Da für diese Verhältnisse keine Daten vorliegen, wird die Funktion zwischen den zwei nächsten Datenpunkten linear mit $y=ax+b$ interpoliert.
Nun wird diese Formel nach $x$ umgestellt und die bestimmten Parameter $a$ und $b$ sowie das theoretische Verhältnis $\num{0.6435}$ für $y$ eingesetzt.
So werden folgende Werte bestimmt:
\begin{align*}
  \nu_- &= \SI{34.98}{\kilo\hertz} \\
  \nu_+ &= \SI{35.34}{\kilo\hertz} .
\end{align*}
Das Verhältnis, die Interpolationen, sowie $\nu_-$ und $\nu_+$ sind ebenfalls in Abbildung \ref{fig:Filterkurve} dargestellt. 
Die Güte lässt sich damit nach Gleichung \eqref{eqn:Güte} folgendermaßen bestimmen:

\begin{equation*}
  Q = \frac{\SI{35.1}{\kilo\hertz}}{\SI{35.34}{\kilo\hertz} - \SI{34.98}{\kilo\hertz}} = 97,5
\end{equation*}

Dieser Wert ist allerdings nicht die tatsächliche Güte, da hierfür die Auflösung der Frequenzen zu gering war.

\subsection{Berechnung der Suszeptibilität durch atomare Größen}
Zur Berechnung der Suszeptibilität durch atomare Größen werden die Spin-, Bahndrehimpuls- und Gesamtdrehimpulsquantenzahl aus den aus \cite{sample} entnommenen Angaben bestimmt.
Daraus wird dann mit Gleichung \eqref{eqn:lande} der Landé-Faktor bestimmt. Die Werte wurden in Tabelle \ref{tab:lande} zusammengetragen.
\begin{table}
  \centering
  \caption{S, L, J und Landé-Faktor der zu untersuchenden Stoffe.}
  \label{tab:lande}
  \begin{tabular}{l c c c c}
    \toprule
Name & S & L & J & $g_J$ \\
    \midrule
  $\text{Nd}_2 \text{O}_3$ & 1,5 & 6 & 4,5 & 0,72 \\
  $\text{Gd}_2 \text{O}_3$ & 3,5 & 0 & 3,5 & 2 \\
  $\text{Dy}_2 \text{O}_3$ & 2,5 & 5 & 7,5 & 1,33 \\
  \bottomrule
  \end{tabular}
\end{table}
Die Momente pro Volumeneinheit $N=2 \, N_A \cdot \frac{\rho}{M}$ werden aus den aus \cite{sample} entnommenen Dichten, den aus \cite{dy}, \cite{gd} und \cite{nd} entnommenen molaren Massen und der
aus \cite{scipy} entnommenen Avogadro-Konstante berechnet. Die Werte wurden in Tabelle \ref{tab:N} zusammengetragen.
\begin{table}
  \centering
  \caption{Dichten, molare Massen und Momente pro Volumeneinheit der zu untersuchenden Stoffe.}
  \label{tab:N}
  \begin{tabular}{l c c c}
    \toprule
Name & $\rho$ / $\si{\kilo\gram\per\meter\cubed}$ & Molare Masse / $\si{\kilo\gram\per\mol}$ & N / $\SI{e28}{\per\meter\cubed}$ \\
    \midrule
  $\text{Nd}_2 \text{O}_3$ & 7240 & 0,34 & 2,59\\
  $\text{Gd}_2 \text{O}_3$ & 7400 & 0,36 & 2,46\\
  $\text{Dy}_2 \text{O}_3$ & 7800 & 0,37 & 2,52 \\
  \bottomrule
  \end{tabular}
\end{table}
Die Temperatur wird als $\SI{298.15}{\kelvin}$ angenommen.
Aus diesen Werten werden nun mittels Gleichung \eqref{eqn:atomchi} die Suszeptibilitäten der zu untersuchenden Stoffe zu folgenden Werten berechnet:
\begin{align*}
  \chi_\text{Nd} &= 2,91\cdot10^{-3} \\
  \chi_\text{Gd} &= 13,56\cdot10^{-3}\\
  \chi_\text{Dy} &= 24,86\cdot10^{-3} .
\end{align*}
\subsection{Bestimmung der Suszeptibilität von Seltener-Erd-Verbindungen}

Die Messwerte der Brückenschaltung sind in Tabelle \ref{tab:Brückenmessung} zusammengetragen.

\begin{table}
  \centering
  \caption{Messwerte der verwendeten Brückenschaltung zur Bestimmung der Suszeptibilität}
  \label{tab:Brückenmessung}
  \begin{tabular}{l c c c c c c}
    \toprule
    {$\text{Name}$} & {$\text{U}_0 \:/\: \si{\milli\volt}$} & {$\text{U}_1 \:/\: \si{\milli\volt}$} & {$\Delta \text{U} \:/\: \si{\milli\volt}$}  & {$\text{R}_0 \:/\: \si{\milli\ohm}$} & {$\text{R}_2 \:/\: \si{\milli\ohm}$} & {$\Delta \text{R} \:/\: \si{\milli\ohm}$} \\
     \midrule
     $\text{Dy}_2 \text{O}_3$ & 3.35 & 4.55 & 1.20 & 4530 & 3275 & 1255 \\
                            & 3.40 & 4.70 & 1.30 & 4560 & 3150 & 1410 \\
                            & 3.30 & 4.60 & 1.30 & 4485 & 2785 & 1700 \\
                            \hline
     $\text{Nd}_2 \text{O}_3$ & 3.35 & 3.40 & 0.05 & 4375 & 3925 & 450  \\
                            & 3.30 & 3.40 & 0.10 & 4230 & 3820 & 410  \\
                            & 3.30 & 3.35 & 0.05 & 4165 & 3800 & 365  \\
                            \hline
     $\text{Gd}_2 \text{O}_3$ & 3.30 & 3.50 & 0.20 & 4115 & 3345 & 770  \\
                            & 3.30 & 3.55 & 0.25 & 4105 & 3360 & 745  \\
                            & 3.35 & 3.65 & 0.30 & 4250 & 3380 & 870  \\
     \bottomrule
  \end{tabular}
\end{table}

Die einzelnen Eigenschaften der jeweiligen Proben sind in Tabelle \ref{tab:Eigenschaften} dargestellt.
Der reale Querschnitt Q$_\text{real}$ wurde nach der Gleichung \eqref{eqn:qreal} errechnet.

\begin{table}
  \centering
  \caption{Physikalische Eigenschaften der Proben.}
  \label{tab:Eigenschaften}
  \begin{tabular}{l c c c c}
    \toprule
    {$\text{Name}$} & {$\text{M}_\text{p} \:/\: \si{\gram}$} & {$\text{L}_\text{p} \:/\: \si{\centi\metre}$} & {$\rho_\text{p} \:/\: \si{\gram\per\cubic\per\centi\per\metre}$} & {$\text{Q}_\text{real} \:/\: \si{\centi\metre\squared}$} \\
    \midrule
    $\text{Dy}_2 \text{O}_3$ & 15.1  & 13   & 7.80 & 0.15 \\
    $\text{Nd}_2 \text{O}_3$ & 9     & 16   & 7.24 & 0.08 \\
    $\text{Gd}_2 \text{O}_3$ & 14.08 & 16.7 & 7.40 & 0.11 \\
    \bottomrule
  \end{tabular}
\end{table}
Der bekannte Widerstand $R_3$ beträgt $\SI{998}{\ohm}$, die Speisespannung $U_\text{Sp}$ beträgt $\SI{1}{\volt}$ und der Querschnitt der Spule
$\SI{8.66}{\milli\metre\squared}$.
Mit diesen Werten wurde die Suszeptibilität einmal durch Gleichung \eqref{eqn:chi1} und einmal durch Gleichung \eqref{eqn:chi2} berechnet.
Die Ergebnisse sind in Tabelle \ref{tab:chi} zusammengefasst.
\FloatBarrier
\begin{table}
  \centering
  \caption{Die aus den Messdaten bestimmten Suszeptibilitäten.}
  \label{tab:chi}
  \setlength\tabcolsep{2pt}
  \begin{tabular}{c c | c c | c c}
    \toprule
    $\chi_\text{Dy}(U_\text{Br})\:/10^{-3}$ & $\chi_\text{Dy}(\Delta R) \:/10^{-3}$& $\chi_\text{Nd}(U_\text{Br})\:/10^{-3}$ & $\chi_\text{Nd}(\Delta R)\:/10^{-3}$ & $\chi_\text{Gd}(U_\text{Br})\:/10^{-3}$ & $\chi_\text{Gd}(\Delta R)\:/10^{-3}$ \\
    \midrule
    27,71 & 14,52 & 2,17 & 9,76 & 6,30 & 12,15 \\
    30,02 & 16,31 & 4,33 & 8,89 & 7,87 & 11,75 \\
    30,02 & 19,67 & 2,17 & 7,92 & 9,44 & 13,73 \\
    \bottomrule
  \end{tabular}
\end{table}
\FloatBarrier
Die berechneten Suszeptibilitäten werden nun mit
\begin{equation}
  \label{eqn:mittelwert}
  \overline{x} = \frac{1}{N} \sum_{i=1}^N x_i
\end{equation}
gemittelt, und der Fehler mit
\begin{equation}
  \label{eqn:mittelwertfehler}
  \Delta \overline{x} = \frac{1}{\sqrt{N}} \sqrt{\frac{1}{N-1} \sum_{i=1}^N (x_i - \overline{x})^2}
\end{equation}
berechnet.
So ergibt sich folgendes:
\begin{align*}
  \chi_\text{Dy}(U_\text{Br}) &= \SI{29.3\pm0.8e-3}{} & \chi_\text{Dy}(\Delta R) &= \SI{16.8\pm1.5e-3}{}\\
  \chi_\text{Nd}(U_\text{Br}) &= \SI{2.9\pm0.7e-3}{} & \chi_\text{Nd}(\Delta R) &= \SI{8.9\pm0.5e-3}{}\\
  \chi_\text{Gd}(U_\text{Br}) &= \SI{7.9\pm0.9e-3}{} & \chi_\text{Gd}(\Delta R) &= \SI{12.5\pm0.6e-3}{} .
\end{align*}
Nun werden noch die über die Spannung und den Widerstand ermittelten mittleren Suszeptibilitäten gemittelt.
So ergeben sich folgende Werte:
\begin{align*}
  \chi_\text{Dy} &= \SI{23.0\pm0.8e-3}{} \\
  \chi_\text{Nd} &= \SI{5.9\pm0.6e-3}{} \\
  \chi_\text{Gd} &= \SI{10.2\pm0.5e-3}{}
\end{align*}
