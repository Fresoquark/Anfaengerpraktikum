\section{Auswertung}
\label{sec:Auswertung}

\subsection{Untersuchen der Filterkurve des verwendeten Selektivverstärkers}

Die Güte des Selektivverstärkers wird nach Gleichung \eqref{eqn:Güte} berechnet.
Um an die Werte für $\nu_0$, $\nu_-$ und $\nu_+$ zu gelangenl, werden die entsprechenden Daten in Abbildung \ref{fig:Filterkurve} aufgezeichnet.
Die Messwerte sind in Tabelle \ref{tab:Filterkurve} zusammengetragen.
Die Eingangsspannung beträgt $\SI{1}{\volt}$.
Die Werte in Tabelle \ref{tab:Filterkurve} für $U_A$ sind durch den Selektivverstärker noch einmal um den Faktor 10 verstärkt worden.
Unter Berücksichtigung dieses Faktors erhält man das richtige Verhältnis  von $U_A$ zu $U_E$ in Abbildung \ref{fig:Filterkurve}.

\begin{table}
  \centering
  \caption{Messwerte der untersuchten Filterkurve des Selektivverstärkers}
  \label{tab:Filterkurve}
  \begin{tabular}[t]{c@{} S [table-format=2.0] S [table-format=1.2] |}
   \toprule
    $\omega \, / \, \si{\kilo\hertz}\:\:$ & $U_A$ \, /  \, $\si{\volt}$\\
     \midrule
     \csvreader[no head,
     late after line=\\,
     late after last line=\\\bottomrule,
     filter test={\ifnumless{\thecsvinputline}{12}}]%
     {data/verstaerker.csv}{}%
     {$\SI{\csvcoli}{}$ & $\SI{\csvcolii}{}$}%
   \end{tabular}
   \begin{tabular}[t]{| c@{} S [table-format=2.0] S [table-format=1.2] |}
    \toprule
     $\omega \, / \, \si{\kilo\hertz}\:\:$ & $U_A$ \, /  \, $\si{\volt}$\\
      \midrule
      \csvreader[filter expr={ test{\ifnumgreater{\thecsvinputline}{11}}
                           and test{\ifnumless{\thecsvinputline}{23}}},
      late after line=\\,
      late after last line=\\\bottomrule]%
      {data/verstaerker.csv}{}%
      {$\SI{\csvcoli}{}$ & $\SI{\csvcolii}{}$}%
    \end{tabular}
    \begin{tabular}[t]{| c@{} S [table-format=2.0] S [table-format=1.2]}
     \toprule
      $\omega \, / \, \si{\kilo\hertz}\:\:$ & $U_A$ \, /  \, $\si{\volt}$\\
       \midrule
       \csvreader[filter test={\ifnumgreater{\thecsvinputline}{22}},
       late after line=\\,
       late after last line=\\\bottomrule]%
       {data/verstaerker.csv}{}%
       {$\SI{\csvcoli}{}$ & $\SI{\csvcolii}{}$}%
     \end{tabular}
 \end{table}

 \begin{figure}
   \centering
   \includegraphics{build/verstaerker.pdf}
   \caption{Filterkurve, die sich aus den Messwerten aus Tabelle \ref{tab:Filterkurve} ergibt.}
   \label{fig:Filterkurve}
 \end{figure}
