\section{Auswertung}
\label{sec:Auswertung}

\subsection{Untersuchen der Filterkurve des verwendeten Selektivverstärkers}

Die Güte des Selektivverstärkers wird nach Gleichung \eqref{eqn:Güte} berechnet.
Um an die Werte für $\nu_0$, $\nu_-$ und $\nu_+$ zu gelangen, werden die entsprechenden Daten in Abbildung \ref{fig:Filterkurve} aufgezeichnet.
Die Messwerte sind in Tabelle \ref{tab:Filterkurve} zusammengetragen.
Die Eingangsspannung beträgt $\SI{1}{\volt}$.
Die Werte in Tabelle \ref{tab:Filterkurve} für $\text{U}_\text{A}$ sind durch den Selektivverstärker noch einmal um den Faktor 10 verstärkt worden.
Unter Berücksichtigung dieses Faktors erhält man das richtige Verhältnis  von $\text{U}_\text{A}$ zu $\text{U}_\text{E}$ in Abbildung \ref{fig:Filterkurve}.

\begin{table}
  \centering
  \caption{Messwerte der untersuchten Filterkurve des Selektivverstärkers}
  \label{tab:Filterkurve}
  \begin{tabular}[t]{c@{} S [table-format=2.0] S [table-format=1.2] |}
   \toprule
    $\omega \, / \, \si{\kilo\hertz}\:\:$ & $\text{U}_\text{A}$ \, /  \, $\si{\volt}$\\
     \midrule
     \csvreader[no head,
     late after line=\\,
     late after last line=\\\bottomrule,
     filter test={\ifnumless{\thecsvinputline}{12}}]%
     {data/verstaerker.csv}{}%
     {$\SI{\csvcoli}{}$ & $\SI{\csvcolii}{}$}%
   \end{tabular}
   \begin{tabular}[t]{| c@{} S [table-format=2.0] S [table-format=1.2] |}
    \toprule
     $\omega \, / \, \si{\kilo\hertz}\:\:$ & $\text{U}_\text{A}$ \, /  \, $\si{\volt}$\\
      \midrule
      \csvreader[filter expr={ test{\ifnumgreater{\thecsvinputline}{11}}
                           and test{\ifnumless{\thecsvinputline}{23}}},
      late after line=\\,
      late after last line=\\\bottomrule]%
      {data/verstaerker.csv}{}%
      {$\SI{\csvcoli}{}$ & $\SI{\csvcolii}{}$}%
    \end{tabular}
    \begin{tabular}[t]{| c@{} S [table-format=2.0] S [table-format=1.2]}
     \toprule
      $\omega \, / \, \si{\kilo\hertz}\:\:$ & $\text{U}_\text{A}$ \, /  \, $\si{\volt}$\\
       \midrule
       \csvreader[filter test={\ifnumgreater{\thecsvinputline}{22}},
       late after line=\\,
       late after last line=\\\bottomrule]%
       {data/verstaerker.csv}{}%
       {$\SI{\csvcoli}{}$ & $\SI{\csvcolii}{}$}%
     \end{tabular}
 \end{table}

 \begin{figure}
   \centering
   \includegraphics{build/verstaerker.pdf}
   \caption{Filterkurve, die sich aus den Messwerten aus Tabelle \ref{tab:Filterkurve} ergibt.}
   \label{fig:Filterkurve}
 \end{figure}

Aus den Daten wird ersichtlich, dass $\nu_0$ bei einer Frequenz von $\SI{35.1}{\kilo\hertz}$ liegt.
Das Spannungsverhältnis beträgt dort $0.91$.
Für $\nu_-$ und $\nu_+$ ergibt sich somit ein theoretisches Verhältnis von 0.6435.
Da es keine Messwerte mit diesem Verhältnis gibt, werden die Messwerte genutz, die diesem Verhältnis am nächsten kommen.
Die sind für $\nu_- = 0.68$ bei einer Frequenz von $\SI{34.9}{\kilo\hertz}$ und für $\nu_+ = 0.72$ bei einer Frequenz von $\SI{35.2}{\kilo\hertz}$.
Die Angaben für die Frequenzen von $\nu_-$ und $\nu_+$ sind hinreichend genau, da bei Betrachtung des Kurvenverlaufs die tatsächlichen Frequenzen für das Verhältnis von 0.6435 lediglich um wenige $\SI{10}{\hertz}$ abweichen würden.
Die Güte lässt sich damit nach Gleichung \eqref{eqn:Güte} folgendermaßen bestimmen:

\begin{equation*}
  Q = \frac{\SI{35.1}{\kilo\hertz}}{\SI{35.2}{\kilo\hertz} - \SI{34.9}{\kilo\hertz}} = 117
\end{equation*}

Dieser Wert ist allerdings nicht die tatsächliche Güte, da hierfür die Auflösung der Frequenzen zu gering war.

\subsection{Bestimmung der Suszeptibilität von Seltener-Erd-Verbindungen}

Die Messwerte der Brückenschaltung sind in Tabelle \ref{tab:Brückenmessung} zusammengetragen.

\begin{table}
  \centering
  \caption{Messwerte der verwendeten Brückenschaltung zur Bestimmung der Suszeptibilität}
  \label{tab:Brückenmessung}
  \begin{tabular}{l c c c c c c}
    \toprule
    {$\text{Name}$} & {$\text{U}_0 \:/\: \si{\milli\volt}$} & {$\text{U}_2 \:/\: \si{\milli\volt}$} & {$\Delta \text{U} \:/\: \si{\milli\volt}$}  & {$\text{R}_0 \:/\: \si{\milli\ohm}$} & {$\text{R}_2 \:/\: \si{\milli\ohm}$} & {$\Delta \text{R} \:/\: \si{\milli\ohm}$} \\
     \midrule
     $\text{Dy}_2 \text{O}_3$ & 3.35 & 4.55 & 1.20 & 4530 & 3275 & 1255 \\
                            & 3.40 & 4.70 & 1.30 & 4560 & 3150 & 1410 \\
                            & 3.30 & 4.60 & 1.30 & 4485 & 2785 & 1700 \\
     $\text{Nd}_2 \text{O}_3$ & 3.35 & 3.40 & 0.05 & 4375 & 3925 & 450  \\
                            & 3.30 & 3.40 & 0.10 & 4230 & 3820 & 410  \\
                            & 3.30 & 3.35 & 0.05 & 4165 & 3800 & 365  \\
     $\text{Gd}_2 \text{O}_3$ & 3.30 & 3.50 & 0.20 & 4115 & 3345 & 770  \\
                            & 3.30 & 3.55 & 0.25 & 4105 & 3360 & 745  \\
                            & 3.35 & 3.65 & 0.30 & 4250 & 3380 & 870  \\
     \bottomrule
  \end{tabular}
\end{table}

Die einzelnen Eigenschaften der jeweiligen Proben sind in Tabelle \ref{tab:Eigenschaften} dargestellt.
Der reale Querschnitt Q$_\text{real}$ wurde nach der Gleichung \eqref{eqn:qreal} errechnet.

\begin{table}
  \centering
  \caption{Physikalische Eigenschaften der Proben.}
  \label{tab:Eigenschaften}
  \begin{tabular}{l c c c c}
    \toprule
    {$\text{Name}$} & {$\text{M}_\text{p} \:/\: \si{\gram}$} & {$\text{L}_\text{p} \:/\: \si{\centi\metre}$} & {$\rho_\text{p} \:/\: \si{\gram\per\cubic\per\centi\per\metre}$} & {$\text{Q}_\text{real} \:/\: \si{\centi\metre}$} \\
    \midrule
    $\text{Dy}_2 \text{O}_3$ & 15.1  & 13   & 7.80 & 0.15 \\
    $\text{Nd}_2 \text{O}_3$ & 9     & 16   & 7.24 & 0.08 \\
    $\text{Gd}_2 \text{O}_3$ & 14.08 & 16.7 & 7.40 & 0.11 \\
  \end{tabular}
\end{table}
