\section{Auswertung}
\label{sec:Auswertung}

Zunächst wird der Acrylblock mit einer Schieblehre vermessen.
Die Messwerte für die einzelnen Löcher sind in Tabelle \ref{tab:schieblehre} dargestellt.
Die Nummern der Bohrungen können der Abbildung \ref{fig:kenblock} entnommen werden.
Die Abmessungen des Blockes lauten:

\begin{align*}
  \text{Breite} &= \SI{150.25}{\milli\metre} \\
  \text{Höhe}   &= \SI{80.45}{\milli\metre} \\
  \text{Tiefe}  &= \SI{40.00}{\milli\metre}
\end{align*}

Wichtig ist für die folgenden Veruchsteile lediglich die Höhe des Blockes.

\begin{table}
  \centering
  \caption{Abmessungen der Bohrungen innerhalb des Acrylblockes.}
  \label{tab:schieblehre}
  \begin{tabular}[t]{c c c c}
   \toprule
    {Nummer} & {$l_\text{oben}$ / $\si{\milli\metre}$} & {$l_\text{unten}$ / $\si{\milli\metre}$} &  {Dicke / $\si{\milli\metre}$} \\
     \midrule
     \csvreader[no head,
     late after line=\\,
     late after last line=\\\bottomrule]%
     {data/abmessungentab.csv}{}%
     {$\num{\csvcoli}$ & $\num{\csvcolii}$ & $\num{\csvcoliii}$ & $\num{\csvcoliv}$ }%
   \end{tabular}
 \end{table}

\FloatBarrier
\subsection{Untersuchung eines Acrylblockes mit dem A-Scan}

Die Laufzeit der Schallwellen durch den Acrylblock bis zu der Störstelle ist in Tabelle \ref{tab:ascan} dargestellt.
Für die Umrechnung der Zeiten in die Länge wird Gleichung \eqref{eqn:wegzeit} genutzt.
Die Schallgeschwindigkeit in Acryl beträgt dabei $\SI{2730}{\metre\per\second}$ \cite{acryl}.
Da die Sonde nicht unmittelbar auf dem Acrylblock ruht, sondern auf einem Film aus bidestilliertem Wasser muss aus jeder Rechnung die Zeit herausgerechnet werden, die der Ultraschall benötigt um das Koppelmittel zu durchqueren.
Dieser wurde zu $\SI{0.43}{\micro\second}$ bestimmt.
Für die Bestimmung der Dicke der Bohrungen wird erneut die Dicke des Acrylblockes benötigt.
Die Gesamtzeit, die der Ultraschall benötigt um den Block zu durchqueren beträgt $\SI{60.55}{\micro\second}$, bzw. $\SI{60.12}{\micro\second}$ unter Berücksichtigung des Kopplungsmittels.
Damit lässt sich die Höhe des Blockes auf $\SI{82.06}{\milli\metre}$ berechnen.

\begin{table}
  \centering
  \caption{Abmessungen der Bohrungen innerhalb des Acrylblockes.}
  \label{tab:ascan}
  \begin{tabular}[t]{c c c c c c}
   \toprule
    {Nummer} & {$t_\text{oben}$ / $\si{\micro\second}$} & {$l_\text{oben}$ / $\si{\milli\metre}$} & {$t_\text{unten}$ / $\si{\micro\second}$} & {$l_\text{unten}$ / $\si{\milli\metre}$} &  {Dicke / $\si{\milli\metre}$} \\
     \midrule
     \csvreader[no head,
     late after line=\\,
     late after last line=\\\bottomrule]%
     {data/ascantab.csv}{}%
     {$\num{\csvcoli}$ & $\num{\csvcolii}$ & $\num{\csvcoliii}$ & $\num{\csvcoliv}$ & $\num{\csvcolv}$ & $\num{\csvcolvi}$ }%
   \end{tabular}
 \end{table}

Das Auflösungsvermögen der Sonden wird in Tabelle \ref{tab:auflösung} dargestellt.

\begin{table}
  \centering
  \caption{Vergleich des Auflösungsvermögen der Ultraschallsonden.}
  \label{tab:auflösung}
  \begin{tabular}[t]{c c c c c c}
   \toprule
    {Nummer} & {$t_\text{oben}$ / $\si{\micro\second}$} & {$l_\text{oben}$ / $\si{\milli\metre}$} & {$t_\text{unten}$ / $\si{\micro\second}$} & {$l_\text{unten}$ / $\si{\milli\metre}$} &  {Dicke / $\si{\milli\metre}$} \\
     \midrule
     \csvreader[no head,
     late after line=\\,
     late after last line=\\\bottomrule]%
     {data/auflosungtab.csv}{}%
     {$\num{\csvcoli}$ & $\num{\csvcolii}$ & $\num{\csvcoliii}$ & $\num{\csvcoliv}$ & $\num{\csvcolv}$ & $\num{\csvcolvi}$ }%
   \end{tabular}
 \end{table}

\begin{figure}
  \centering
  \includegraphics{plot.pdf}
  \caption{Plot.}
  \label{fig:plot}
\end{figure}
