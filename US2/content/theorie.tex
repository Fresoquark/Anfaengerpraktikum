\section{Theorie}
\label{sec:Theorie}
Schall ist eine longitudinale Welle welche sich durch periodische Druckänderungen in einem Medium fortbewegt. Dabei lassen sich einer Schallwelle klassische Wellencharakteristika
zuordnen. Besonders wichtig sind dabei in der Praxis die Intensität $I$, welche die Lautstärke darstellt, die Frequenz $\omega$, welche die Tonhöhe angibt und die Ausbreitungs- oder
Schallgeschwindigkeit $c$, welche je nach Medium verschieden ist. Quantitativ lässt sich eine Schallwelle durch
\begin{equation}
  p(x,t) = p_0 + v_0 Z \cos(\omega t - k x),
\end{equation}
mit der akustischen Impedanz $Z=c\cdot\rho$, ausdrücken. Die Äbhangigkeit der Schallgeschwindigkeit vom jeweiligen Material wird für Flüssigkeiten durch folgende Gleichung beschrieben:
\begin{equation}
  c_\text{Fl} = \sqrt{\frac{1}{\kappa \rho}}.
\end{equation}
Sie ist somit von der Kompressibilität $\kappa$ und der Dichte $\rho$ abhängig.
Bei Festkörpern gilt folgender Zusammenhang:
\begin{equation}
  c_\text{Fe} = \sqrt{\frac{E}{\rho}}.
\end{equation}
Dabei ist jedoch zu beachten, dass in Festkörpern nicht nur longitudinale, sondern auch transversale Wellen auftreten, welche verschiedene Schallgeschwindigkeiten besitzen.
Die Intensität des Schalls nimmt bei der Ausbreitung in einem Medium exponentiell ab:
\begin{equation}
  I(x)= I_0 \cdot e^{\alpha}
\end{equation}
Der ebenfalls materialabhängige Absoprtionskoeffizient $\alpha$ gibt dabei an wie stark die Intensität abnimmt.
Aufgrund der vielen Materialabhängigkeiten kommt es an Grenzflächen zwischen verschiedenen Materialien zu Reflexion und Transmission.
Der Reflexionskoeffizient $R$ ist dabei von den verschiedenen Impedanzen der Materialien abhängig und gibt das Verhältnis der Intensitäten der einfallenden und reflektierten Welle an.
Bei Frequenzen überhalb des menschlichen Hörvermögens, welches Frequenzen von $\SI{16}{\hertz}$ bis
$\SI{20}{\kilo\hertz}$ umfasst, handelt es sich um Ultraschall. Dieser wird genutzt um Informationen über ein Objekt zu erhalten ohne es zu zerstören.
Dabei werden die Reflexion sowie die diversen Materialabhängigkeiten des Schalls genutzt. Durch den Zeitunterschied zwischen einer ausgesandten Welle und dem Eintreffen ihrer Reflexion
lässt sich, bei bekannter Schallgeschwindigkeit in dem zu untersuchenden Medium, die Position des Reflexionskante mit dem Weg-Zeit Gesetz
\begin{equation}
  s=\frac{1}{2}ct
  \label{eqn:wegzeit}
\end{equation}
bestimmen. Dies wird als Puls-Echo Verfahren bezeichnet und kann mit verschiedenen Scanverfahren genutzt werden. Beim Amplituden-Scan, welcher ein eindimensionales Verfahren ist, wird die
Intensität des Echos in abhängigkeit der Laufzeit gemessen. Beim Brightness-Scan wird durch Bewegen der Ultraschallsonde bei gleichzeitigem eine zweidimensionale Darstellung erzeugt.
Dabei wird die Intensität des Echos als Helligkeit dargestellt wodurch ein Schnittbild des zu untersuchenden Objektes entsteht. Beim Time-Motion-Scan wird durch eine schnelle Abtastung
eine Bildfolge aufgenommen, welche Bewegungen im beziehungsweise des zu untersuchenden Objektes sichtbar macht.
\cite{sample}
