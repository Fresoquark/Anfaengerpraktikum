\section{Durchführung}
\label{sec:Durchführung}

Das A- und B-Scan-Verfahren werden zur Untersuchung eines Acrylblockes mit unterschiedlichen Bohrungen genutzt.
Der TM-Scan zur Untersuchung eines rudimentären Herzmodells.
Der Acrylblock wird dafür zunächst vermessen.
Es werden besonders die Tiefen, bzw. Höhen der Bohrungen benötigt.
Die schematische Darstellung des Blockes ist in Abbildung \ref{fig:kenblock} dargestellt.

\begin{figure}
  \centering
  \includegraphics[width=0.7\textwidth]{images/kenblock.png}
  \caption{Darstellung des verwendeten Acrylblockes, entnommen der Versuchsanleitung \cite[5]{sample}.}
  \label{fig:kenblock}
\end{figure}

Für die Messungen werden Ultraschallsonden mit unterschiedlichen Frequenzen genutzt.
Diese sind mit einem Rechner verbunden, an denen die entsprechenden Scanverfahren dargestellt werden können.
Für den A-Scan wird der Block auf ein Papiertaschentuch gestellt und seine Oberseite mit bidestilliertem Wasser befeuchtet.
Dieses dient als Koppelmittel um den Intensitätsverlust der Schallwellen durch die Luft zu minimieren.
Für die Messung wird eine $\SI{1}{\mega\hertz}$ Sonde genutzt.
Nun wird für jedes Loch einzeln die Zeit bis zur Störstelle gemessen.
Der Block wird anschließend umgedreht.
Es wird erneut jedes Loch gemessen.
Aus beiden Daten lässt sich später die Größe der Bohrungen bestimmen.
Es muss außerdem einmal die Daten der Kopplungsschicht bestimmt werden, da diese zur genauen Berechnung berücksichtigt werden müssen.
Für die Bestimmung des Auflösungsvermögen werden die Löcher 1 und 2 jeweils mit einer $\SI{1}{\mega\hertz}$ und einer $\SI{2}{\mega\hertz}$ Sonde untersucht.

Für die Vermessung des Acrylblockes mithilfe des B-Scans wird eine $\SI{2}{\mega\hertz}$ Sonde genutzt.
Der Block wird wie oben mit bidestilliertem Wasser präpariert.
Für den B-Scan wird die Sonde anschließend langsam und gleichmäßig über den Block geführt.
Das Gleiche wird für die Unterseite wiederholt.
Aus den so entstehenden Bildern können die Abmessungen erneut bestimmt werden.

Das Herzmodell wird mittels TM-Scan untersucht.
Dafür wird das Modell so mit Wasser gefüllt, so dass die verwendete Sonde gerade die Wasseroberfläche berührt.
Zur Vorbereitung wird mit einem A-Scan die Tiefe des Wassers ermittelt.
Für den TM-Scan kann die Membran des Herzens mit einem Gummiball bewegt werden.
Das Herzvolumen kann damit erhöht werden.
Nach Start des TM-Scans wird die Membran 30 mal gepumpt.
Aus der so entstehenden Kurve kann die Herz-Frequenz und das -Volumen bestimmt werden.
