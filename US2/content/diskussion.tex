\section{Diskussion}
\label{sec:Diskussion}
Wie der Vergleich der Werte in \ref{sec:vergleich} zeigt, zeigen sowohl der A- als auch der B-Scan stark variierende Messgenauigkeiten.
Die Messwerte für Loch 1 und 2 sind dabei beim A-Scan zu vernachlässigen, da diese Löcher mit der $\SI{1}{\mega\hertz}$-Sonde nicht aufgelöst werden konnten.
Durch die $\SI{2}{\mega\hertz}$-Sonde konnten diese zwar aufgelöst, jedoch die Dicken nicht genau bestimmt werden.
Abgesehen von diesen Löchern zeigen beide Messungen größere Abweichungen bei Loch 8. Des Weiteren liegt eine sehr starke Abweichung beim B-Scan für Loch 4 vor, welche vielleicht
auf einen Ablese- oder Messfehler zurück geht. Faktoren die die Messungen beeinträchtigt haben könnten sind zum Beispiel Unregelmäßigkeiten im untersuchten Acrylblock.
Außerdem wurde in der Auswertung nicht die transversale Ausbreitung der Schallwellen im Festkörper berücksichtigt. Bei der Untersuchung des Herzmodells kann kein Vergleich zu
Literatur- oder Referenzwerten gezogen werden. Unter Berücksichtigung der erhaltenen Messwerte sind die Scanverfahren in der Ultraschalltechnik nicht geeignet um das innere von
Objekten hochgenau zu messen, jedoch hinreichend genau für grobe Messungen und Lagebestimmungen. Für eine qualifizertere Bewertung der Scanverfahren müsste eine erneute Messung unter
Auschluss der zuvor genannten Fehlerquellen durchgeführt werden.
