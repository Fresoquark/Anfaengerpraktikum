\section{Theorie}
\label{sec:Theorie}

Die Reichweite von $\alpha$-Strahlung lässt sich über den Energieverlust der Teilchen berechnen.
Die Hauptverlustquellen für die $\alpha$-Teilchen stellen dabei Ionisationsprozesse, sowie die Anregung und Dissoziation von Molekülen dar.
Der Energieverlust der Teilchen ist dabei proportional zu der Energie der $\alpha$-Teilchen und die Dichte des durchlaufenden Materials.
Bei hohen Energien wird die Bethe-Bloch-Gleichung für den Energieverlust benötigt.
Ist die Energie der $\alpha$-Teilchen, wie in diesem Versuch, allerdings zu gering, so muss die Reichweite durch empirische Kurven ermittelt werden.
Grund hierfür ist, dass bei niedrigeren Energien vermehrt Ladungsaustauschprozesse auftreten und die Bethe-Bloch-Gleichung ihre Gültigkeit verliert.
Es handelt sich daher um die mittlere Reichweite, d.h. die Reichweite, die die Hälfte der $\alpha$-Teilchen noch erreichen.
Bei Energien der $\alpha$-Strahlung in Luft unter $\SI{2.5}{\mega\electronvolt}$ gilt folgende Beziehung:

\begin{equation}
  R_m = 3.1 \cdot E_{\alpha}^{\frac{3}{2}}
\end{equation}

Hält man die Temperatur und die Temperatur des Gases konstant, so ergibt sich für die Energie ein Ausdruck, welcher proportional zum Druck p ist.
Man erhält dann für die effektive Länge für $p_0 = \SI{1013}{\milli\bar}$:

\begin{equation}
  x = x_0 \frac{p}{p_0}
\end{equation}



\cite{1}
