\section{Auswertung}
\label{sec:Auswertung}
\subsection{Bestimmung der  mittleren Reichweite von \texorpdfstring{$\alpha$}{alpha}-Strahlung}

Um die Reichweite der $\alpha$-Strahlung bestimmen zu können wird zunächst die Energie der $\alpha$-Teilchen benötigt.
Diese können aus der Position der Kanäle des Energiemaximums bestimmt werden.
Die Energie ist bei $\SI{0}{\milli\bar}$ mit $\SI{4.0}{\mega\electronvolt}$ maximal. Durch Anwendung des Dreisatzes lässt sich folgende Gleichung herleiten:
\begin{equation*}
  E = 4 \cdot \frac{k_p}{k_0} \cdot \si{\mega\electronvolt} \: \: \: \text{mit} \: k \equiv \text{Kanalnummer}
\end{equation*}
Damit wird die Energie der $\alpha$-Teilchen bestimmt.
Die effektive Länge wird mit Gleichung \ref{eqn:xeff} bestimmt.
Die Ergebnisse der Messungen für $\SI{25}{\milli\metre}$ können Tabelle \ref{tab:25mm}, die für $\SI{5}{\milli\metre}$ Tabelle \ref{tab:5mm} entnommen werden.
\FloatBarrier
\begin{table}
  \centering
  \caption{Messwerte für 25 Millimeter Abstand}
  \label{tab:25mm}
  \begin{tabular}{c c c c c}
    \toprule
     Druck / \si{\milli\bar} & Effektive Länge / \si{\milli\metre} & Counts pro 120 \si{\second} & Kanalnummer & Energie / \si{\mega\electronvolt}\\
    \midrule
    0   &  0,00 & 75000 & 1199 & 4,00 \\
    50  &  1,23 & 74024 & 1127 & 3,76 \\
    120 &  2,96 & 72906 & 1078 & 3,60 \\
    160 &  3,95 & 72697 & 1035 & 3,45 \\
    220 &  5,43 & 71795 &  985 & 3,29 \\
    300 &  7,40 & 70766 &  925 & 3,09 \\
    400 &  9,87 & 69496 &  846 & 2,82 \\
    500 & 12,34 & 67839 &  770 & 2,57 \\
    600 & 14,81 & 67532 &  703 & 2,35 \\
    700 & 17,28 & 64271 &  594 & 1,98 \\
    800 & 19,74 & 55844 &  224 & 0,75 \\
    900 & 22,21 & 34068 &  352 & 1,17 \\
    1000& 24,68 & 8969  &  347 & 1,16 \\
    \bottomrule
  \end{tabular}
\end{table}
\begin{table}
  \centering
  \caption{5 Millimeter Abstand}
  \label{tab:5mm}
  \begin{tabular}{c c c c c}
    \toprule
     Druck / \si{\milli\bar} & Effektive Länge / \si{\milli\metre} & Counts pro 120 \si{\second} & Kanalnummer & Energie / \si{\mega\electronvolt}\\
    \midrule
    0   & 0,00 & 284417 & 1150 & 4,00 \\
    50  & 0,25 & 280453 & 1101 & 3,83 \\
    100 & 0,49 & 281443 & 1090 & 3,79 \\
    150 & 0,74 & 278470 & 1057 & 3,68 \\
    200 & 0,99 & 280453 & 1070 & 3,72 \\
    300 & 1,48 & 277480 & 1038 & 3,61 \\
    400 & 1,97 & 275508 & 1009 & 3,51 \\
    500 & 2,47 & 273516 &  990 & 3,44 \\
    600 & 2,96 & 271534 &  945 & 3,29 \\
    700 & 3,46 & 269552 &  930 & 3,23 \\
    800 & 3,95 & 270550 &  921 & 3,20 \\
    900 & 4,44 & 268561 &  881 & 3,06 \\
    1000& 4,94 & 276772 &  921 & 3,20 \\
    \bottomrule
  \end{tabular}
\end{table}
\FloatBarrier
Nun wird die Zählrate graphisch als Funktion der effektiven Länge aufgetragen.
Dies ist in den Abbildungen \ref{fig:25mmcounts} und \ref{fig:5mmcounts} dargestellt.
\FloatBarrier
\begin{figure}
    \centering
    \includegraphics[scale=0.8]{build/25mmcounts.pdf}
    \caption{Die graphische Darstellung der Zählrate gegen die effektive Länge aufgetragen für $\SI{25}{\milli\metre}$.}
    \label{fig:25mmcounts}
\end{figure}

\begin{figure}
    \centering
    \includegraphics[scale=0.8]{build/5mmcounts.pdf}
  \caption{Die graphische Darstellung der Zählrate gegen die effektive Länge aufgetragen für $\SI{5}{\milli\metre}$.}
  \label{fig:5mmcounts}
\end{figure}
\FloatBarrier
Zur Bestimmung der mittleren Reichweite wird der linear abfallende Teil der Daten mit $y=ax+b$ gefittet.
So ergeben sich die Parameter:
\begin{align*}
  a&=\SI{-9496.88\pm388.70}{\per\milli\metre}\\
  b&=243897.83\pm8668.86
\end{align*}
Nun wird die so erhaltene Funktion nach x umgestellt und für y die Hälfte der maximalen Counts, welche aus Tabelle \ref{tab:25mm}
entnommen werden, eingesetzt.
\begin{equation*}
  R_m = x = \frac{\frac{y}{2}-b}{a}
\end{equation*}
Der Fehler wird mit
\begin{equation*}
  \Delta R_m = \sqrt{\frac{\Delta_{b}^{2}}{a^{2}} + \frac{\Delta_{y}^{2}}{4 a^{2}} + \frac{\Delta_{a}^{2}}{a^{4}} \left(- b + \frac{y}{2}\right)^{2}}
\end{equation*}
berechnet.
So ergibt sich für die mittlere Reichweite
\begin{equation*}
  R_m = \SI{21.7\pm1.3}{\milli\metre}
\end{equation*}.
Die dazu gehörige Energie wird mit der durch Umstellen von \eqref{eqn:rm} hergeleiteten Formel
\begin{equation*}
  E_{\alpha}=(\frac{R_m}{3.1})^{\frac{2}{3}}
\end{equation*}
und der Fehler mit
\begin{equation*}
  \Delta E_{\alpha} = \frac{2}{3 \cdot 3.1^{\frac{2}{3}}} \sqrt{\frac{\Delta_{R_{m}}^{2}}{R_{m}^{\frac{2}{3}}}}
\end{equation*}
berechnet.
So ergibt sich eine Energie von
\begin{equation*}
  E_{\alpha}=\SI{3.66\pm0.14}{\mega\electronvolt}
\end{equation*}.
Bei Abbildung \ref{fig:5mmcounts} kann so keine mittlere Reichweite bestimmt werden, da die Counts nicht auf das halbe Niveau absinken.

\subsection{Bestimmung des Energieverlustes von \texorpdfstring{$\alpha$}{alpha}-Strahlung}



Um den Energieverlust bestimmen zu können, wird die Energie gegen die effektive Länge aufgetragen und mit der Funktion
\begin{equation*}
  y= ax+b
\end{equation*}
 mit Python \cite{numpy} \cite{scipy} linear gefittet.
 Diese Kurven sind in den Abbildungen \ref{fig:25mmenergie} und \ref{fig:5mmenergie} zu finden.
 So wurden die Parameter zu
 \begin{align*}
   a_5 &= \SI{-0.17\pm0.01}{\mega\electronvolt\per\milli\metre}\\
   b_5 &= \SI{3.87\pm0.04}{\mega\electronvolt}\\
   a_{25} &= \SI{-0.13\pm0.01}{\mega\electronvolt\per\milli\metre}\\
   b_{25} &= \SI{3.99\pm0.12}{\mega\electronvolt}
 \end{align*}
 bestimmt.
\FloatBarrier
 \begin{figure}
     \centering
     \includegraphics[scale=0.8]{build/25mmenergie.pdf}
     \caption{Die graphische Darstellung der Energie in Abhängigkeit der effektive Länge aufgetragen für $\SI{25}{\milli\metre}$.}
     \label{fig:25mmenergie}
 \end{figure}

 \begin{figure}
     \centering
     \includegraphics[scale=0.8]{build/5mmenergie.pdf}
    \caption{Die graphische Darstellung der Energie in Abhängigkeit der effektive Länge aufgetragen für $\SI{5}{\milli\metre}$.}
     \label{fig:5mmenergie}
 \end{figure}
 \FloatBarrier
 Da ein linearer Zusammenhang zwischen Energie und effektiver Länge besteht, gilt folgender Zusammenhang:
 \begin{equation}
   - \frac{\symup{d} E}{\symup{d}x} = a .
 \end{equation}
 So wird der jeweilige Energieverlust zu
\begin{align*}
   - \frac{\symup{d} E}{\symup{d}x} = (125,9 \pm 9,1) \cdot \si{\kilo\electronvolt\per\milli\metre}
  \intertext{für $\SI{25}{\milli\metre}$ und}\\
   - \frac{\symup{d} E}{\symup{d}x} = (168,9 \pm 13,1) \cdot \si{\kilo\electronvolt\per\milli\metre}
  \intertext{für $\SI{5}{\milli\metre}$ bestimmt.}
\end{align*}

\newpage
\subsection{Bestimmung der Statistik des radioaktiven Zerfalls}
Die Messwerte sind in Tabelle \ref{tab:hist} zu finden.
Zur Bestimmung der Statistik des radioaktiven Zerfalls wird mit den Messwerten ein Histogramm erstellt. Dafür werden die gemessenen Counts in zehn
gleich große Bins eingeordnet diese gegen die relative Häufigkeit aufgetragen. Dies ist in Abbildung \ref{fig:hist} zu sehen.
Das Histogramm ähnelt, wie durch die zusätzlich geplotteten Verteilungen deutlich wird, keiner gängigen statistischen Verteilung.
Der Mittelwert der Counts wird mit
\begin{align*}
  \overline{x} &= \frac{1}{N} \sum_{i=1}^N x_i
  \intertext{und} \\
  \Delta \overline{x} &= \frac{1}{\sqrt{N}} \sqrt{\frac{1}{N-1} \sum_{i=1}^N (x_i - \overline{x})^2}
\end{align*}
zu $ 6238 \pm 215.30$ bestimmt.
Die Varianz ${\Delta\overline{x}}^2$ beträgt $46352.65$.
\FloatBarrier
\begin{table}
  \centering
  \caption{Messwerte für 100 Messungen}
  \label{tab:hist}
  \begin{tabular}{c c c c c}
    \toprule
  \multicolumn{5}{c}{Counts pro 10 \si{\second}} \\
    \midrule
    6418 & 5909 & 6062 & 6516 & 6080 \\
    6548 & 6473 & 6183 & 6495 & 5897 \\
    6614 & 6022 & 6053 & 6142 & 5931 \\
    6580 & 6193 & 6323 & 6127 & 6545 \\
    6148 & 6541 & 6078 & 6245 & 6122 \\
    6393 & 6037 & 6042 & 6541 & 6082 \\
    5977 & 6478 & 6358 & 5872 & 6063 \\
    5954 & 6306 & 6313 & 6135 & 6410 \\
    6556 & 6469 & 6101 & 6415 & 6466 \\
    5948 & 6114 & 6005 & 6572 & 6047 \\
    6096 & 5916 & 6542 & 6520 & 6340 \\
    6356 & 6429 & 6503 & 6303 & 6035 \\
    6039 & 6530 & 6038 & 6367 & 6121 \\
    6369 & 6338 & 6012 & 6525 & 6382 \\
    6029 & 6210 & 5907 & 6469 & 6492 \\
    6103 & 6111 & 6238 & 6301 & 6414 \\
    5900 & 6167 & 5791 & 6367 & 6245 \\
    6034 & 6186 & 5940 & 6639 & 6118 \\
    6367 & 6376 & 6269 & 6168 & 6194 \\
    6508 & 6025 & 6267 & 6392 & 5932 \\
    \bottomrule
  \end{tabular}
\end{table}
\begin{figure}
    \centering
    \includegraphics[scale=0.8]{build/histo.pdf}
    \caption{Histogramm des radioaktiven Zerfalls mit Normal- und Laplaceverteilung.}
    \label{fig:hist}
\end{figure}
