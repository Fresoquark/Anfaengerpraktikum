\section{Auswertung}
\label{sec:Auswertung}
\subsection{25 Millimeter Abstand}
Die für einen Abstand von 25 Millimetern zwischen Detektor und Präparat gemessenen Werte befinden sich in Tabelle \ref{tab:25mm}.
Die effektive Länge wird dabei mit Gleichung \ref{eqn:xeff} und die Energie mit
\begin{equation*}
  E = 4 \cdot \frac{k}{1199} \cdot \si{\mega\electronvolt} \: \: \: \text{mit} \: k \equiv \text{Kanalnummer}
\end{equation*}
bestimmt.
\begin{table}
  \centering
  \caption{25 Millimeter Abstand}
  \label{tab:25mm}
  \begin{tabular}{c c c c c}
    \toprule
     Druck \,/\, \si{\milli\bar} & Effektive Länge \,/\, \si{\milli\metre} & Counts pro 120 \si{\second} & Kanalnummer & Energie \,/\, \si{\mega\electronvolt}\\
    \midrule
    0   &  0,00 & 75000 & 1199 & 4,00 \\
    50  &  1,23 & 74024 & 1127 & 3,76 \\
    120 &  2,96 & 72906 & 1078 & 3,60 \\
    160 &  3,95 & 72697 & 1035 & 3,45 \\
    220 &  5,43 & 71795 &  985 & 3,29 \\
    300 &  7,40 & 70766 &  925 & 3,09 \\
    400 &  9,87 & 69496 &  846 & 2,82 \\
    500 & 12,34 & 67839 &  770 & 2,57 \\
    600 & 14,81 & 67532 &  703 & 2,35 \\
    700 & 17,28 & 64271 &  594 & 1,98 \\
    800 & 19,74 & 55844 &  224 & 0,75 \\
    900 & 22,21 & 34068 &  352 & 1,17 \\
    1000& 24,68 & 8969  &  347 & 1,16 \\
    \bottomrule
  \end{tabular}
\end{table}
\subsubsection{Bestimmung der mittleren Reichweite}
Zur Bestimmung der mittleren Reichweite der $\alpha$-Strahlung werden zunächst die Counts pro 120 Sekunden gegen die effektive Länge in einem Diagramm abgetragen (Abbildung \ref{fig:25counts}).
\begin{figure}
  \centering
  \includegraphics{25mmcounts.pdf}
  \caption{Plot der Counts in Abhängigkeit von der effektiven Länge.}
  \label{fig:25counts}
\end{figure}
In dem Plot ist zu sehen, dass die Counts bis circa $17 \si{\milli\metre}$ relativ konstant sind und dann stark abfallen. Als mittlere Reichweite wird nun die effektive Länge
gewählt, bei welcher die Counts auf die Hälfte des konstanten Niveaus gesunken sind. Durch Ablesen aus Abbildung \ref{fig25counts} wird die mittlere Reichweite so zu
\begin{equation*}
  R_m = 22 \: \si{\milli\metre}
 \end{equation*}
bestimmt. Durch Umstellen von Gleichung \ref{eqn:rm} nach der Energie $E_{\alpha}$ und Einsetzen der zuvor ermittelten mittleren Reichweite ergibt sich eine
Energie von
\begin{equation*}
  E_{\alpha} = 3,69 \, \si{\mega\electronvolt}.
\end{equation*}
\subsubsection{Bestimmung des Energieverlustes}
Zur Bestimmung des Energieverlustes wird die Energie gegen die effektive Länge in einem Diagramm abgetragen und mit der Funktion
\begin{equation*}
  y= ax+b
\end{equation*}
 mit Python \cite{numpy} \cite{scipy} linear gefittet (Abbildung \ref{25energie}).
 \begin{figure}
   \centering
   \includegraphics{25mmenergie.pdf}
   \caption{Linearer Fit der Energie in Abhängigkeit der effektiven Länge.\cite{matplotlib}}
   \label{fig:25energie}
 \end{figure}
 Da ein linearer Zusammenhang zwischen Energie und effektiver Länge besteht, gilt folgender Zusammenhang:
 \begin{equation}
   - \frac{\symup{d} E}{\symup{d}x} = a .
   \ref{eqn:dedx}
 \end{equation}
 So wurde der Energieverlust zu
 \begin{equation*}
   - \frac{\symup{d} E}{\symup{d}x} = (125,9 \pm 9,1) \cdot \si{\kilo\electronvolt\per\milli\metre}
 \end{equation*}
 bestimmt.
\subsection{5 Millimeter Abstand}
Die für einen Abstand von 5 Millimetern zwischen Detektor und Präparat gemessenen Werte befinden sich in Tabelle \ref{tab:5mm}.
Die effektive Länge wird dabei wieder mit Gleichung \ref{eqn:xeff} und die Energie mit
\begin{equation*}
  E = 4 \cdot \frac{k}{1150} \cdot \si{\mega\electronvolt} \: \: \: \text{mit} \: k \equiv \text{Kanalnummer}
\end{equation*}
bestimmt.
\begin{table}
  \centering
  \caption{5 Millimeter Abstand}
  \label{tab:5mm}
  \begin{tabular}{c c c c c}
    \toprule
     Druck \,/\, \si{\milli\bar} & Effektive Länge \,/\, \si{\milli\metre} & Counts pro 120 \si{\second} & Kanalnummer & Energie \,/\, \si{\mega\electronvolt}\\
    \midrule
    0   & 0,00 & 284417 & 1150 & 4,00 \\
    50  & 0,25 & 280453 & 1101 & 3,83 \\
    100 & 0,49 & 281443 & 1090 & 3,79 \\
    150 & 0,74 & 278470 & 1057 & 3,68 \\
    200 & 0,99 & 280453 & 1070 & 3,72 \\
    300 & 1,48 & 277480 & 1038 & 3,61 \\
    400 & 1,97 & 275508 & 1009 & 3,51 \\
    500 & 2,47 & 273516 &  990 & 3,44 \\
    600 & 2,96 & 271534 &  945 & 3,29 \\
    700 & 3,46 & 269552 &  930 & 3,23 \\
    800 & 3,95 & 270550 &  921 & 3,20 \\
    900 & 4,44 & 268561 &  881 & 3,06 \\
    1000& 4,94 & 276772 &  921 & 3,20 \\
    \bottomrule
  \end{tabular}
\end{table}
\subsubsection{Bestimmung der mittleren Reichweite}
