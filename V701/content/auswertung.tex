\section{Auswertung}
\label{sec:Auswertung}
\subsection{Bestimmung der Reichweite von \texorpdfstring{$\alpha$}{alpha}-Strahlung}

Um die Reichweite der $\alpha$-Strahlung bestimmen zu können wird zunächst die Energie der $\alpha$-Teilchen benötigt.
Diese können aus der Position der Kanäle des Energiemaximums bestimmt werden.
Die Energie ist bei $\SI{0}{\milli\bar}$ mit $\SI{4.0}{\mega\electronvolt}$ maximal. Durch Anwendung des Dreisatzes lässt sich folgende Gleichung herleiten:
\begin{equation*}
  E = 4 \cdot \frac{k_p}{k_0} \cdot \si{\mega\electronvolt} \: \: \: \text{mit} \: k \equiv \text{Kanalnummer}
\end{equation*}
Damit wird die Energie der $\alpha$-Teilchen bestimmt.
Die effektive Länge wird mit Gleichung \ref{eqn:xeff} bestimmt.
Die Ergebnisse der Messungen für $\SI{25}{\milli\metre}$ können Tabelle \ref{tab:25mm}, die für $\SI{5}{\milli\metre}$ Tabelle \ref{tab:5mm} entnommen werden.

\begin{table}
  \centering
  \caption{Messwerte für 25 Millimeter Abstand}
  \label{tab:25mm}
  \begin{tabular}{c c c c c}
    \toprule
     Druck \,/\, \si{\milli\bar} & Effektive Länge \,/\, \si{\milli\metre} & Counts pro 120 \si{\second} & Kanalnummer & Energie \,/\, \si{\mega\electronvolt}\\
    \midrule
    0   &  0,00 & 75000 & 1199 & 4,00 \\
    50  &  1,23 & 74024 & 1127 & 3,76 \\
    120 &  2,96 & 72906 & 1078 & 3,60 \\
    160 &  3,95 & 72697 & 1035 & 3,45 \\
    220 &  5,43 & 71795 &  985 & 3,29 \\
    300 &  7,40 & 70766 &  925 & 3,09 \\
    400 &  9,87 & 69496 &  846 & 2,82 \\
    500 & 12,34 & 67839 &  770 & 2,57 \\
    600 & 14,81 & 67532 &  703 & 2,35 \\
    700 & 17,28 & 64271 &  594 & 1,98 \\
    800 & 19,74 & 55844 &  224 & 0,75 \\
    900 & 22,21 & 34068 &  352 & 1,17 \\
    1000& 24,68 & 8969  &  347 & 1,16 \\
    \bottomrule
  \end{tabular}
\end{table}
\begin{table}
  \centering
  \caption{5 Millimeter Abstand}
  \label{tab:5mm}
  \begin{tabular}{c c c c c}
    \toprule
     Druck \,/\, \si{\milli\bar} & Effektive Länge \,/\, \si{\milli\metre} & Counts pro 120 \si{\second} & Kanalnummer & Energie \,/\, \si{\mega\electronvolt}\\
    \midrule
    0   & 0,00 & 284417 & 1150 & 4,00 \\
    50  & 0,25 & 280453 & 1101 & 3,83 \\
    100 & 0,49 & 281443 & 1090 & 3,79 \\
    150 & 0,74 & 278470 & 1057 & 3,68 \\
    200 & 0,99 & 280453 & 1070 & 3,72 \\
    300 & 1,48 & 277480 & 1038 & 3,61 \\
    400 & 1,97 & 275508 & 1009 & 3,51 \\
    500 & 2,47 & 273516 &  990 & 3,44 \\
    600 & 2,96 & 271534 &  945 & 3,29 \\
    700 & 3,46 & 269552 &  930 & 3,23 \\
    800 & 3,95 & 270550 &  921 & 3,20 \\
    900 & 4,44 & 268561 &  881 & 3,06 \\
    1000& 4,94 & 276772 &  921 & 3,20 \\
    \bottomrule
  \end{tabular}
\end{table}
Nun wird die Zählrate graphisch als Funktion der effektiven Länge aufgetragen.
Dies ist in den Abbildungen \ref{fig:25mmcounts} und \ref{fig:5mmcounts} dargestellt.
Als mittlere Reichweite wird nun die effektive Länge gewählt, bei welcher die Counts auf die Hälfte des konstanten Niveaus gesunken sind.
Dies entspricht bei Abbildung \ref{fig:25mmcounts} einer mittleren Reichweite von
\begin{equation*}
  R_m = 22 \: \si{\milli\metre}.
 \end{equation*}
Durch Umstellen von Gleichung \ref{eqn:rm} nach der Energie $E_{\alpha}$ und Einsetzen der zuvor ermittelten mittleren Reichweite ergibt sich eine
Energie von
\begin{equation*}
  E_{\alpha} = 3,69 \, \si{\mega\electronvolt}.
\end{equation*}
Bei Abbildung \ref{fig:5mmcounts} kann so keine mittlere Reichweite bestimmt werden, da die Counts nicht auf das halbe Niveau absinken.

\begin{figure}
    \centering
    \includegraphics{build/25mmcounts.pdf}
    \caption{Die graphische Darstellung der Zählrate gegen die effektive Länge aufgetragen für $\SI{25}{\milli\metre}$}
    \label{fig:25mmcounts}
\end{figure}

\begin{figure}
    \centering
    \includegraphics{build/5mmcounts.pdf}
    \label{fig:5mmcounts}
  \caption{Die graphische Darstellung der Zählrate gegen die effektive Länge aufgetragen für $\SI{5}{\milli\metre}$}
\end{figure}

Um den Energieverlust bestimmen zu können, wird die Energie gegen die effektive Länge aufgetragen und mit der Funktion
\begin{equation*}
  y= ax+b
\end{equation*}
 mit Python \cite{numpy} \cite{scipy} linear gefittet.
 Diese Kurven sind in den Abbildungen \ref{fig:25mmenergie} und \ref{fig:5mmenergie} zu finden.
 Da ein linearer Zusammenhang zwischen Energie und effektiver Länge besteht, gilt folgender Zusammenhang:
 \begin{equation}
   - \frac{\symup{d} E}{\symup{d}x} = a .
 \end{equation}
 So wird der jeweilige Energieverlust zu
\begin{align*}
   - \frac{\symup{d} E}{\symup{d}x} = (125,9 \pm 9,1) \cdot \si{\kilo\electronvolt\per\milli\metre}
  \intertext{für $\SI{25}{\milli\metre}$ und}\\
   - \frac{\symup{d} E}{\symup{d}x} = (168,9 \pm 13,1) \cdot \si{\kilo\electronvolt\per\milli\metre}
     \intertext{für $\SI{5}{\milli\metre}$}
\end{align*}
bestimmt.


\begin{figure}
    \centering
    \includegraphics{build/25mmenergie.pdf}
    \caption{Die graphische Darstellung der Energie in Abhängigkeit der effektive Länge aufgetragen für $\SI{25}{\milli\metre}$}
    \label{fig:25mmenergie}
\end{figure}

\begin{figure}
    \centering
    \includegraphics{build/5mmenergie.pdf}
    \label{fig:5mmenergie}
  \caption{Die graphische Darstellung der Energie in Abhängigkeit der effektive Länge aufgetragen für $\SI{5}{\milli\metre}$}
\end{figure}
