\section{Auswertung}
\label{sec:Auswertung}

\subsection{Bestimmung der Reichweite von \texorpdfstring{$\alpha$}{alpha}-Strahlung}

Um die Reichweite der \alpha-Strahlung bestimmen zu können wird zunächst die Energie der $\alpha$-Teilchen benötigt.
Diese können aus der Position des Channels in Relation zur Position des Channels des Ursprungsmaximum im Vakuum bestimmt werden.
Es ergibt sich ein linearer Zusammenhang zwischen dem Druck und den Energiemaxima.
Die diskreten Energie der $\alpha$-Teilchen für bestimmte Drücke kann durch die folgende Überlegung ermittelt werden.
Die Energie ist bei $\SI{0}{\milli\bar}$ mit $\SI{4.0}{\mega\electronvolt}$ maximal.
Mithilfe des Dreisatzes kann aus der Position des Energiemaximums im Vakuum der lineare Zusammenhang zwischen Druck und Energie hergestellt werden.
Die Ergebnisse der Messungen für $\SI{25}{\milli\metre}$ können Tabelle \ref{tab:25mm}, die für $\SI{5}{\milli\metre}$ Tabelle \ref{tab:5mm} entnommen werden.

\begin{table}
  \centering
  \caption{Messergebnisse für einen Abstand von $\SI{25}{\milli\metre}$}
  \label{tab:25mm}
  \sisetup{table-format=5.0}
  \begin{tabular}{S S [table-format=4.0]  S [table-format=5.0] S [table-format=1.2]}
    \toprule
    {$p \:/\: \si{\milli\bar}$} & {Position des Energiemaximums} & {$\text{Counts} \:/\: \SI{120}{\second}$} & {$E \:/\: \si{\mega\electronvolt}$} \\
    \midrule
    0    & 1199 & 75000 & 4.00 \\
    50   & 1127 & 74024 & 3.76 \\
    120  & 1078 & 72906 & 3.60 \\
    160  & 1035 & 72697 & 3.45 \\
    220  & 985  & 71795 & 3.29 \\
    300  & 925  & 70766 & 3.09 \\
    400  & 846  & 69496 & 2.82 \\
    500  & 770  & 67839 & 2.57 \\
    600  & 703  & 67532 & 2.35 \\
    700  & 594  & 64271 & 1.98 \\
    800  & 224  & 55844 & 0.75 \\
    900  & 352  & 34068 & 1.17 \\
    1000 & 347  & 8969  & 1.16 \\
  \end{tabular}
\end{table}

\begin{table}
  \centering
  \caption{Messergebnisse für einen Abstand von $\SI{5}{\milli\metre}$}
  \label{tab:5mm}
  \sisetup{table-format=6.0}
  \begin{tabular}{S S [table-format=4.0]  S [table-format=6.0] S [table-format=1.2]}
    \toprule
    {$p \:/\: \si{\milli\bar}$} & {Position des Energiemaximums} & {$\text{Counts} \:/\: \SI{120}{\second}$} & {$E \:/\: \si{\mega\electronvolt}$} \\
    \midrule
    0    & 1150 & 284417 & 4.00 \\
    50   & 1101 & 280453 & 3.83 \\
    120  & 1090 & 281443 & 3.79 \\
    160  & 1057 & 278470 & 3.68 \\
    220  & 1070 & 280453 & 3.72 \\
    300  & 1038 & 277480 & 3.61 \\
    400  & 1009 & 275508 & 3.51 \\
    500  & 990  & 273516 & 3.44 \\
    600  & 945  & 271534 & 3.29 \\
    700  & 930  & 269552 & 3.23 \\
    800  & 921  & 270550 & 3.20 \\
    900  & 881  & 268561 & 3.06 \\
    1000 & 921  & 276772 & 3.20 \\
  \end{tabular}
\end{table}

Nun wird die Zählrate als Funktion der effektiven Länge aufgetragen.
\begin{figure}
    \centering
    \includegraphics{build/25mmcounts.pdf}
    \caption{Die graphische Darstellung der Zählrate gegen die effektive Länge aufgetragen für $\SI{25}{\milli\metre}$}
    \label{fig:25mmcounts}
\end{figure}

\begin{figure}
    \centering
    \includegraphics{build/5mmcounts.pdf}
    \caption{$\SI{5}{\milli\metre}$}
    \label{fig:5mmcounts}
  \caption{Die graphische Darstellung der Zählrate gegen die effektive Länge aufgetragen für $\SI{5}{\milli\metre}$}
\end{figure}




\begin{figure}
  \centering
  \includegraphics{plot.pdf}
  \caption{Plot.}
  \label{fig:plot}
\end{figure}
