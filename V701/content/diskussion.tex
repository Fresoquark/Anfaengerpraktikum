\section{Diskussion}
\label{sec:Diskussion}
Die ermittelten Werte sind:
\begin{align*}
  R_{m,25 \si{\milli\metre}} &= \SI{21.7\pm1.3}{\milli\metre} \\
  E_{\alpha , 25 \si{\milli\metre}} &= \SI{3.66\pm0.14}{\mega\electronvolt} \\
  - \frac{\symup{d} E_{25 \si{\milli\metre}}}{\symup{d}x} &= (125,9 \pm 9,1) \cdot \si{\kilo\electronvolt\per\milli\metre} \\
  - \frac{\symup{d} E_{5 \si{\milli\metre}}}{\symup{d}x} &= (168,9 \pm 13,1) \cdot \si{\kilo\electronvolt\per\milli\metre} \\
  \overline{x_{\text{Counts}}} &= 6238 \pm 215.30 \\
  {\Delta\overline{x}}^2 &= 46352.65
\end{align*}
Da für den Abstand von $5 \si{\milli\metre}$ keine mittlere Reichweite bestimmt werden konnte, ist ein Vergleich mit der mittleren Reichweite und Energie von der
$25 \si{\milli\metre}$-Messung nicht möglich. Die mittlere Reichweite konnte nicht bestimmt werden, da das Präparat zu nah am Detektor war,
wodurch der Detektor wahrscheinlich nicht genug Zeit zum Abklingen hatte, und somit falsche Countzahlen gemessen hat.
Die bestimmte Energie lässt sich jedoch mit der Energie im Vakuum, welche aus \cite{1} entnommen wurde, vergleichen.
Sie beträgt $4 \si{\mega\electronvolt}$ und die Abweichung der bestimmten Energie $ E_{\alpha , 25 \si{\milli\metre}} $ beträgt $7,75 \%$.
Diese Abweichung liegt im Rahmen der Messungenauigkeit.
Beim Vergleich des Energieverlustes ist zu beobachten, dass der Energieverlust für die kürzere Distanz von $5 \si{\milli\metre}$ um $25,46 \%$ größer ist als
bei $25 \si{\milli\metre}$. Diese Abweichung geht warscheinlich auf die kleinere Energie der Teilchen, welche durch die längere Strecke und die damit verbundene geringere
Energieabgabe durch Ladungsaustauschprozesse bedingt ist, zurück. Bei der Messung der Statistik des radioaktiven Zerfalls lies sich im Histogramm keine gängige stochastische
Verteilung erkennen, was warscheinlich auf eine zu geringe Anzahl an Messungen zurückgeht. Dies bestätigt sich auch in Anbetracht dessen, dass die zwei Bins in welchen
sich der Mittelwert mitsamt der Standardabweichung befinden zwar eine geringe Häufigkeit aufweisen, die Bins direkt neben diesem Intervall jedoch eine in etwa gleich hohe
Häufigkeit aufweisen. Zur genaueren Bestimmung der Statistik sollten jedoch mehr Messwerte aufgenommen werden.
