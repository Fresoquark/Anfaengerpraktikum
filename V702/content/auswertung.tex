\section{Auswertung}
\label{sec:Auswertung}

\subsection{Halbwertszeiten von Silber-108 und Silber-110}
\label{sec:Silberhalbwertszeit}

Aus den Messdaten wird die Halbwertszeit der zwei Silberisotope berechnet.
Die Probe besteht aus den zwei Isotopen $\ce{^{108}Ag}$ und $\ce{^{110}Ag}$.
$\ce{^{110}Ag}$ zerfällt wesentlich schneller als $\ce{^{108}Ag}$.
Dies stellt ein Problem dar, da beide Zerfallsprozesse simultan ablaufen, wie in der Theorie \ref{sec:Theorie} beschrieben.
Aus den gewonnenen Messdaten müssen somit die beiden Zerfallsprozesse separiert werden.

Die Messwerte befinden sich in Tabelle \ref{tab:silber}.
$N_0$ stellt die ursprünglich gemessene Aktivität im Zeitintervall $\Delta t = \SI{10}{\second}$ dar.
Für die echte Aktivität wird der Nulleffekt subtrahiert.

\begin{equation}
  N = N_0 - N_{\text{Nulleffekt}}
  \label{eqn:nulleffekt}
\end{equation}

Der Nulleffekt wird mit $N_\text{Nulleffekt} = 223$ Counts innerhalb von $\SI{900}{\second}$ gemessen.
Es wird also bei einem Zeitintervall von $\SI{10}{\second}$ konstant eine Aktivität von $2.5$ von $N_0$ subtrahiert.
Die Fehler berechnen sich nach dem Fehler einer Poisson-Verteilung:

\begin{equation}
  \Delta N = \sqrt{N}.
  \label{eqn:Poisson}
\end{equation}

Es handelt sich um eine halblogarithmische Ausgleichsrechnung, somit werden die Werte für $N$ noch logarithmiert.
Die entsprechenden logarithmischen Fehler von $N$ ergeben sich nach

\begin{equation}
  \Delta^{+} \ln(N) = \ln(N + \Delta N) - \ln(N)
  \label{eqn:oben}
\end{equation}

für den oberen, und nach

\begin{equation}
  \Delta^{-} \ln(N) = \ln(N) - \ln(N - \Delta N)
  \label{eqn:unten}
\end{equation}

für den unteren Fehler.
Die Messdaten sind in Tabelle \ref{tab:silber} mit entsprechenden Fehlern dargestellt.
\FloatBarrier
 \begin{table}
   \centering
   \caption{Messwerte zur Zerfallskurve des Silber-Isotopengemisches}
   \label{tab:silber}
   \begin{tabular}[t]{c c c c c c}
    \toprule
     $t$ / $\si{\second}$ & $N_0$ & $N$ & $\ln(N)$ & $\Delta^{+} \ln(N)$ & $\Delta^{-} \ln(N)$ \\
      \midrule
      \csvreader[no head,
      late after line=\\,
      late after last line=\\\bottomrule]%
      {data/silbtab.csv}{}%
      {$\num{\csvcoli}$ & $\num{\csvcolii \pm \csvcoliii}$ & $\num{\csvcoliv \pm \csvcolv}$ & $\num{\csvcolvi}$ & $\num{\csvcolvii}$&$\num{\csvcolviii}$ }%
    \end{tabular}
  \end{table}
\FloatBarrier
Es ergibt sich eine ähnliche Kurve wie in Abbildung \ref{fig:2Isotope}.
Es werden nun die Kurven für den langlebigen und den kurzlebigen Zerfall berechnet.
Es wird mit dem langlebigen Zerfall begonnen, da dieser bei den Berechnungen des kurzlebigen Zerfalles berücksichtigt werden muss.
Das $t^*$ wird auf $\SI{110}{\second}$ gelegt.
Ab diesem soll lediglich der langlebige Zerfall eine Rolle spielen.
Es wird somit für die $N$-Werte mit $t \geq t^*$ die Werte für $N_{\Delta t \text{, lang}}$ gefittet.
Dies ist in Abbildung \ref{fig:silberduo} dargestellt.
\FloatBarrier
\begin{figure}
  \centering
  \includegraphics{build/silber2.pdf}
  \caption{Linearer Fit für die Messwerte des langlebigen Zerfalls.}
  \label{fig:silberduo}
\end{figure}
\FloatBarrier
Die lineare Ausgleichsrechnung wird mittels SciPy und uncertainties nach Gleichung \eqref{eqn:Zerfallskonstante} durchgeführt.
Es ergeben sich diese Werte für die Parameter der Funktion:

\begin{align*}
  \lambda_l &= \SI{0.005 \pm 0.001}{\per\second} \\
  N_{0, l} \left(1-\symup{e}^{-\lambda_l \Delta t} \right) &= 3.40 \pm 0.24
\end{align*}

Dies wird zu Anfang analog für den kurzlebigen Zerfall durchgeführt.
Es werden nun lediglich die Zeitwerte für $t \leq t^*$ untersucht.
Würde lediglich eine Ausgleichsrechnung durch die vorliegenden $N$ durchgeführt werden, wäre allerdings noch nicht der Anteil des langsamen Zerfalls berücksichtigt worden.
Es werden somit von den vorliegenden $N$ die Werte $N_{\Delta t, l}$ des langlebigen Zerfalls abgezogen.
Dafür werden die $t \leq t^*$ in die Gleichung \eqref{eqn:Zerfallskonstante} für den langlebigen Zerfall eingesetzt.
\newpage
Die entsprechenden Parameter werden analog zu der vorherigen Ausgleichsrechnung gefittet.
Der entsprechende Plot ist in Abbildung \ref{fig:silbertres} dargestellt.
Die berechneten Werte für die Ausgleichsrechnung können in Tabelle \ref{tab:silber3} gefunden werden.
Es ergibt sich dann für die Parameter der Ausgleichsfunktion:

\begin{align*}
  \lambda_k &= \SI{0.026 \pm 0.002}{\per\second} \\
  N_{0, k} \left(1-\symup{e}^{-\lambda_k \Delta t} \right) &= 5.43 \pm 0.11
\end{align*}
\FloatBarrier
\begin{table}
  \centering
  \caption{Messwerte Bestimmung der Zerfallskonstante des kurzlebigen Zerfalls.}
  \label{tab:silber3}
  \begin{tabular}[t]{c c c c c c c}
   \toprule
    $t$ / $\si{\second}$ & $N$ & $N_{\Delta t, l} $ & $N - N_{\Delta t, l} = K$ & $\ln(K)$ & $\Delta^{+} \ln(K)$ & $\Delta^{-} \ln(K)$ \\
     \midrule
     \csvreader[no head,
     late after line=\\,
     late after last line=\\\bottomrule]%
     {data/Kurztab.csv}{}%
     {$\num{\csvcoli}$ & $\num{\csvcolii \pm \csvcoliii}$ & $\num{\csvcoliv \pm \csvcolv}$ & $\num{\csvcolvi \pm \csvcolvii}$ & $\num{\csvcolviii}$ & $\num{\csvcolix}$ & $\num{\csvcolx}$ }%
   \end{tabular}
 \end{table}

 \begin{figure}
   \centering
   \includegraphics{build/silber3.pdf}
   \caption{Linearer Fit für die Messwerte des kurzlebigen Zerfalls.}
   \label{fig:silbertres}
 \end{figure}

 \begin{figure}
   \centering
   \includegraphics{build/silberf.pdf}
   \caption{Summenkurve der beiden Zerfallsprozesse.}
   \label{fig:silberf}
 \end{figure}
\FloatBarrier
Die entsprechende Summenkurve der beiden ist in Abbildung \ref{fig:silberf} dargestellt.
Die Halbwertszeiten werden nach Gleichung \eqref{eqn:Halbwertszeit} berechnet.
Die Fehler berechnen sich nach der klassischen Fehlerfortpflanzung.
Es ergibt sich daher:

\begin{align}
  T(\lambda) &= \frac{\ln(2)}{\lambda} \Rightarrow \Delta T = \frac{\ln(2)}{\lambda^2} \cdot \Delta \lambda \label{eqn:deltaT}\\
  \Delta T_l &= \pm 27.73 \notag\\
  \Delta T_K &= \pm 2.05 \notag
\end{align}

und für die Halbwertszeit:

\begin{align*}
  T_l = \SI{138.63 \pm 27.73}{\second} \\
  T_k = \SI{26.66 \pm 2.05}{\second}
\end{align*}
\subsection{Halbwertszeit von Indium-116}
Die Messdaten sowie die nach \eqref{eqn:Poisson},\eqref{eqn:nulleffekt}, \eqref{eqn:nfehler}, \eqref{eqn:oben} und \eqref{eqn:unten} berechneten Werte und Fehler befinden sich in
Tabelle \ref{tab:indium}.
\FloatBarrier
\begin{table}
  \centering
  \caption{Messwerte zur Zerfallskurve von Indium}
  \label{tab:indium}
  \begin{tabular}[t]{c c c c c c}
   \toprule
    $t$ / $\si{\second}$ & $N_0$ & $N$ & $\ln(N)$ & $\Delta^{+} \ln(N)$ & $\Delta^{-} \ln(N)$ \\
     \midrule
     \csvreader[no head,
     late after line=\\,
     late after last line=\\\bottomrule]%
     {data/indtab.csv}{}%
     {$\num{\csvcoli}$ & $\num{\csvcolii \pm \csvcoliii}$ & $\num{\csvcoliv \pm \csvcolv}$ & $\num{\csvcolvi}$ & $\num{\csvcolvii}$&$\num{\csvcolviii}$ }%
   \end{tabular}
 \end{table}
 \FloatBarrier
 Nun wird $\ln(N)$ gegen $t$ aufgetragen und mit der Fitfunktion $y=ax+b$ mittels SciPy linear gefittet. Dies ist in Abbildung \ref{fig:indium} dargestellt.
 \FloatBarrier
 \begin{figure}
   \centering
   \includegraphics{build/indium.pdf}
   \caption{Zerfallskurve von Indium.}
   \label{fig:indium}
 \end{figure}
 \FloatBarrier
 Die Parameter ergeben sich so zu
 \begin{align*}
   a &= \SI{-2.16\pm0.09e-4}{\per\second} \\
   b &= \num{7.71\pm0.02} .
 \end{align*}
 Nun wird der natürliche Logarithmus auf \eqref{eqn:Zerfallsgesetz} angewandt, wodurch sich folgende Gleichung ergibt:
 \begin{equation}
   \ln(N(t)) = \ln(N_0) - \lambda t .
   \label{eqn:lnZerfallsgesetz}
 \end{equation}
 Durch Vergleich der Fitparameter mit \eqref{eqn:lnZerfallsgesetz} wird die Zerfallskonstante und damit mittels \eqref{eqn:Halbwertszeit} und \eqref{eqn:deltaT} die Halbwertszeit zu
  \begin{align*}
   \lambda &= \SI{2.16\pm0.09e-4}{\per\second} \\
   T &= \SI{3.21\pm0.14e+3}{\second}
 \end{align*}
 bestimmt.
