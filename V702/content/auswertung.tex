\section{Auswertung}
\label{sec:Auswertung}

\begin{table}
  \centering
  \caption{Messwerte zur Zerfallskurve von Indium}
  \label{tab:indium}
  \begin{tabular}[t]{c c c c c c}
   \toprule
    $t$ / $\si{\second}$ & $N_0$ & $N$ & $\ln(N)$ & $\Delta^{+} \ln(N)$ & $\Delta^{-} \ln(N)$ \\
     \midrule
     \csvreader[no head,
     late after line=\\,
     late after last line=\\\bottomrule]%
     {data/indtab.csv}{}%
     {$\num{\csvcoli}$ & $\num{\csvcolii \pm \csvcoliii}$ & $\num{\csvcoliv \pm \csvcolv}$ & $\num{\csvcolvi}$ & $\num{\csvcolvii}$&$\num{\csvcolviii}$ }%
   \end{tabular}
 \end{table}

\subsection{Halbwertszeit der Silberisotope}
\label{sec:Silberhalbwertszeit}

Aus den Messdaten wird jetzt die Halbwertszeit der zwei Silberisotope berechnet.
Die Probe besteht aus den zwei Isotopen $\ce{^{108}Ag}$ und $\ce{^{110}Ag}$.
$\ce{^{110}Ag}$ zerfällt wesentlich schneller als $\ce{^{108}Ag}$.
Dies stellt ein Problem dar, da beide Zerfallsprozesse simultan ablaufen, wie in der Theorie \ref{sec:Theorie} bereits erwähnt.
Aus den gewonnenen Messdaten müssen somit die beiden Zerfallsprozesse separiert werden.

Die Messwerte befinden sich in Tabelle \ref{tab:silber}.
$N_0$ stellt die ursprünglich gemessene Aktivität im Zeitintervall $\Delta t = \SI{10}{\second}$ dar.
Die Fehler berechnen sich nach dem fehler einer Poisson-Verteilung:

\begin{equation}
  \Delta N = \sqrt{N}
  \label{eqn:Poisson}
\end{equation}

Für die echte Aktivität wird der Nulleffekt mitsamt Fehler subtrahiert.
Der Nulleffekt wird mit $N_\text{Nulleffekt} = 223$ Counts innerhalb von $\SI{900}{\second}$ gemessen.
Es wird also bei einem Zeitintervall von $\SI{10}{\second}$ konstant eine Aktivität von $2.5\pm1.6$ von $N_0$ subtrahiert.
Dies ergibt $N$.
Der Fehler ergibt sich nach der selben Gleichung \eqref{eqn:Poisson}.
Es handelt sich um eine halblogarithmische Ausgleichsrechnung, somit werden die Werte für $N$ noch logarithmiert.
Die entsprechenden logarithmischen Fehler von $N$ ergeben sich nach

\begin{equation}
  \ln{N + \Delta N} - \ln{N}
  \label{eqn:oben}
\end{equation}

für den oberen, und nach

\begin{equation}
  \ln{N} - \ln{N - \Delta N}
\end{equation}

für den unteren Fehler.
Die Messdaten sind in Abbildung \ref{fig:silberuno} mit entsprechenden Fehlern dargestellt.

 \begin{table}
   \centering
   \caption{Messwerte zur Zerfallskurve des Silber-Isotopengemisches}
   \label{tab:silber}
   \begin{tabular}[t]{c c c c c c}
    \toprule
     $t$ / $\si{\second}$ & $N_0$ & $N$ & $\ln(N)$ & $\Delta^{+} \ln(N)$ & $\Delta^{-} \ln(N)$ \\
      \midrule
      \csvreader[no head,
      late after line=\\,
      late after last line=\\\bottomrule]%
      {data/silbtab.csv}{}%
      {$\num{\csvcoli}$ & $\num{\csvcolii \pm \csvcoliii}$ & $\num{\csvcoliv \pm \csvcolv}$ & $\num{\csvcolvi}$ & $\num{\csvcolvii}$&$\num{\csvcolviii}$ }%
    \end{tabular}
  \end{table}

\begin{figure}
  \centering
  \includegraphics{build/silber.pdf}
  \caption{Messwerte für das Silber-Isotopengemisch}
  \label{fig:silberuno}
\end{figure}

Es ergibt sich eine ähnliche Kurve wie in Abbildung \ref{fig:2Isotope}.
Es werden nun die Kurven für den langlebigen und den kurzlebigen Zerfall berechnet.
Es wird mit dem langlebigen Zerfall begonnen, da dieser bei den Berechnungen des kurzlebigen Zerfalles subtrahiert werden muss.
Das $t^{\*}$ wird auf $\SI{200}{\second}$ gelegt.
Ab diesem soll lediglich der langlebige Zerfall eine Rolle spielen.
