\section{Diskussion}
\label{sec:Diskussion}
Die gemessenen Halbwertszeiten, sowie die entsprechenden aus \cite{In} und \cite{Ag} entnommenen Literaturwerte und
die relativen Abweichung sind in der folgenden Auflistung aufgeführt:
\begin{align*}
  T_{\ce{^{108}Ag},\text{gemessen}} &= \SI{138.63 \pm 27.73}{\second} &   T_{\ce{^{108}Ag},\text{Literatur}} &= \SI{143.40}{\second} \\
  & \Rightarrow \text{relative Abweichung} = \SI{3.33}{\percent} \\
  T_{\ce{^{110}Ag},\text{gemessen}} &= \SI{26.66 \pm 2.05}{\second}  &   T_{\ce{^{110}Ag},\text{Literatur}} &= \SI{24.6}{\second} \\
  & \Rightarrow \text{relative Abweichung} = \SI{8.37}{\percent} \\
  T_{\ce{^{116}In},\text{gemessen}} &= \SI{53.5 \pm 2.3}{\minute}  &   T_{\ce{^{116}In},\text{Literatur}} &= \SI{54.2}{\minute} \\
  & \Rightarrow \text{relative Abweichung} = \SI{1.29}{\percent} \\ .
\end{align*}
Es ist zu beobachten, dass die Abweichungen für Indium gering sind, was auf eine genaue Messung mit wenigen systematischen Fehlern schließen lässt.
Für Silber sind diese hingegen etwas größer.
Allerdings liegt bei allen betrachteten Isotopen der Literaturwert der Halbwertszeit innerhalb des ermittelten Fehlerintervalls.
Die geringeren Fehler gehen wahrscheinlich darauf zurück, dass beim Zerfall von Indium nur ein Isotop betrachtet wird und somit keine Näherungen genutzt
werden mussten um die Halbwertszeit zu ermitteln.
Die Theorie des radioaktiven Zerfalls wurde somit in diesem Versuch mit hinreichender Genauigkeit bestätigt.
