\section{Diskussion}
\label{sec:Diskussion}

Die Braggbedingung wurde hinreichend genau nachgewiesen.
Mit einer Abweichung von lediglich $\SI{0.7}{\percent}$ ist die 2:1 Kopplung des Kristalls ohne Bedenken verwendbar.

Ebenfalls stimmen die Energien der gemessenen Grenzwellenlängen und die theoretisch berechneten Energien sehr gut überein.
Hier ergibt sich eine Abweichung von $\SI{0.89}{\percent}$.
Die Energieauflösung wurde auf $\Delta\text{E} = \SI{0.4435\pm0.0695}{\kilo\electronvolt}$ bestimmt.
Diese Angabe ist einigermaßen genau.
Die Werte für die exakten Halbwertsbreiten wurden zwar nur linear interpoliert, allerdings dürften die genutzten Werte nur um wenige zehntel Grad von den tatsächlichen Halbwertsbreiten abweichen.
Hinzukommt, dass es sich hierbei um statistische Verteilungen handelt, die Messwerte werden sich somit von Messung zu Messung unterscheiden.
Genaue Aussagen können nur gemacht werden, wenn man die Ergebnisse über eine hinreichend große Anzahl an Wiederholungen mitteln würde.
Für eine Näherung des Energieauflösungsvermögen ist diese Methode allerdings ausreichend.
Die Energien der K$_\alpha$- und K$_\beta$-Linien konnten mit sehr geringen Fehlern berechnet werden.
Die daraus resultierenden Abschirmungszahlen wiesen ebenfalls Abweichungen von lediglich $\SI{3.53}{\percent}$ für $\sigma_1$ und von $\SI{3.77}{\percent}$ für $\sigma_2$.

Die Energien der Elemente mit Kernladungszahl von $30 \leq Z \leq 50$ weisen nur kleine Abweichungen zu den Literaturwerten auf.
Die Abschirmkonstanten konnten allerdings nicht genau angegeben werden.
Vor allem Zirkonium weicht vergleichsweise stark von den Literaturwerten ab.
Dies schlägt sich auch in der Genauigkeit der Bestimmung der Rydbergkonstante nieder.
Auch hier gibt es eine relativ große Abweichung.
Die Fehler hier können mit der statistischen Natur des Versuches erklärt werden.
Diese verhindert, dass die Daten über mehrere Versuche hinweg konstant bleiben, es müssten somit die Messungen mehrfach durchgeführt und dann über die Daten gemittelt werden.
Hinzukommt die geringe Auflösung der Messdaten an den wichtigen Stellen an den K-Kanten.
Desweiteren wurden bei der Bestimmung der Rydbergkonstante die jeweiligen Abschirmzahlen vernachlässigt. Eine genauere Besimmung dieser Konstante wäre bei Elementen mit kleineren $Z$,
und somit auch kleineren Abschirmzahlen möglich.
Die Messung der L-Kanten von Bismut sind wiederum recht genau.
Jedoch potenzieren sich die Fehler der L-Kante bei der Bestimmung der Abschirmkonstante.
