\section{Auswertung}
\label{sec:Auswertung}

\begin{figure}
  \centering
  \includegraphics{180V.pdf}
  \caption{Lineare Regression für 180 Volt Beschleunigungsspannung.}
  \label{fig:180V}
\end{figure}

\subsection{Spezifische Ladung der Elektronen}
\label{sec:SpezifischeLadung}

Die Messwerte für die konstante Beschleunigungsspannung von $\SI{250}{\volt}$ sind in Tabelle \ref{tab:Ladung1} aufgetragen.

\begin{table}
  \centering
  \caption{Messwerte für die konstante Beschleunigungsspannung von $\SI{250}{\volt}$.}
  \label{tab:Ladung1}
  \begin{tabular}[t]{c c}
   \toprule
     $\text{Liniennummer}$ & $ I \, / \, \si{\milli\ampere}$ \\
     \midrule
     \csvreader[no head,
     late after line=\\,
     late after last line=\\\bottomrule,
     filter test={\ifnumless{\thecsvinputline}{32}}]%
     {data/250Vmag.csv}{}%
     {\csvcoli & \csvcolii}%
   \end{tabular}
 \end{table}

Nach Gleichung \eqref{eqn:helmholtz} wird das B- Feld des Helmholtzspulenpaares für $N = 20$ Windungen und einem Radius von $R = \SI{0.282}{\metre}$ berechnet.
Dies wird in Gleichung \eqref{eqn:AblenkungBFeld} eingesetzt.
Es wird dann $\text{D} / (\text{L}^2 + \text{D}^2)$ gegen B aufgetragen, um die Größe $\text{e}_0/\text{m}_0$ durch eine lineare Ausgleichsrechnung zu bestimmen.
Der Plot ist in Abbildung \ref{fig:250Vmag} zu finden.
Die Berechnung der linearen Ausgleichsgeraden erfolgt mit SciPy, der Fehler wird mit uncertainties berechnet.
Es ergeben sich für die Steigung a und dem y- Achsenabschnitt der Geraden folgende Werte:

\begin{align*}
  a &= 0.87\pm 0.02 \\
  b &= 0.39\pm 0.23
\end{align*}

Vergleicht man Gleichung \eqref{eqn:AblenkungBFeld} mit einer normalen Geradengleichung, so fällt auf, dass um $\text{e}_0/\text{m}_0$ zu bestimmen eine Umrechnung von a notwendig ist.

\begin{equation}
  a^2 \cdot 8 \text{U}_{\text{B}} = \frac{\text{e}_0}{\text{m}_0}
  \label{eqn:SpezifischeLadung}
\end{equation}

Es ergibt sich somit für $\text{e}_0/\text{m}_0$:

\begin{align*}
  \frac{\text{e}_0}{\text{m}_0} = \SI{1.52 \pm 0.08e3}{\coulomb\per\kilo\per\gram}
\end{align*}

\begin{figure}
  \centering
  \includegraphics{250Vmag.pdf}
  \caption{Lineare Regression für 250 Volt Beschleunigungsspannung.}
  \label{fig:250Vmag}
\end{figure}

Selbiges wird für die Beschleunigungsspannung von $\SI{350}{\volt}$ durchgeführt.
Die Messwerte für die konstante Beschleunigungsspannung von $\SI{350}{\volt}$ sind in Tabelle \ref{tab:Ladung2} aufgetragen.

\begin{table}
  \centering
  \caption{Messwerte für die konstante Beschleunigungsspannung von $\SI{350}{\volt}$.}
  \label{tab:Ladung1}
  \begin{tabular}[t]{c c}
   \toprule
     $\text{Liniennummer}$ & $ I \, / \, \si{\milli\ampere}$ \\
     \midrule
     \csvreader[no head,
     late after line=\\,
     late after last line=\\\bottomrule,
     filter test={\ifnumless{\thecsvinputline}{32}}]%
     {data/350Vmag.csv}{}%
     {\csvcoli & \csvcolii}%
   \end{tabular}
 \end{table}

 Nach Gleichung \eqref{eqn:helmholtz} wird das B- Feld des Helmholtzspulenpaares für $N = 20$ Windungen und einem Radius von $R = \SI{0.282}{\metre}$ berechnet.
 Dies wird in Gleichung \eqref{eqn:AblenkungBFeld} eingesetzt.
 Es wird dann $\text{D} / (\text{L}^2 + \text{D}^2)$ gegen B aufgetragen, um die Größe $\text{e}_0/\text{m}_0$ durch eine lineare Ausgleichsrechnung zu bestimmen.
 Der Plot ist in Abbildung \ref{fig:350Vmag} zu finden.
 Es ergeben sich für die Steigung a und dem y- Achsenabschnitt der Geraden folgende Werte:

 \begin{align*}
   a &= 0.73\pm 0.01 \\
   b &= 0.17\pm 0.14
 \end{align*}

Die Berechnung von $\text{e}_0/\text{m}_0$ erfolgt analog mittels Gleichung \eqref{eqn:SpezifischeLadung}.
Es ergibt sich dann:

\begin{align*}
  \frac{\text{e}_0}{\text{m}_0} = \SI{1.48 \pm 0.05e3}{\coulomb\per\kilo\per\gram}
\end{align*}

\begin{figure}
  \centering
  \includegraphics{350Vmag.pdf}
  \caption{Lineare Regression für 350 Volt Beschleunigungsspannung.}
  \label{fig:350Vmag}
\end{figure}

\subsection{Bestimmung der Erdmagnetfeldes}

Der Spulenstrom, welcher das Erdmagnetfeld kompensiert, wurde auf $\SI{40}{\milli\ampere}$ bestimmt.
Die horizontale Komponente des Erdmagnetfeldes entspricht also der entgegengesetzten horizontalen Komponente des Magnetfeldes des Helmholtzspulenpaares.
Es ergibt sich nach Gleichung \eqref{eqn:helmholtz} für $\text{B}_\text{Erd, hor}$:

\begin{align*}
  \text{B}_\text{Erd, hor} = \SI{-2.55e-6}{\tesla}
\end{align*}

Die restlichen Parameter bleiben unverändert zum vorherigen Versuchsteil \ref{sec:SpezifischeLadung}
Die Gesamtfeldstärke von $\text{B}_\text{Erd}$ erhält man nach folgender Gleichung:

\begin{equation}
  \text{B}_\text{Erd} = \frac{\text{B}_\text{Erd, hor}}{\cos \mleft(\phi \mright)}
  \label{eqn:Erdmagnetfeld}
\end{equation}

Hierbei stellt der Winkel $\phi$ den Inklinationswinkel dar.
Er wird auf $\phi = \SI{82}{\degree}$ festgestellt.

Nach Gleichung \eqref{eqn:Erdmagnetfeld} ergibt sich dann:

\begin{align*}
  \text{B}_\text{Erd} = \SI{-18.3e-6}{\tesla}
\end{align*}
