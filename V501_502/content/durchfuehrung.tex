\section{Durchführung}
\label{sec:Durchführung}
\subsection{Messung der Ablenkung im elektrischen Feld}
Zur Messung der Ablenkung des Elektronenstrahls im elektrischen Feld wird die Kathodenstrahlröhre zunächst an eine Spannungsversorgung für die Beschleunigungsspannung,
und eine Spannungsversorgeung für die Ablenkungsspannung angeschlossen. Dabei ist darauf zu Achten, dass die Röhre geerdet ist.
Außerdem wird noch ein Voltmeter mit der Spannungsversorgung der Ablenkungsspannung parallel
geschaltet. Dann wird eine Beschleunigungsspannung eingestellt, und die Elektronenlinse und der Wehnelt-Zylinder so eingestellt, dass ein möglichst kleiner
Punkt auf dem Leuchtschirm zu sehen ist. Dieser wird dann durch die y-Ablenkung auf die unterste der neun äquidistanten Messlinien des Schirms gelenkt.
Nun wird die Ablenkspannung auf dem Voltmeter abgelesen und notiert.
Dann wird die y-Ablenkung so verändert, dass der Punkt auf dem Leuchtschirm auf der nächsten Linie steht. Die
zugehörige Ablenkspannung wird abermals notiert. Dies wird nun wiederholt, bis der Punkt auf allen Linien war, also 9 Ablenkspannungen aufgenommen wurden.
Dann wird eine andere Beschleunigungsspannung angelegt, und der komplette Vorgang wiederholt. Dies wird für insgesamt 5 Beschleunigungsspannungen durchgeführt.
Nun wird anstelle der y-Ablenkung ein Sinusspannungsgenerator, und anstelle der x-Ablenkung ein Sägezahnspannungsgenerator mitsamt Frequenzzähler angeschlossen.
Nun wird die Frequenz der Sägezahnspannung so eingestellt, dass ein stehendes Bild einer oder mehrere überlagerter Sinuskurven zu sehen ist.
Die zugehörige Frequenz wird notiert. Dies wird für drei weitere Frequenzen wiederholt. Zuletzt wird noch die Gesamtablenkung der Sinuspannung in y-Richtung mit
der Skala des Leuchtschirms vermessen.

\subsection{Messung der Ablenkung im magnetischen Feld}
Bei dieser Messung wird die Kathodenstrahlröhre in der Mitte eines Helmholtzspulenpaares platziert, und wieder die Beschleunigungsspannung und Ablenkungsspannung
angelegt. Außerdem werden die Helmholtzspulen an eine Stromversorgung angeschlossen. Dann wird mit einem Deklinatorium-Inklinatorium die Richtung des Erdmagnetfeldes
bestimmt, und die Röhre samt Spulen in Nord-Süd Richtung gedreht, damit die Wirkung des Erdmagnetfeldes minimiert wird.
Nun wird einmal für 250V und einmal für 350V Beschleunigungsspannung eine Messreihe durchgeführt. Dabei wird bei ausgeschaltetem Spulenstrom der Punkt auf dem
Leuchtschirm auf die oberste Linie gelenkt. Nun wird der Spulenstrom soweit hochgeregelt, dass der Punkt auf der nächsten Linie zum stehen kommt. Dies wird für alle
neun Linien durchgeführt. Zur Messung der Intensität des Erdmagnetfeldes wird zunächst die Anordnung bei ausgeschaltetem Spulenstrom in Nord-Süd Richtung ausgerichtet,
dann wird bei 200V Beschleunigungsspannung mithilfe der Ablenkspannung der Punkt auf die Mitte des Schirmes fokussiert. Nun wird die Anordnung um 90° gedreht, und die
dadurch entstandene Ablenkung des Elektronenstrahls durch Erhöhung des Spulenstromes ausgeglichen. Der zugehörige Spulenstrom wird notiert. Zuletzt wird noch der
Inklinationswinkel des Erdmagnetfeldes mithilfe des Deklinatorium-Inklinatorium bestimmt und notiert. 
