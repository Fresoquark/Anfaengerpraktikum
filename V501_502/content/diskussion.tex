\section{Diskussion}
\label{sec:Diskussion}

Die experimentell bestimmte Empfindlichkeit der Kathodenstrahlröhre weicht relativ stark von der berechneten ab.
Der experimentell bestimmte Wert beträgt $\SI{338.03\pm13.24}{\milli\metre}$, wohingegen der berechnete Wert bei $\SI{193.8}{\milli\metre}$ liegt.
Dies ergibt eine Abweichung von $\SI{42.66}{\percent}$.
Die für die einzelnen Beschleunigungsspannungen bestimmten Werte für die Steigung a weisen jeweils nur geringfügige Fehler auf.
Allerdings hat die Steigung der aus den Empfindlichkeiten durchgeführten linearen Regression einen größeren Fehler.
Vermutlich ist dies auf Ungenauigkeiten beim Ablesen auf den Äquidistanzlinien zurückzuführen, welche dann durch Fehlerfortpflanzung verstärkt wurden.
Desweiteren wurden alle Berechnungen nichtrelativistisch durchgeführt.
Dies wird wahrscheinlich sogar noch größere Auswirkungen auf die Abweichung gehabt haben.

Die bestimmte Sinusfrequenz scheint hingegen sehr genau bestimmt zu sein.
Es gibt nur eine geringfügige Abweichung.

Für die spezifische Ladung der Elektronen wurde $\SI{1.52 \pm 0.08e11}{\coulomb\per\kilo\per\gram}$ und \\ $\SI{1.48 \pm 0.05e11}{\coulomb\per\kilo\per\gram}$ experimentell bestimmt.
Vergleicht man dies mit dem Literaturwert von $\SI{1.76e11}{\coulomb\per\kilo\per\gram}$\cite{Ladung}, so erhält man einen Fehler von $\SI{13.64}{\percent}$, bzw. von $\SI{15.91}{\percent}$.
Die spezifischen Ladungen konnten somit recht genau bestimmt werden.

Das Erdmagnetfeld wurde auf eine Stärke von $\SI{18.3e-6}{\tesla}$ bestimmt.
Vergleicht man dies mit den für uns üblichen Werten von $\SI{45e-6}{\tesla}$ bis $\SI{50e-6}{\tesla}$\cite{Erdmagnetfeld}, so ist eine sehr starke Abweichung feststellbar.
