\section{Durchführung}
\label{sec:Durchführung}

Die $\SI{2}{\mega\hertz}$-Ultraschallsonde wird mit dem Ultraschall Doppler-Generator verbunden.
Dieser ist zur Datenaufnahme und Auswertung mit einem Rechner verbunden.
Die vom Echoskop aufgenommenen Messwerte werden durch daS Programm FlowView ausgewertet und angezeigt.

Untersucht wird eine Anordnung von Röhren mit verschiedenen Durchmessern, die mit einer steuerbaren Pumpe verbunden sind.
Das Flüssigkeitsgemisch besteht dabei aus Wasser, Glycerin und Glaskugeln.
Die Viskosität ist so beschaffen, dass sich für die im Messbereich enthaltenen Pumpfrequenzen laminare Strömungen ausbilden.
Die drei Messstellen haben die Innendurchmesser $\SI{7}{\milli\metre}$, $\SI{10}{\milli\metre}$ $\SI{16}{\milli\metre}$.
Um konsistente Messungen mit verschiedenen Winkeln vornehmen zu können werden Dopplerprismen (s.h. Abbildung \ref{fig:prisma}) mit einem Koppelmittel präpariert und auf die Messstellen gesetzt.

Um die Strömungsgeschwindigkeit bestimmen zu können, wird für fünf Pumpfrequenzen/ Spannungen an jeder Messstation für alle drei Dopplerwinkel die Frequenzverschiebung $\Delta \mu$ gemessen.
Dafür muss in dem Programm FlowView das Sample Volume auf Large gestellt werden.
Anschließend soll das Strömungsprofil der Flüssigkeit in dem $\SI{10}{\milli\metre}$ Schlauch bestimmt werden.
Die Messung erfolgt an dem Prismenwinkel von $\SI{15}{\degree}$.
Dafür wird das Sample Volume auf Small gestellt.
Am Ultraschall Doppler-Generator kann mit dem Regler Depth die Messtiefe eingestellt werden.
Um den Schlauch einmal vollständig erfassen zu können, wird die Messung von 12 bis $\SI{19}{\micro\second}$ durchgeführt.
Dies wird für Pumpleistungen von $\SI{70}{\percent}$ und $\SI{30}{\percent}$ durchgeführt.
