\section{Auswertung}
\label{sec:Auswertung}
\subsection{Bestimmung der Strömungsgeschwindigkeiten für verschiedene Rohre, Flussgeschwindigkeiten und Winkel}
Zunächst werden nach Gleichung \eqref{eqn:prisma} die Dopplerwinkel $\alpha_i$ für die Prismenwinkel 15°, 30° und 60° berechnet:
\begin{align*}
  \alpha_{15} &= \SI{80.06}{\degree} \\
  \alpha_{30} &= \SI{70.53}{\degree} \\
  \alpha_{60} &= \SI{54.74}{\degree} .
\end{align*}
Dabei wurden die Schallgeschwindigkeiten $c_P=\SI{2700}{\metre\per\second}$ und $c_L=\SI{1800}{\metre\per\second}$ aus \cite{sample} entnommen.
Zur Bestimmung der Strömungsgeschwindigkeiten aus den in \ref{tab:duenn}, \ref{tab:mittel} und \ref{tab:dick} aufgetragenen Messdaten wird
Gleichung \eqref{eqn:deltanu} nach der Geschwindigkeit $v$ umgestellt:
\begin{equation}
  v=\frac{\Delta\nu\,c}{2 \nu_0 \cos(\alpha)} .
  \label{eqn:geschwindigkeit}
\end{equation}
Nun werden mit dieser Gleichung, $\nu_0=\SI{2}{\mega\hertz}$, $c=c_L$ und den zuvor bestimmten $\alpha_i$ die Strömungsgeschwindigkeiten für die verschiedenen Rohre
und Winkel berechnet. Dann werden die Geschwindigkeiten der verschiedenen Winkel mit
\begin{equation}
  \label{eqn:mittelwert}
  \overline{x} = \frac{1}{N} \sum_{i=1}^N x_i
\end{equation}
gemittelt und der entsprechende Fehler mit
\begin{equation}
  \label{eqn:mittelwertfehler}
  \Delta \overline{x} = \frac{1}{\sqrt{N}} \sqrt{\frac{1}{N-1} \sum_{i=1}^N (x_i - \overline{x})^2}
\end{equation}
berechnet. Die Ergebnisse sind in den Tabellen  \ref{tab:duenn}, \ref{tab:mittel} und \ref{tab:dick} dargestellt.

 \begin{table}
   \centering
   \caption{Messdaten und -ergebnisse für das $\SI{7}{\milli\metre}$-Rohr.}
   \label{tab:duenn}
   %\sisetup{table-number-alignment=center}
   \noindent\makebox[\textwidth]{ %Nur zur Zentrierung der sonst zu großen Tabelle
   %\begin{tabular}[t]{c@{} S S S S S S S S}
   \begin{tabular}[t]{c c c c c c c c c}
    \toprule
     &{$v_\text{Fluss}$ / $\si{\percent}$}&{${\Delta\nu}_{15}$ / $\si{\hertz}$}&{$v_{15}$ / $\si{\metre\per\second}$}&{${\Delta\nu}_{30}$ / $\si{\hertz}$}&{$v_{30}$ / $\si{\metre\per\second}$}&{${\Delta\nu}_{60}$ / $\si{\hertz}$}&{$v_{60}$ / $\si{\metre\per\second}$}&{$\overline{v}$ / $\si{\metre\per\second}$}\\
      \midrule
      \csvreader[no head,
      late after line=\\,
      late after last line=\\\bottomrule]%
      {data/duenntab.csv}{}%
      {&$\num{\csvcoli}$ & $\num{\csvcolii}$ & $\num{\csvcoliii}$ & $\num{\csvcoliv}$& $\num{\csvcolv}$& $\num{\csvcolvi}$& $\num{\csvcolvii}$& $\num{\csvcolviii\pm\csvcolix}$}%
    \end{tabular}
    }
  \end{table}

\begin{table}
  \centering
  \caption{Messdaten und -ergebnisse für das $\SI{10}{\milli\metre}$-Rohr.}
  \label{tab:mittel}
  %\sisetup{table-number-alignment=center}
  \noindent\makebox[\textwidth]{ %Nur zur Zentrierung der sonst zu großen Tabelle
  %\begin{tabular}[t]{c@{} S S S S S S S S}
  \begin{tabular}[t]{c c c c c c c c c}
   \toprule
    &{$v_\text{Fluss}$ / $\si{\percent}$}&{${\Delta\nu}_{15}$ / $\si{\hertz}$}&{$v_{15}$ / $\si{\metre\per\second}$}&{${\Delta\nu}_{30}$ / $\si{\hertz}$}&{$v_{30}$ / $\si{\metre\per\second}$}&{${\Delta\nu}_{60}$ / $\si{\hertz}$}&{$v_{60}$ / $\si{\metre\per\second}$}&{$\overline{v}$ / $\si{\metre\per\second}$}\\
     \midrule
     \csvreader[no head,
     late after line=\\,
     late after last line=\\\bottomrule]%
     {data/mitteltab.csv}{}%
     {&$\num{\csvcoli}$ & $\num{\csvcolii}$ & $\num{\csvcoliii}$ & $\num{\csvcoliv}$& $\num{\csvcolv}$& $\num{\csvcolvi}$& $\num{\csvcolvii}$& $\num{\csvcolviii\pm\csvcolix}$}%
   \end{tabular}
   }
 \end{table}

  \begin{table}
    \centering
    \caption{Messdaten und -ergebnisse für das $\SI{16}{\milli\metre}$-Rohr.}
    \label{tab:dick}
    %\sisetup{table-number-alignment=center}
    \noindent\makebox[\textwidth]{ %Nur zur Zentrierung der sonst zu großen Tabelle
    %\begin{tabular}[t]{c@{} S S S S S S S S}
    \begin{tabular}[t]{c c c c c c c c c}
     \toprule
      &{$v_\text{Fluss}$ / $\si{\percent}$}&{${\Delta\nu}_{15}$ / $\si{\hertz}$}&{$v_{15}$ / $\si{\metre\per\second}$}&{${\Delta\nu}_{30}$ / $\si{\hertz}$}&{$v_{30}$ / $\si{\metre\per\second}$}&{${\Delta\nu}_{60}$ / $\si{\hertz}$}&{$v_{60}$ / $\si{\metre\per\second}$}&{$\overline{v}$ / $\si{\metre\per\second}$}\\
       \midrule
       \csvreader[no head,
       late after line=\\,
       late after last line=\\\bottomrule]%
       {data/dicktab.csv}{}%
       {&$\num{\csvcoli}$ & $\num{\csvcolii}$ & $\num{\csvcoliii}$ & $\num{\csvcoliv}$& $\num{\csvcolv}$& $\num{\csvcolvi}$& $\num{\csvcolvii}$& $\num{\csvcolviii\pm\csvcolix}$}%
     \end{tabular}
     }
   \end{table}

Nun wird für einen Primsenwinkel von $\SI{30}{\degree}$ bei jedem Rohr $\Delta\nu/\cos(\alpha)$ gegen $v$ aufgetragen.
Dies ist in den Abbildungen \ref{fig:duenn30}, \ref{fig:mittel30} und \ref{fig:dick30} dargestellt.

\begin{figure}
  \centering
  \includegraphics{duenn30.pdf}
  \caption{Grafische Darstellung der Daten für das $\SI{7}{\milli\metre}$-Rohr bei einem Winkel von 30°.}
  \label{fig:duenn30}
\end{figure}

\begin{figure}
  \centering
  \includegraphics{mittel30.pdf}
  \caption{Grafische Darstellung der Daten für das $\SI{10}{\milli\metre}$-Rohr bei einem Winkel von 30°.}
  \label{fig:mittel30}
\end{figure}

\begin{figure}
  \centering
  \includegraphics{dick30.pdf}
  \caption{Grafische Darstellung der Daten für das $\SI{16}{\milli\metre}$-Rohr bei einem Winkel von 30°.}
  \label{fig:dick30}
\end{figure}

\FloatBarrier
\subsection{Bestimmung des Geschwindigkeitsprofils in einer Röhre}

Die Messwerte zur Bestimmung des Geschwindigkeitsprofils sind in Tabelle \ref{tab:vprofil} dargestellt.
Die Geschwindigkeiten wurden mit Gleichung \eqref{eqn:geschwindigkeit} berechnet.
Die Tiefe im Rohr $z$ wird mit den aus \cite{sample} entnommenen Umrechnungen zwischen $\si{\micro\second}$ und $\si{\milli\metre}$, sowie der Vorlaufstrecke im Prisma von $\SI{30.7}{\milli\metre}$ durch folgende Formel bestimmt:
\begin{equation}
  z= \frac{3}{2} \cdot \bigl(t-30,7\cdot\frac{2}{5}\bigr) .
\end{equation}
Die entsprechenden Plots des Geschwindigkeitsprofils mit entsprechenden Streuintensitäten ist für eine Pumpleistung von $\SI{70}{\percent}$ in Abbildung \ref{fig:v1} und für eine Pumpleistung von $\SI{45}{\percent}$ in Abbildung \ref{fig:v2} dargstellt.

\begin{table}
  \centering
  \caption{Messdaten und -ergebnisse für das Geschwindigkeitsprofil des $\SI{10}{\milli\metre}$-Rohr.}
  \label{tab:vprofil}
  \noindent\makebox[\textwidth]{ %Nur zur Zentrierung der sonst zu großen Tabelle
  %\begin{tabular}[t]{c@{} S S S S S S S S}
  \begin{tabular}[t]{c c c c c c c c c}
   \toprule
    &{$t$ / $\si{\micro\second}$}&{$z$ / $\si{\milli\metre}$}&{$\Delta f_{70}$ / $\si{\hertz}$}&{$v_{70}$ / $\si{\metre\per\second}$}&{$I_\text{Streu,70}$ / $\si{\percent}$}&{$\Delta f_{45}$ / $\si{\hertz}$}&{$v_{45}$ / $\si{\metre\per\second}$}&{$I_\text{Streu,45}$ / $\si{\percent}$}\\
     \midrule
     \csvreader[no head,
     late after line=\\,
     late after last line=\\\bottomrule]%
     {data/vprofiltab.csv}{}%
     {&$\num{\csvcoli}$ & $\tablenum{\csvcolii}$ & $\num{\csvcoliii}$ & $\num{\csvcoliv}$& $\tablenum{\csvcolv}$& $\num{\csvcolvi}$& $\num{\csvcolvii}$& $\tablenum{\csvcolviii}$}%
   \end{tabular}
   }
 \end{table}

 \begin{figure}
   \centering
   \includegraphics{v1.pdf}
   \caption{Grafische Darstellung der Daten des Geschwindigkeitsprofils bei einer Pumpleistung von $\SI{70}{\percent}$.}
   \label{fig:v1}
 \end{figure}

 \begin{figure}
   \centering
   \includegraphics{v2.pdf}
   \caption{Grafische Darstellung der Daten des Geschwindigkeitsprofils bei einer Pumpleistung von $\SI{45}{\percent}$.}
   \label{fig:v2}
 \end{figure}
