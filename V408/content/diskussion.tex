\section{Diskussion}
\label{sec:Diskussion}
Die ermittelten Ergebnisse lauten:
\begin{align*}
  f_\text{1,analytisch} &=  \SI{4.45\pm0.06}{\centi\metre} &   f_\text{1,Hersteller} &= \SI{5}{\centi\metre} \\
  & \Rightarrow \text{Abweichung} = \SI{10}{\percent} \\
  f_\text{1,grafisch} &=  \SI{4.58}{\centi\metre} &   f_\text{1,Hersteller} &= \SI{5}{\centi\metre} \\
  & \Rightarrow \text{Abweichung} = \SI{8.4}{\percent} \\
  f_\text{Wasser,analytisch} &=  \SI{6.134\pm0.01}{\centi\metre} &  f_\text{Wasser,grafisch} &=  \SI{6.17}{\centi\metre} \\
  & \Rightarrow \text{Abweichung} = \SI{5.9}{\percent} \\
  f_\text{2,Bessel} &=  \SI{9.95\pm0.07}{\centi\metre} &  f_\text{2,Hersteller} &=  \SI{10}{\centi\metre} \\
  & \Rightarrow \text{Abweichung} = \SI{0.5}{\percent} \\
  f_\text{2,Bessel, rot} &=  \SI{9.94\pm0.11}{\centi\metre} &  f_\text{2,Bessel,blau} &= \SI{10.4\pm0.5}{\centi\metre} \\
  f_\text{2,Abbe} &=  \SI{6.8\pm0.5}{\centi\metre} &  f_\text{2,Hersteller} &=  \SI{10}{\centi\metre} \\
  & \Rightarrow \text{Abweichung} = \SI{32}{\percent} \\
\end{align*}
Bei der ersten, direkten Messung sind sowohl das analytische als auch das grafische Ergebnis hinreichend genau. Jedoch ist bei der grafischen Methode ersichtlich,
dass wahrscheinlich ein Fehler bei der Bestimmung der Bildweite passiert ist. Dies lässt sich aus der Tatsache folgern, dass in Abbildung \ref{fig:Messgenauigkeit} der Schnittpunkt
der Geraden auf der x-Achse, auf welcher g aufgetragen ist, sehr nah an der Herstellerangabe ist, jedoch auf der y-Achse eine größere Abweichung zeigt. Bei einer
genauen Messung sollte die x- und y-Koordinate des Schnittpunktes gleich sein, da beide die Brennweite darstellen.
Der Fehler bei der Bildweite ist warscheinlich der sehr subjektiven Betrachtung eines "scharfen" Bildes geschuldet.
Bei der Wasserlinse gibt es keine Herstellerangabe, da aber der Schnittpunkt der Geraden in Abbildung \ref{fig:MessgenauigkeitII} die selbe x- und y-Koordinate hat, kann auf eine
hohe Messgenauigkeit geschlossen werden und die Brennweite mit hinreichender Genauigkeit zu den oben angegebenen Werten bestimmt werden. Dies bestätigt sich
ebenfalls durch die geringe Abweichung zwischen dem analytisch und grafisch bestimmten Ergebnis.
Die Bessel Methode liefert ein sehr genaues Messergebnis für die Brennweite der Linse. Bei der Messung der chromatischen Abberation jedoch ist die Brennweite für
blaues Licht größer als die für rotes Licht, was en Gesetzen der Optik widerspricht. Da beide Brennweiten jedoch sehr nah aneinander liegen, ist es sehr Wahrscheinlich,
dass die Messgenauigkeit nicht ausrechend war um diese kleinen Unterschiede aufzulösen.
Die Abbe Methode liefert das am stärksten Abweichende Ergebnis für die Brennweite der Linse. Dies geht vermutlich auf Messfehler bei der Bestimmung der Bildgröße
zurück, welche mit einem per Henad gehaltenem Lineal durchgeführt wurde und deshalb einen großen Ablesefehler generiert. Zur genauen Überprüfung dee Messmethode nach Abbe
müsste die Messung erneut mit einer genaueren Messapparatur durchgeführt werden.
