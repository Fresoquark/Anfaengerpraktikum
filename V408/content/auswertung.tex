\section{Auswertung}
\label{sec:Auswertung}

\subsection{Messung der Gegenstands- und Bild-weite bei bekannter Brennweite}
\FloatBarrier
Die Messwerte für die Bestimmung der Brennweite mittels Messen der Gegenstands- und Bild-weite sind in Tabelle \ref{tab:bekannt} aufgetragen.
Hierbei wurden nicht die Größen g und b direkt gemessen, sondern die Position der optischen Elemente und die Bildgröße B.
Die Position des Perl L ist für jeden Versuch mit $\SI{20}{\centi\metre}$ gemessen worden. Die Gegenstandsgröße G des Perl L berträgt laut Hersteller $\SI{3}{\centi\metre}$.

\begin{table}
  \centering
  \caption{Messwerte für die Linse mit bekannter Brennweite}
  \label{tab:bekannt}
  \begin{tabular}[t]{c c c}
   \toprule
     $\text{Pos d. Linse} \, / \, \si{\centi\metre}$ & $\text{Pos d. Schrims} \, / \, \si{\centi\metre}$ & $B \, / \, \si{\centi\metre}$ \\
     \midrule
     \csvreader[no head,
     late after line=\\,
     late after last line=\\\bottomrule,
     filter test={\ifnumless{\thecsvinputline}{32}}]%
     {data/linse1.csv}{}%
     {\csvcoli & \csvcolii  & \csvcoliii}%
   \end{tabular}
 \end{table}

Um die Brennweite der Linse berechnen zu können, wird die Linsengleichung \eqref{eqn:Linsengleichung} verwendet.
Der Mittelwert wird mittels Numpy nach folgender Formel berechnet:

\begin{equation}
  \label{eqn:mittelwert}
  \overline{x} = \frac{1}{N} \sum_{i=1}^N x_i
\end{equation}

Der Fehler des Mittelwertes nach dieser mittels uncertainties:

\begin{equation}
  \label{eqn:mittelwertfehler}
  \Delta \overline{x} = \frac{1}{\sqrt{N}} \sqrt{\frac{1}{N-1} \sum_{i=1}^N (x_i - \overline{x})^2}
\end{equation}

Es ergibt sich folglich für die Brennweite:

\begin{align*}
  f = \SI{4.45\pm0.06}{\centi\metre}
\end{align*}

Dies wird mit der Herstellerangabe von $\SI{50}{\milli\metre}$ verglichen.
Es ergibt sich ein Fehler von $\SI{10}{\percent}$.
Die Messgenauigkeit wird durch den folgenden Plot überprüft.
Hierbei werden die Wertepaare für $g_i, b_i$ aufgetragen.
Bei hoher Genauigkeit sollten sich alle Geraden in einem Punkt schneiden. Die x und y Koordinate dieses Punktes stellt dann die Brennweite dar.
Da die Graphen für $g = \SI{5}{\centi\metre}, \SI{13}{\centi\metre}, \SI{14}{\centi\metre}$ zu stark abweichen werden diese für diese Berechnung vernachlässigt.
Durch Ablesen aus dem Graphen wird der Schnittpunkt der Geraden zu $(4.97,4.19)$ bestimmt. Die Brennweite beträgt dann $f=\SI{4.97}{\centi\metre}$ oder
$f=\SI{4.19}{\centi\metre}$. Die durch die x-Koordinate bestimmte Brennweite weicht um 0.6 \% von den Herstellerangaben ab, die durch die y-Koordinate bestimmte
jedoch um 16.2 \%. Der Mittelwert der beiden liegt bei $f=\SI{4.58}{\centi\metre}$ und weicht um 8.4 \% ab. Desweiteren wird noch das Abbildungsgesetz \eqref{eqn:Abbildungsgesetz} überprüft.
Dafür wird der Abbildungsmaßstab einmal durch die gemessenen b und g bestimmt, und einmal durch die Bild- und Gegenstandsgröße B und G.
Die Ergebnisse befinden sich in Tabelle \ref{tab:V}.

\begin{table}
  \centering
  \caption{Vergleich der Abbildungsmaßstäbe}
  \label{tab:V}
  \begin{tabular}{c c c }
    \toprule
    $V(b,g)$ & $V(B,G)$ & $Relative Abweichung \, / \, \%$ \\
    \midrule
    4.71 & 4.10 & 14.88\\
    3.28 & 3.60 & 9.00\\
    2.06 & 2.27 & 9.25\\
    1.34 & 1.43 & 6.72\\
    1.02 & 1.13 & 10.78\\
    0.83 & 0.93 & 12.05\\
    0.67 & 0.77 & 14.93\\
    0.60 & 0.67 & 11.66\\
    0.51 & 0.60 & 17.65\\
    0.41 & 0.53 & 29.27\\
    \bottomrule
  \end{tabular}
\end{table}

\begin{figure}
  \centering
  \includegraphics{linse1.pdf}
  \caption{Überprüfen der Messgenauigkeit der Linse mit bekannter Brennweite}
  \label{fig:Messgenauigkeit}
\end{figure}
\FloatBarrier
\subsection{Messung der Gegenstands- und Bild-weite bei unbekannter Brennweite}
\FloatBarrier
Für die Messung der Brennweite einer Linse mit unbekannter Brennweite wird eine Wasserlinse verwendet.
Die Messungen erfolgen analog.
Die Messwerte für die Positionen der optischen Elemente sind in Tabelle \ref{tab:unbekannt} dargestellt.
Das Perl L befindet sich erneut bei $\SI{20}{\centi\metre}$.

\begin{table}
  \centering
  \caption{Messwerte für die Linse mit unbekannter Brennweite}
  \label{tab:unbekannt}
  \begin{tabular}[t]{c c}
   \toprule
     $\text{Pos d. Linse} \, / \, \si{\centi\metre}$ & $\text{Pos d. Schrims} \, / \, \si{\centi\metre}$ \\
     \midrule
     \csvreader[no head,
     late after line=\\,
     late after last line=\\\bottomrule,
     filter test={\ifnumless{\thecsvinputline}{32}}]%
     {data/linse2.csv}{}%
     {\csvcoli & \csvcolii}%
   \end{tabular}
 \end{table}

Die Brennweite wird erneut mit der Linsengleichung \eqref{eqn:Linsengleichung} ermittelt.
Der Mittelwert nach Gleichung \eqref{eqn:mittelwert} und dessen Fehler nach Gleichung \eqref{eqn:mittelwertfehler}.
Es ergibt sich folglich für die Brennweite:

\begin{align*}
  f = \SI{6.134\pm0.010}{\centi\metre}
\end{align*}

Da bei der Wasserlinse keine Herstellerangabe überprüft werden kann, muss die Messgenauigkeit einzig mithilfe des Plotes \ref{fig:MessgenauigkeitII} überprüft werden.
Hier wird erneut der Schnittpunkt der einzelnen Geraden ermittelt. Die Koordinaten des Schnittpunktes entsprechen jeweils der Brennweite.
Der Schnittpunkt wurde wieder durch Ablesen aus dem Graphen zu $(6.17,6.17)$ bestimmt. Somit ergibt sich für beide Koordinaten übereinstimmend eine Brennweite
von $f=\SI{6.17}{\centi\metre}$. Diese weicht um 5.9 \% von der zuvor bestimmten Brennweite ab.

\begin{figure}
  \centering
  \includegraphics{linse2.pdf}
  \caption{Überprüfen der Messgenauigkeit der Wasserlinse}
  \label{fig:MessgenauigkeitII}
\end{figure}

\FloatBarrier
\subsection{Methode nach Bessel}
\FloatBarrier
Wie in Abschnitt \ref{sec:bessel} der Durchführung beschrieben wird bei der Messung nach Bessel der Abstand zwischen Schirm und Perl variiert.
Dieser Abstand wird mit e bezeichnet. Dieser Abstand ist die Differenz zwischen der Position des Schirmes $p_0$ und des Gegenstandes.
Ferner werden die zwei Positonen der Linsen festgehalten, für die das Bild bei gegebenem Abstand e scharf ist.
Diese werden mit $p_1$ und $p_2$ bezeichnet.
Die Ergebnisse sind in Tabelle \ref{tab:Bessel1} dargestellt.

\begin{table}
  \centering
  \caption{Messwerte für die Besselsche Methode}
  \label{tab:Bessel1}
  \begin{tabular}[t]{c c c}
   \toprule
     $p_0 \, / \, \si{\centi\metre}$ & $p_1 \, / \, \si{\centi\metre}$ & $p_2 \, / \, \si{\centi\metre}$ \\
     \midrule
     \csvreader[no head,
     late after line=\\,
     late after last line=\\\bottomrule,
     filter test={\ifnumless{\thecsvinputline}{32}}]%
     {data/bessel1.csv}{}%
     {\csvcoli & \csvcolii & \csvcoliii}%
   \end{tabular}
 \end{table}

Die Besselsche Methode erfordert die Nutzung der Gleichung \eqref{eqn:Bessel}.
Wendet man dies für alle Abstände an, erhält man folgende Werte, die in Tabelle \ref{tab:Besselbrennweite2} aufgetragen sind:

\begin{table}
  \centering
  \caption{Brennweiten nach Bessel}
  \label{tab:Besselbrennweite}
  \begin{tabular}{S[table-format=3.0] S [table-format=2.2] S [table-format=1.2]}
    \toprule
    {$e \, / \, \si{\centi\metre}$} & {$f_1 \, / \, \si{\centi\metre}$} & {$f_2 \, / \, \si{\centi\metre}$} \\
    \midrule
    45  & 10.03 & 9.53 \\
    50  & 10.12 & 9.52 \\
    55  & 10.08 & 9.71 \\
    60  & 10.18 & 9.60 \\
    65  & 10.28 & 9.78 \\
    70  & 10.27 & 9.81 \\
    75  & 10.22 & 9.67 \\
    80 & 10.20 & 9.70 \\
    85 & 10.45 & 9.72 \\
    90 & 10.55 & 9.66 \\
    \bottomrule
  \end{tabular}
\end{table}

Mittelt man diese Werte zunächst für $f_1$ und anschließend für $f_2$, so erhält man die Mittelwerte nach Gleichung \eqref{eqn:mittelwert} und deren Fehler nach Gleichung \eqref{eqn:mittelwertfehler}.
Anschließend werden noch einmal alle Brennweiten gemittelt und zu einer Brennweite zusammengefasst.

\begin{align*}
  f_1 &= \SI{10.24\pm0.05}{\centi\metre} \\
  f_2 &= \SI{9.67\pm0.03}{\centi\metre} \\
  f   &= \SI{9.95\pm0.07}{\centi\metre}
\end{align*}

Vergleicht man diesen Wert mit den Herstellerangaben von $f = \SI{100}{\milli\metre}$, so sieht man, dass die gemessene Brennweite sehr nah an der Herstellerangabe dran ist.
Es ergibt sich ein Fehler von lediglich $\SI{0.5}{\percent}$.

\FloatBarrier
\subsection{Chromatische Abberation und Bessel}
\FloatBarrier

Der Aufbau nach Bessel wird nun für einen roten und blauen Filter reproduziert.
An dem Prozess ändert sich im Vergleich zu vorher nichts, es wird lediglich jeweils ein roter und ein blauer Filter hinter dem Perl L montiert.
Die Positionen der Linsen sind jeweils mit $p_{1,rot}$ und $p_{2,rot}$, bzw. mit $p_{1,blau}$ und $p_{2,blau}$ gekennzeichnet.
Die Messwerte sind in Tabelle \ref{tab:Bessel2} zusammengefasst.

\begin{table}
  \centering
  \caption{Messwerte für die Besselsche Methode}
  \label{tab:Bessel2}
  \begin{tabular}[t]{c c c c c}
   \toprule
     $p_0 \, / \, \si{\centi\metre}$ & $p_{1,rot} \, / \, \si{\centi\metre}$ & $p_{2,rot} \, / \, \si{\centi\metre}$  & $p_{1,blau} \, / \, \si{\centi\metre}$ & $p_{2,blau} \, / \, \si{\centi\metre}$\\
     \midrule
     \csvreader[no head,
     late after line=\\,
     late after last line=\\\bottomrule,
     filter test={\ifnumless{\thecsvinputline}{32}}]%
     {data/bessel2.csv}{}%
     {\csvcoli & \csvcolii & \csvcoliii & \csvcoliv  & \csvcolv}%
   \end{tabular}
 \end{table}

Auch hier wird die Gleichung \eqref{eqn:Bessel} verwendet.
Es ergeben sich dann folgende Werte, die in Tabelle \ref{tab:Besselbrennweite2} aufgetragen sind:

\begin{table}
  \centering
  \caption{Brennweiten nach Bessel für farbige Filter}
  \label{tab:Besselbrennweite2}
  \begin{tabular}{S[table-format=3.0] S [table-format=2.2] S [table-format=1.2] S [table-format=2.2] S [table-format=1.2]}
    \toprule
    {$e \, / \, \si{\centi\metre}$} & {$f_{1, rot} \, / \, \si{\centi\metre}$} & {$f_{2, rot} \, / \, \si{\centi\metre}$} & {$f_{1, blau} \, / \, \si{\centi\metre}$} & {$f_{2, blau} \, / \, \si{\centi\metre}$} \\
    \midrule
    50 & 10.08 & 9.74 & 10.12 & 9.72 \\
    55 & 10.24 & 9.55 & 10.26 & 9.66 \\
    60 & 10.27 & 9.60 & 14.82 & 9.66 \\
    65 & 10.31 & 9.56 & 10.34 & 9.72 \\
    70 & 10.36 & 9.68 & 10.36 & 9.78 \\
    \bottomrule
  \end{tabular}
\end{table}

Mittelt man diese Werte zunächst für $f_{1, rot}$, $f_{2, rot}$, $f_{1, blau}$ und $f_{2, blau}$, so erhält man die Mittelwerte nach Gleichung \eqref{eqn:mittelwert} und deren Fehler nach Gleichung \eqref{eqn:mittelwertfehler}.
Anschließend werden noch einmal die farblich entsprechenden Brennweiten gemittelt und zu einer Brennweite zusammengefasst.

\begin{align*}
  f_{1, rot}  &= \SI{10.25\pm0.05}{\centi\metre} \\
  f_{2, rot}  &= \SI{9.63\pm0.04}{\centi\metre}  \\
  f_{rot}     &= \SI{9.94\pm0.11}{\centi\metre}  \\
  f_{1, blau} &= \SI{11.2\pm0.90}{\centi\metre}  \\
  f_{2, blau} &= \SI{9.7\pm0.02}{\centi\metre}   \\
  f_{blau}    &= \SI{10.4\pm0.5}{\centi\metre}   \\
\end{align*}

\FloatBarrier
\subsection{Methode nach Abbe}
\FloatBarrier
Die gemessenen Bildgrößen und Referenzpunktpositionen befinden sich in Tabelle \ref{tab:abbe}. Der Abstand zwischen Gegenstand und Schirm wurde auf $\SI{35}{\centi\metre}$
eingestellt.
\begin{table}
  \centering
  \caption{Messwerte für die Abbesche Methode}
  \label{tab:abbe}
  \begin{tabular}[t]{c c}
   \toprule
     $p_\text{A} \, / \, \si{\centi\metre}$ & $B \, / \, \si{\centi\metre}$ \\
     \midrule
     \csvreader[no head,
     late after line=\\,
     late after last line=\\\bottomrule]
     {data/abbe.csv}{}%
     {\csvcolii & \csvcoli}%
   \end{tabular}
 \end{table}
Aus den Positionen des Gegenstandes und des Referenzpunktes des Linsensystems lassen sich nun $g'$ und $b'$ bestimmen. Der Abbildungsmaßstab V wird mittels Gleichung
\eqref{eqn:Abbildungsgesetz} und der gemessenen Bildgröße B sowie der bekannten Gegenstandsgröße bestimmt.
Die so ermittelten Werte befinden sich in Tabelle \ref{tab:abbe2}.
\FloatBarrier
\begin{table}
  \centering
  \caption{Gegenstandsweite, Bildweite und Abbukdungsmaßstab der Abbe Methode}
  \label{tab:abbe2}
  \begin{tabular}{S[table-format=3.0] S [table-format=2.2] S [table-format=1.2]}
    \toprule
    $g' \, / \, \si{\centi\metre}$ &   $b' \, / \, \si{\centi\metre}$ & V \\
    \midrule
    8  & 27 & 1.63\\
    9  & 26 & 1.77\\
    10 & 25 & 1.43\\
    11 & 24 & 1.23\\
    12 & 23 & 1.07\\
    13 & 22 & 1.27\\
    14 & 21 & 0.80\\
    15 & 20 & 0.70\\
    16 & 19 & 0.63\\
    17 & 18 & 0.57\\
    \bottomrule
  \end{tabular}
\end{table}
\FloatBarrier
Nun werden die Gleichungen \eqref{eqn:Abbe1} und \eqref{eqn:Abbe2} mittels Scipy mit
\begin{equation*}
  y=ax+b
\end{equation*}
linear gefittet, indem $g'$ gegen $1+\sfrac{1}{V}$ und $b'$ gegen $1+V$ aufgetragen wird.
Die so entstandenen Plots sind in den Abbildungen \ref{fig:abbe1} und \ref{fig:abbe2} zu sehen.
\begin{figure}
  \centering
  \includegraphics{abbe1.pdf}
  \caption{Lineare Regression für die Abbe Methode.}
  \label{fig:abbe1}
\end{figure}
\begin{figure}
  \centering
  \includegraphics{abbe2.pdf}
  \caption{Lineare Regression für die Abbe Methode.}
  \label{fig:abbe2}
\end{figure}
Bei \eqref{eqn:Abbe1} und \eqref{eqn:Abbe2} entspricht der Parameter $a$ der Brennweite der Sammellinse. Der Parameter $b$ entspricht bei der ersten Gleichung dem Abstand
$h$ zwischen der ersten Hauptebene und dem Referenzpunkt, bei der zweiten Gleichung entspricht $b$ dem Abstand $h'$ zwischen der zweiten Hauptebene und dem
Referenzpunkt. So wurden die Brennweiten und Äbstande der Hauptebenen zu
\begin{align*}
  f_{g'} &= \SI{6.71\pm0.76}{\centi\metre} \\
  f_{b'} &= \SI{6.81\pm0.71}{\centi\metre} \\
  h &= \SI{-1.21\pm1.59}{\centi\metre} \\
  h' &= \SI{8.14\pm1.52}{\centi\metre}
\end{align*}
bestimmt.
Die aus $f_{g'}$ und $f_{b'}$ mittels \eqref{eqn:mittelwert} und \eqref{eqn:mittelwertfehler} errechnete mittlere Brennweite beträgt $f_\text{Abbe}=\SI{6.8\pm0.5}{\centi\metre}$.
Die Brennweiten der verwendeten Sammel- und Zerstreuungslinse betragen laut Herstellerangabe $\SI{\pm10}{\centi\metre}$. Somit ergibt sich eine
relative Abweichung von $32 \%$.
