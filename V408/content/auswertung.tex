\section{Auswertung}
\label{sec:Auswertung}

\subsection{Messung der Gegenstands- und Bild-weite}

Die Messwerte für die Bestimmung der Brennweite mittels Messen der Gegenstands- und Bild-weite sind in Tabelle \ref{tab:Gegenstands} aufgetragen.
Hierbei wurden nicht die Größen g und b direkt gemessen, sondern die Position der optischen Elemente und die Bildgröße B.
Die Postion des Perl L ist für jeden Versuch mit $\SI{20}{\centi\metre}$ gemessen worden.

\begin{table}
  \centering
  \caption{bekannte Brennweite}
  \label{tab:Gegenstands}
  \begin{tabular}[t]{c|c|c}
   \toprule
     $\text{Pos d. Linse} \, / \, \si{\centi\metre}$ & $\text{Pos d. Schrims} \, / \, \si{\centi\metre}$ & $B \, / \, \si{\centi\metre}$ \\
     \midrule
     \csvreader[no head,
     late after line=\\,
     late after last line=\\\bottomrule,
     filter test={\ifnumless{\thecsvinputline}{32}}]%
     {data/linse1.csv}{}%
     {\csvcoli & \csvcolii }%
   \end{tabular}
 \end{table}

Um die Brennweite der Linse berechnen zu können, wird die Linsengleichung \eqref{eqn:Linsengleichung} verwendet.
Der Mittelwert wird mittels Numpy nach folgender Formel berechnet:

\begin{equation}
  \label{eqn:mittelwert}
  \overline{x} = \frac{1}{N} \sum_{i=1}^N x_i
\end{equation}

Der Fehler des Mittelwertes nach dieser mittels uncertainties:

\begin{equation}
  \label{eqn:mittelwertfehler}
  \Delta \overline{x} = \frac{1}{\sqrt{N}} \sqrt{\frac{1}{N-1} \sum_{i=1}^N (x_i - \overline{x})^2}
\end{equation}

Es ergibt sich folglich für die Brennweite:

\begin{align*}
  f = \SI{4.45\pm0.06}{\centi\metre}
\end{align*}

Dies wird mit der Herstellerangabe von $\SI{50}{\milli\metre}$ verglichen.
Es ergibt sich ein Fehler von $\SI{10}{\percent}$.
Die Messgenauigkeit wird durch den folgenden Plot überprüft.
Hierbei werden die Wertepaare für $g_i, b_i$ aufgetragen.
Bei hoher Genauigkeit sollten sich alle Geraden in einem Punkt treffen.
Da die Graphen für $g = \SI{5}{\centi\metre}, \SI{13}{\centi\metre}, \SI{14}{\centi\metre}$ zu stark abweichen werden diese für diese Berechnung vernachlässigt.
Damit lässt sich die Brennweite auf $f= $ berechnen.

\begin{figure}
  \centering
  \includegraphics{linse1.pdf}
  \caption{Überprüfen der Messgenauigkeit}
  \label{fig:Messgenauigkeit}
\end{figure}
