\section{Durchführung}
\label{sec:Durchführung}
\subsection{Messung der Gegenstands- und Bildweite}
Zunächst werden auf einer optischen Bank eine Halogenlampe, ein Gegenstand, in diesem Fall ein "Perl L", die zu vermessende Sammellinse sowie ein Schirm befestigt.
Dabei sollten alle Instrumente auf der selben Höhe und nicht gegeneinader verdreht eingestellt sein. Die Position des Gegenstandes wird nun
auf der Längenskala der optischen Bank abgelesen und notiert.Nun wird die Halogenlampe eingeschaltet. Auf dem Schirm hinter der Linse ist nun ein Bild zu sehen.
Der Schirm wird nun so lange verschoben, bis das Bild scharf ist. Dann werden die Position des Schirmes sowie der Linse notiert. Dieser Vorgang wird dann für neun
andere Linsenpositionen wiederholt. Dies wird einmal für eine Linse mit bekannter Brennweite und einmal für eine Wasserlinse mit unbekannter Brennweite durchgeführt.
Bei der Linse mit der bekannten Brennweite wird zusätzlich noch die Größe der Abbildung auf dem Schirm mit einem Lineal gemessen und notiert.

\subsection{Methode nach Bessel}
\label{sec:bessel}
Bei der Methode nach Bessel wird der Gegenstand, sowie der Schirm auf einer festen Position montiert und die jeweiligen Positionen notiert. Dabei ist darauf zu achten,
dass der Abstand zwischen Schirm und Gegenstand mindestens vier mal so groß wie die Brennweite der zu vermessenden Sammellinse ist. Die Sammellinse wird zwischen
Gegenstand und Schirm montiert. Nun wird die Linse verschoben, bis auf dem Schirm ein scharfes Bild erkennbar ist und die zugehörige Position notiert. Danach wird die
Linse weiter verschoben, bis an einer anderen Stelle ein scharfes Bild sichtbar wird. Diese Position wird auch notiert. Dann wird der Vorgang für neun andere
Abstände zwischen Gegenstand und Schirm wiederholt. Danach wird ein Rotfilter hinter dem Gegenstand angebracht und die vorherige Messmethode für fünf Abstände zwischen
Gegenstand und Schirm angewandt. Dies wird dann nochmal mit einem Blaufilter durchgeführt.

\subsection{Methode nach Abbe}
Bei dieser Methode wird wieder ein fester Abstand zwischen Gegenstand und Schirm eingestellt. Diesmal wird jedoch vor der Sammllinse noch eine Zerstreuungslinse mit
einer zur Sammellinse inversen Brennweite angebracht.Die beden Linsen befinden sich dabei so nah aneinader wie möglich. Nun werden für zehn verschiedene Positionen des
Linsensystems jeweils die Größe der Abbildung auf dem Schirm gemessen und die Datenpaare notiert. Dabei ist zu beachten, dass die Position des Linsensystems die Position
eines Referenzpunktes ist, welcher nicht unbedingt in der Hauptebene einer der beiden Linsen liegt.
