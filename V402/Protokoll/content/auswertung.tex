\section{Auswertung}
\label{sec:Auswertung}
\subsection{Berechnung der Brechungsindices $n_i$}
\subsubsection{Berechnung des \texorpdfstring{$\upvarphi$}{phi}-Winkels des verwendeten Prismas}

Für die Berechnung der Brechungsindices ist es notwendig zunächst den Winkel $\upvarphi$ zu bestimmen.
Seine Lage kann der Abbildung \ref{fig:Phi} entnommen werden.
Die Messwerte sind in Tabelle \ref{tab:Phi} zu finden.

\begin{table}
  \centering
  \caption{Gemessene Werte für $\upvarphi_{\text{r}}$ und $\upvarphi_{\text{l}}$, sowie die berechneten $\upvarphi$-Werte}
  \label{tab:Phi}
  \sisetup{table-format=3.1}
  \begin{tabular}{S S [table-format=3.1] S [table-format=3.2]}
    \toprule
    {$\upvarphi_{\text{r}} \:/\: \si{\degree}$} & {$\upvarphi_{\text{l}} \:/\: \si{\degree}$} & {$\upvarphi \:/\: \si{\degree}$} \\
    \midrule
    79.0  & 199.0 & 60.0  \\
    209.3 & 329.5 & 60.1  \\
    198.4 & 318.5 & 60.05 \\
    201.4 & 321.4 & 60.0  \\
    180.1 & 300.2 & 60.05 \\
    223.2 & 343.3 & 60.05 \\
    216.3 & 336.3 & 60.0  \\
    \bottomrule
  \end{tabular}
\end{table}

Um nun den eigentlichen Winkel $\upvarphi$ aus $\upvarphi_{\text{r}}$ und $\upvarphi_{\text{l}}$ der Tabelle \ref{tab:Phi} zu berechnen wird folgende Formel verwendet:

\begin{equation}
  \upvarphi = \frac{1}{2} \left(\upvarphi_r - \upvarphi_l\right)
\end{equation}

Der Mittelwert der errechneten $\upvarphi$ wird mit folgender Formel berechnet:

\begin{equation}
  \label{eqn:mittelwert}
  \overline{x} = \frac{1}{N} \sum_{i=1}^N x_i
\end{equation}

Der entsprechende Fehler mittels dieser:

\begin{equation}
  \label{eqn:mittelwertfehler}
  \Delta \overline{x} = \frac{1}{\sqrt{N}} \sqrt{\frac{1}{N-1} \sum_{i=1}^N (x_i - \overline{x})^2}
\end{equation}

Die Berechnungen erfolgen mit numpy und uncertainties.
Für $\upvarphi$ ergibt sich dann folgender Wert:

\begin{align*}
  \upvarphi &= \SI{60.04 \pm 0.01}{\degree}
\end{align*}

\subsubsection{Berechnung des \texorpdfstring{$\eta$}{eta}-Winkels des parallelen Strahlenganges}

\begin{figure}
  \centering
  \includegraphics[scale=0.6]{images/Eta2.png}
  \caption{Der Brechungswinkel $\eta$: Aus der Anleitung des Versuches \cite[25]{1}}
  \label{fig:Eta2}
\end{figure}

Der Änderungswinkel $\eta$ des einfallenden Lichtstrahles ist in Abbildung \ref{fig:Eta2} dargestellt.
Auch dieser Winkel ist für die Bestimmung des Brechungsindex entscheidend.
Die Größen $\Omega_{\text{r}}$ und $\Omega_{\text{l}}$ der Tabelle \ref{tab:Eta} in denen die Messwerte aufgetragen sind, können der Abbildung \ref{fig:Drehung} entnommen werden.
Die entsprechenden Wellenlängen der Spektrallinien wurden der am Versuch beiliegenden Tabelle entnommen.
Diese ist in Abbildung \ref{fig:Etatabelle}
Der Winkel $\eta$ berechnet sich nach der folgenden Formel:

\begin{equation}
  \eta = 180 - \left(\Omega_r - \Omega_l\right)
\end{equation}

Die Ergebnisse für die entsprechenden Wellenlängen $\lambda_i$ sind ebenfalls in Tabelle \ref{tab:Eta} enthalten.

\begin{table}
  \centering
  \caption{Gemessene Werte für $\Omega_{\text{r}}$ und $\Omega_{\text{l}}$}
  \label{tab:Eta}
  \sisetup{table-format=3.1}
  \begin{tabular}{S S S S [table-format=3.1] S [table-format=2.1]}
    \toprule
    {$\text{Farbe}$} & {$\lambda_i \:/\: \si{\nano\metre}$} & {$\Omega_{\text{r}} \:/\: \si{\degree}$} & {$\Omega_{\text{l}} \:/\: \si{\degree}$} & {$\eta_i \:/\: \si{\degree}$} \\
    \midrule
    \text{rot}         & 579.1 & 237.1 & 119.8 & 62.7 \\
    \text{gelb}        & 577.0 & 236.7 & 120.2 & 63.5 \\
    \text{hellgrün}    & 546.1 & 236.4 & 120.6 & 64.2 \\
    \text{grün}        & 491.6 & 235.8 & 121.1 & 65.3 \\
    \text{hellblau}    & 435.8 & 235.3 & 121.6 & 66.3 \\
    \text{blau}        & 434.7 & 235.1 & 121.9 & 66.8 \\
    \text{hellviolett} & 407.8 & 234.3 & 122.7 & 68.4 \\
    \text{violett}     & 404.7 & 233.3 & 123.7 & 70.4 \\
    \bottomrule
  \end{tabular}
\end{table}

\begin{figure}
  \centering
  \includegraphics[scale=0.1]{images/Tabelle.png}
  \caption{Die Wellenlängen der Spektrallinien: Fotografie der dem Versuch beiliegenden Tabelle}
  \label{fig:Etatabelle}
\end{figure}

\subsubsection{Berechnung der Brechungsindices $n_i$}

\begin{figure}
  \centering
  \includegraphics[scale=0.6]{images/Brechungsindex.png}
  \caption{Veranschaulichung der Winkelbeziehungen für die Berechnung der Brechungsindices $n_i$: Aus der Anleitung des Versuches \cite[23]{1}}
  \label{fig:Brechungsindex}
\end{figure}

Die Brechungsindices $n_i$ in Abhängigkeit der Wellenlängen $\lambda_i$ werden nun nach dem snelliusschen Brechungsgesetz berechnet.
Die Winkelbeziehungen sind in Abbildung \ref{fig:Brechungsindex} dargestellt.
Für n ergibt sich dann folgende Formel:

\begin{equation}
  n = \frac{sin\frac{\eta + \upvarphi}{2}}{sin\frac{\upvarphi}{2}}
\end{equation}

Da es sich bei dem Winkel $\upvarphi$ um eine fehlerbehaftete Größe handelt, wird die Gauß'sche Fehlerfortpflanzung genutzt:

\begin{equation}
  \label{eqn:gauß}
  \Delta f = \sqrt{ \sum_{i=1}^N \left(\frac{\partial}{\partial x_i}\right)^2 \cdot \left(\Delta x_i\right)^2}
\end{equation}

Diese werden mit uncertainties berechnet.
Die Ergebnisse sind in Tabelle \ref{tab:Brechungsindex} zusammengefasst.

\begin{table}
  \caption{Die Brechungsindices $n_i$ in Abhängigkeit von den Wellenlängen $\lambda_i$}
  \label{tab:Brechungsindex}
  \centering
  \sisetup{table-format=3.1}
  \begin{tabular}{S [table-format=3.1]
    S [table-format=1.2]
    @{${}\pm{}$}
    S [table-format=1.2]
    }
    \toprule
    {$\lambda_i \:/\: \si{\nano\metre}$} & \multicolumn{2}{c}{$n_i$} \\
    \midrule
    579.1 & 1.75 & 0.00 \\
    577.0 & 1.76 & 0.00 \\
    546.1 & 1.77 & 0.00 \\
    491.6 & 1.78 & 0.00 \\
    435.8 & 1.78 & 0.00 \\
    434.7 & 1.79 & 0.00 \\
    407.8 & 1.80 & 0.00 \\
    404.7 & 1.81 & 0.00 \\
    \bottomrule
  \end{tabular}
\end{table}

Der Brechungsindex nimmt mit zunehmender Wellenlänge ab.
Es handelt sich also um eine normale Dispersion.

\subsection{Bestimmung der Dispersionskurve}

Nun muss den bestimmten Brechungsindices eine entsprechende Dispersionskurve zugewiesen werden.
Dafür muss zwischen zwei Arten unterschieden werden.
Der normalen Dispersion, wie in Gleichung \ref{eqn:4} dargestellt und der anomalen Dispersion, dargestellt in Gleichung \ref{eqn:5}.
Die entsprechenden Verläufe sind in Abbildung \ref{fig:2} aufgetragen.
Die Quadrate der Brechungsindices wurde in Abbildung \ref{fig:Brechungsplot} gegen die Wellenlänge aufgetragen.
Hier kann man sehen, dass mit hoher Wahrscheinlichkeit eine normale Dispersionsrelation vorliegt.

\begin{figure}
  \centering
  \includegraphics[scale=0.6]{build/plot3.pdf}
  \caption{Die Quadrate der Brechungsindices in Abhängigkeit der Wellenlänge}
  \label{fig:Brechungsplot}
\end{figure}

Nun werden für die Gleichungen \ref{eqn:4} und \ref{eqn:5} die entsprechenden A$_i$, bzw A'$_i$ -Werte gefittet.
Dies erfolgt mit numpy und es ergeben sich folgende Werte:

\begin{align*}
  A_0 &= 2.99 \pm 0.01 \\
  A_2 &= 42087.30 \pm 2904.23 \\
  A'_0 &= 3.36 \pm 0.02 \\
  A'_2 &= (7.59 \pm 0.84) * 10^{-7} \\
\end{align*}

Mit diesen wird jetzt die Summe der Abstandsquadrate nach folgenden Formel berechnet:

\begin{align}
  s^2_n &= \frac{1}{z - 2} \sum_{i=1}^z \left\{n^2(\lambda_i) - A_0 - \frac{A_2}{\lambda_i^2} \right\}^2
  \shortintertext{bzw.} \\
  s^2_{n'} &= \frac{1}{z - 2} \sum_{i=1}^z \left\{n^2(\lambda_i) - A'_0 - A'_2\lambda_i^2 \right\}^2
\end{align}

Mit z = Anzahl der Messwerte.
Mit entsprechend eingesetzten Werten ergibt sich:

\begin{align}
  s^2_n &= foo \\
  s^2_{n'} &= bar
\end{align}

Wie man sieht ist ... kleiner.
Es handelt sich somit bei der Dispersionskurve um eine ... Dispersion.
Die Ergebnisse der Messungen werden mit dem entsprechenden Fit für die Dispersionskurve in Abbildung ... dargestellt.

\subsection{Abbesche Zahl}

Die Abbesche Zahl stellt ein Maß der Farbzerstreuung für ein gegebenes Material dar.
In diesem Fall für Schwerflint SF14.
Durch die zuvor berechnete Dispersionsgleichung kann nun mit der folgenden Formel:

\begin{equation}
  \nu = \frac{n_D - 1}{n_F - n_C}
\end{equation}

die Abbesche Zahl $\nu$ berechnet werden.
Dabei stellen die Konstanten $n_D$, $n_F$ und $n_C$ die entsprechenden Brechungsindices zu sogenannten Fraunhoferschen Linien mit folgenden Werten dar:

\begin{align*}
  \lambda_C &= \SI{656}{\nano\metre} & n_C &= 1.76\\
  \lambda_D &= \SI{589}{\nano\metre} & n_D &= 1.76\\
  \lambda_F &= \SI{486}{\nano\metre} & n_F &= 1.78
\end{align*}

Für die Abbesche Zahl ergibt sich dann:

\begin{align*}
  \nu = 33.52
\end{align*}

\subsection{Auflösungsvermögen}

Das Auflösungsvermögen eines Prismen-Spektralapparates ist die Fähigkeit dessen zwei benachbarte Wellenlängen mit dem Wellenlängenunterschied $\delta\lambda$ gerade noch auflösen, bzw. trennen zu können.
Das Auflösungsvermögen A wird damit folgendermaßen definiert:

\begin{align}
  A &:= \frac{\lambda}{\delta\lambda}
  \intertext{Es kann gezeigt werden, dass das Auflösungsvermögen durch folgende Gleichung gegeben ist:}
  A &=  b \frac{d}{d\lambda} n(\lambda)
\end{align}

Hierbei ist b die Basislänge des Prismas.
Diese wurde nicht explizit gemessen und wird der Anleitung \cite[28]{1} mit $b = 3cm$ entnommen.
Die Ableitung von $n(\lambda)$ nach $\lambda$  ergibt:

\begin{equation}
  \frac{d}{d\lambda} n(\lambda) = \frac{d}{d\lambda} \sqrt{A_0 + \frac{A_2}{\lambda^2}} = - \frac{A_2}{\lambda^3 \sqrt{A_0 + \frac{A_2}{\lambda^2}}}
\end{equation}

Die Ergebnisse können der Tabelle \ref{tab:Auflösungsvermögen} entnommen werden.

\begin{table}
  \centering
  \caption{Die berechneten Auflösungsvermögen für die Fraunhoferschen Linien}
  \label{tab:Auflösungsvermögen}
  \sisetup{table-format=4.2}
  \begin{tabular}{S S [table-format=3.0] S [table-format=4.2]}
    \toprule
    {$\text{Linie}$} & {$\lambda \:/\: \si{\nano\metre}$} & {$\text{Auflösungsvermögen}$} \\
    \midrule
    \lambda_C & 656 &  2547.33\\
    \lambda_D & 589 &  3505.91\\
    \lambda_F & 486 &  6184.41\\
    \bottomrule
  \end{tabular}
\end{table}
