\section{Auswertung}
\label{sec:Auswertung}
\subsection{Berechnung der Brechungsindices $n_i$}
\subsubsection{Berechnung des \texorpdfstring{$\upvarphi$}{phi}-Winkels des verwendeten Prismas}

Für die Berechnung der Brechungsindices ist es notwendig zunächst den Winkel $\upvarphi$ zu bestimmen.
Seine Lage kann der Abbildung \ref{fig:Phi} entnommen werden.
Die Messwerte sind in Tabelle \ref{tab:Phi} zu finden.

\begin{table}
  \centering
  \caption{Gemessene Werte für $\upvarphi_{\text{r}}$ und $\upvarphi_{\text{l}}$, sowie die berechneten $\upvarphi$-Werte}
  \label{tab:Phi}
  \sisetup{table-format=3.1}
  \begin{tabular}{S S [table-format=3.1] S [table-format=3.2]}
    \toprule
    {$\upvarphi_{\text{r}} \:/\: \si{\degree}$} & {$\upvarphi_{\text{l}} \:/\: \si{\degree}$} & {$\upvarphi \:/\: \si{\degree}$} \\
    \midrule
    79.0  & 199.0 & 60.0  \\
    209.3 & 329.5 & 60.1  \\
    198.4 & 318.5 & 60.05 \\
    201.4 & 321.4 & 60.0  \\
    180.1 & 300.2 & 60.05 \\
    223.2 & 343.3 & 60.05 \\
    216.3 & 336.3 & 60.0  \\
    \bottomrule
  \end{tabular}
\end{table}

Um nun den eigentlichen Winkel $\upvarphi$ aus $\upvarphi_{\text{r}}$ und $\upvarphi_{\text{l}}$ der Tabelle \ref{tab:Phi} zu berechnen wird die Gleichung !!!! verwendet.
Der Mittelwert der errechneten $\upvarphi$ wird mit folgender Formel berechnet:

\begin{equation}
  \label{eqn:mittelwert}
  \overline{x} = \frac{1}{N} \sum_{i=1}^N x_i
\end{equation}

Der entsprechende Fehler mittels dieser:

\begin{equation}
  \label{eqn:mittelwertfehler}
  \Delta \overline{x} = \frac{1}{\sqrt{N}} \sqrt{\frac{1}{N-1} \sum_{i=1}^N (x_i - \overline{x})^2}
\end{equation}

Die Berechnungen erfolgen mit numpy und uncertainties.
Für $\upvarphi$ ergibt sich dann folgender Wert:

\begin{align*}
  \upvarphi &= \SI{60.04 \pm 0.01}{\degree}
\end{align*}
