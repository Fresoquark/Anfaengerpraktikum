\section{Diskussion}
\label{sec:Diskussion}
Der Vergleich der Messwerte mit den Literaturwerten, welche aus \cite{3} entnommen wurden, liefert für die Fraunhoferschen Linien und die Abbe-Zahl folgendes:
\begin{align*}
  n_\text{C,gemessen} &= 1.76 &   n_\text{C,Literatur} &= 1.75 \\
  & \Rightarrow \text{Abweichung} = 0.05 \% \\
  n_\text{d,gemessen} &= 1.76 &   n_\text{d,Literatur} &= 1.76 \\
  & \Rightarrow \text{Abweichung} = 0.00 \% \\
  n_\text{F,gemessen} &= 1.78 &   n_\text{F,Literatur} &= 1.78 \\
  & \Rightarrow \text{Abweichung} = 0.00 \% \\
  {\nu}_\text{gemessen} &= 33.52 &   {\nu}_\text{Literatur} &= 26.53 \\
  & \Rightarrow \text{Abweichung} = 26.35 \% .
\end{align*}
Die Abweichung bei den gerundeten Brechungsindices ist sehr gering, was auf eine sehr genaue Messung schließen lässt.
Bei der Abbe-Zahl ist jedoch die Abweichung sehr groß, dies geht warscheinlich auf Rundungsfehler bei den Brechungsindices zurück.
Die Absorptionsstelle liegt bei
\begin{equation*}
      {\lambda}_1 = 145 \pm 5 \: \si{\nano\metre}
\end{equation*}
und ist damit kleiner als die verwendeten Messwellenlängen $ \lambda $, was die Annahme bestätigt, dass Gleichung \ref{eqn:4} die Dispersionskurve besser beschreibt.
Trotz der Abweichung bei der Abbe-Zahl lässt sich das verwendete Material des Prismas mit hinreichender Genauigkeit als Schwerflint SF-14 bestätigen.
