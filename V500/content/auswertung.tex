\section{Auswertung}
\label{sec:Auswertung}
Um $\text{U}_\text{g}$ zu bestimmen, wird für die einzelnen Lichtwellenlängen $\sqrt{\text{I}}$ / $\sqrt{\SI{100}{\pico\ampere}}$ gegen $\text{U}$ / $\si{\volt}$ in einem Diagramm aufgetragen.
Die entsprechenden Messwerte sind in Tabelle \ref{tab:messreihe1} zu finden.
Die so entstehenden Plots wurden in Abbildung \ref{fig:messreihe1} zusammengetragen.
Es ergeben sich für die verschiedenen Ausgleichsgeraden der entsprechenden Lichtwellenlängen folgende Werte, hierbei entspricht $\text{a}_\text{x}$ der Steigung und $\text{b}_\text{x}$ dem Y-Achsenabschnitt der entsprechenden Lichtwellenlänge:

\begin{align*}
  \text{a}_\text{gelb} &= -2.85 \pm 0.11    \,\frac{\sqrt{\SI{100}{\pico\ampere}}}{\si{\volt}}   \\
  \text{b}_\text{gelb} &= 1.75 \pm 0.04     \,\sqrt{\SI{100}{\pico\ampere}}  \\
  \text{a}_\text{rot} &= -0.62 \pm 0.05    \,\frac{\sqrt{\SI{100}{\pico\ampere}}}{\si{\volt}}        \\
  \text{b}_\text{rot} &= 0.66 \pm 0.04       \,\sqrt{\SI{100}{\pico\ampere}}      \\
  \text{a}_\text{grün} &= -3.99 \pm 0.17    \,\frac{\sqrt{\SI{100}{\pico\ampere}}}{\si{\volt}}       \\
  \text{b}_\text{grün} &= 2.80 \pm 0.07        \,\sqrt{\SI{100}{\pico\ampere}}    \\
  \text{a}_\text{violett} &= -3.18 \pm 0.14     \,\frac{\sqrt{\SI{100}{\pico\ampere}}}{\si{\volt}}   \\
  \text{b}_\text{violett} &= 3.89 \pm 0.09       \,\sqrt{\SI{100}{\pico\ampere}}  \\
  \text{a}_\text{ultraviolett} &= -1.93 \pm 0.05   \,\frac{\sqrt{\SI{100}{\pico\ampere}}}{\si{\volt}}\\
  \text{b}_\text{ultraviolett} &= 2.67 \pm 0.03  \,\sqrt{\SI{100}{\pico\ampere}} 
\end{align*}

\FloatBarrier
\begin{figure}
  \centering
  \includegraphics{messreihe1.pdf}
  \caption{Lineare Regressionen für die verschiedenen Spektrallinien.}
  \label{fig:messreihe1}
\end{figure}
\FloatBarrier

Die Berechnung erfolgt mit SciPy, die Fehlerrechnung mittels uncertainties.
Aus diesen Parametern wird die Nullstelle der Funktionen und damit $\text{U}_\text{g}$ bestimmt.
Es ergibt sich folglich für die Nullstellen $\text{N}_\text{x} = \text{U}_\text{g, x}$:

\begin{align*}
  \text{N}_\text{gelb}         &= \SI{0.61\pm0.03}{\volt} \\
  \text{N}_\text{rot}          &= \SI{1.07\pm0.10}{\volt} \\
  \text{N}_\text{grün}         &= \SI{0.70\pm0.03}{\volt} \\
  \text{N}_\text{violett}      &= \SI{1.22\pm0.06}{\volt} \\
  \text{N}_\text{ultraviolett} &= \SI{1.38\pm0.04}{\volt}
\end{align*}
\FloatBarrier
\begin{table}
  \centering
  \caption{Messwerte für die verschiedenen Spektrallinien.}
  \label{tab:messreihe1}
  \begin{tabular}{c c c c c c}
   \toprule
    $U$ / $\si{\volt}$ & $I_\text{Gelb}$ / $\SI{100}{\pico\ampere}$ & $I_\text{Rot}$ / $\SI{100}{\pico\ampere}$ & $I_\text{Grün}$ / $\SI{100}{\pico\ampere}$ & $I_\text{Violett}$ / $\SI{100}{\pico\ampere}$ & $I_\text{Ultraviolett}$ / $\SI{100}{\pico\ampere}$ \\
     \midrule
     \csvreader[no head,
     late after line=\\,
     late after last line=\\\bottomrule]%
     {data/messreihe1.csv}{}%
     {\csvcoli & \csvcolii  & \csvcoliii & \csvcoliv & \csvcolv & \csvcolvi}%
   \end{tabular}
 \end{table}
\FloatBarrier

Um nun das Verhältnis $\text{h} / \text{e}_0$ und die Austrittsarbeit $A_k$ bestimmen zu können, werden die zuvor berechneten $\text{N}_\text{x}$ / $\text{U}_\text{g, x}$ gegen die Lichtfrequenz $\nu$ aufgetragen und linear gefittet.
Die Lichtwellenlänge werden der Versuchsanleitung\cite[80]{sample} entnommen.
Die Werte sind in Tabelle \ref{tab:Lichtwellenlänge} zusammengefasst.
Der entsprechende Plot wurde in Abbildung \ref{fig:he0} dargestellt.
Die so bestimmten Parameter sind:
\begin{align*}
  a&=\SI{2.60(9)e-3}{\volt\per\tera\hertz} \\
  b&=\SI{-0.73(6)}{\volt}.
\end{align*}
Durch Vergleich mit Gleichung \eqref{eqn:energie2} werden folgende Zusämmenhänge zwischen der Parametern und dem gesuchten Verhältnis $\text{h} / \text{e}_0$ sowie der Austrittsarbeit $A_k$ in $\si{\electronvolt}$ hergeleitet:
\begin{align*}
  \frac{h}{e_0} &= a \\
  A_k &= -b \cdot [\si{\coulomb}] .
\end{align*}
Somit wurden folgende Werte bestimmt:
\begin{align*}
    \frac{h}{e_0} &= \SI{2.60(9)e-15}{\joule\second\per\coulomb} \\
    A_k &= \SI{0.73(6)}{\electronvolt} .
\end{align*}


\begin{table}
  \centering
  \caption{Die untersuchten Spektrallinien des Lichtes.}
  \label{tab:Lichtwellenlänge}
  \begin{tabular}{c|c|c|l}
    \toprule
    $\lambda$ / $\si{\nano\metre}$ & $\nu$ / $\si{\tera\hertz}$ & $\text{U}_\text{g}$ / $\si{\volt}$ & Farbe \\
    \midrule
    671 & 467.69 & 1.07\pm0.10 & rot \\
    578 & 518.67 & 0.61\pm0.03 & gelb \\
    546 & 549.07 & 0.70\pm0.03 & grün \\
    405 & 740.23 & 1.22\pm0.06 & violett \\
    366 & 819.11 & 1.38\pm0.04 & ultraviolett \\
    \bottomrule
  \end{tabular}
\end{table}

\begin{figure}
  \centering
  \includegraphics{he0.pdf}
  \caption{Lineare Regressionen der einzelnen $\text{U}_\text{g}$ gegen die Lichtwellenlänge $\nu$.}
  \label{fig:he0}
\end{figure}
\FloatBarrier

Bei der zweiten Messreihe wurden die in Tabelle \ref{tab:messreihe2} befindlichen Daten aufgenommen.
Nun wird $I$ gegen $U$ in einem Diagramm aufgetragen. Dies ist in Abbildung \ref{fig:messreihe2} zu sehen.
Zur Deutung des Kurvenverlaufes müssen verschiedene Fregestellungen geklärt werden.
So zum Einen, dass sich der Photostrom bei steigender Beschleunigungspannung einem Sättigungswert annähert.
Dies geht darauf zurück, dass der maximale Photostrom von der Intensität des Lichtes abhängt. Ist die
Beschleunigungsspannung groß genug um alle herausgelösten Elektronen zur Anode zu bewegen, lässt sich der
Strom nicht mehr vergrößern. Das asymptotische Annähern an den Sättigungswert ist dabei unter anderem der
Energieverteilung der Elektronen im Festkörper zuzuschreiben, wodurch manche Elektronen mit einer sehr geringen
Energie austreten und deshalb erst bei großen Beschleunigungsspannungen die Anode erreichen.
Analog dazu wird das Absinken des Photostromes schon vor Erreichen der Bremsspannung $U_g$ darauf zurück geführt,
dass die Elektronen unterschiedliche Energien im Festkörper besitzen. Außerdem kann unter Umständen ein negativer
Strom auftreten. Dieser geht darauf zurück, dass das Kathodenmaterial bereits bei $T=\SI{20}{\celsius}$
verdampfen kannl, und somit Elektronen frei werden. Diese lagern sich an der Kathode an und werden dann von der
Bremsspannung zur Kathode beschleunigt. Da es sich hirbei jedoch um eine sehr geringe Anzahl an Elektronen handelt, wird
sehr schnell ein Sättigungswert erreicht. Da solche negativen Ströme auch bei energiearmen Licht auftreten, kann geschlussfolgert werden, dass
die Anode eine Austrittsarbeit besitzt, welche der der Kathode ähnelt.

\begin{table}
  \centering
  \caption{Messwerte für die gelbe Spektrallinie bei $\SI{278}{\nano\metre}$}
  \label{tab:messreihe2}
  \begin{tabular}[t]{c|c}
   \toprule
     $U$ / $\si{\volt}$ & $I_\text{Gelb}$ / $\SI{100}{\pico\ampere}$ \\
     \midrule
     \csvreader[no head,
     late after line=\\,
     late after last line=\\\bottomrule,
     filter test={\ifnumless{\thecsvinputline}{20}}]%
     {data/messreihe2.csv}{}%
     {\csvcoli & \csvcolii }%
   \end{tabular}
  \begin{tabular}[t]{c|c}
   \toprule
      $U$ / $\si{\volt}$ & $I_\text{Gelb}$ / $\SI{100}{\pico\ampere}$ \\
    \midrule
    \csvreader[filter test={\ifnumgreater{\thecsvinputline}{19}},
    late after line=\\,
    late after last line=\\\bottomrule]%
    {data/messreihe2.csv}{}%
    {\csvcoli & \csvcolii}%
  \end{tabular}
\end{table}

\begin{figure}
  \centering
  \includegraphics{messreihe2.pdf}
  \caption{Verluaf von $I$ gegen $U$ für die gelbe Spektrallinie bei $\SI{278}{\nano\metre}$.}
  \label{fig:messreihe2}
\end{figure}
