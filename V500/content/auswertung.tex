\section{Auswertung}
\label{sec:Auswertung}

Um $\text{U}_\text{g}$ zu bestimmen, wird für die einzelnen Lichtwellenlängen $\sqrt{\text{I}}$ / $\sqrt{\SI{100}{\pico\ampere}}$ gegen $\text{U}$ / $\si{\volt}$ in einem Diagramm aufgetragen.
Die entsprechenden Messwerte sind in Tabelle \ref{tab:messreihe1} zu finden.
Die so entstehenden Plots wurden in Abbildung \ref{fig:messreihe1} zusammengetragen.

\begin{table}
  \centering
  \caption{Messwerte für die verschiedenen Spektrallinien.}
  \label{tab:messreihe1}
  \begin{tabular}{c c c c c c}
   \toprule
    $U$ / $\si{\volt}$ & $I_\text{Gelb}$ / $\SI{100}{\pico\ampere}$ & $I_\text{Rot}$ / $\SI{100}{\pico\ampere}$ & $I_\text{Grün}$ / $\SI{100}{\pico\ampere}$ & $I_\text{Violett}$ / $\SI{100}{\pico\ampere}$ & $I_\text{Ultraviolett}$ / $\SI{100}{\pico\ampere}$ \\
     \midrule
     \csvreader[no head,
     late after line=\\,
     late after last line=\\\bottomrule]%
     {data/messreihe1.csv}{}%
     {\csvcoli & \csvcolii  & \csvcoliii & \csvcoliv & \csvcolv & \csvcolvi}%
   \end{tabular}
 \end{table}

 \begin{figure}
   \centering
   \includegraphics{messreihe1.pdf}
   \caption{Lineare Regressionen für die verschiedenen Spektrallinien.}
   \label{fig:messreihe1}
 \end{figure}

Es ergeben sich für die verschiedenen Ausgleichsgeraden der entsprechenden Lichtwellenlängen folgende Werte, hierbei entspricht $\text{a}_\text{x}$ der Steigung und $\text{b}_\text{x}$ dem Y-Achsenabschnitt der entsprechenden Lichtwellenlänge:

\begin{align*}
  \text{a}_\text{gelb} &= -2.85 +- 0.11         \\
  \text{b}_\text{gelb} &= 1.75 +- 0.04          \\
  \text{a}_\text{rot} &= -0.62 +- 0.05          \\
  \text{b}_\text{rot} &= 0.66 +- 0.04           \\
  \text{a}_\text{grün} &= -3.99 +- 0.17         \\
  \text{b}_\text{grün} &= 2.80 +- 0.07          \\
  \text{a}_\text{violett} &= -3.18 +- 0.14      \\
  \text{b}_\text{violett} &= 3.89 +- 0.09       \\
  \text{a}_\text{ultraviolett} &= -1.93 +- 0.05 \\
  \text{b}_\text{ultraviolett} &= 2.67 +- 0.03
\end{align*}

Die Berechnung erfolgt mit SciPy, die Fehlerrechnung mittels uncertainties.
Aus diesen Parametern wird die Nullstelle der Funktionen und damit $\text{U}_\text{g}$ bestimmt.
Es ergibt sich folglich für die Nullstellen $\text{N}_\text{x} = \text{U}_\text{g, x}$:

\begin{align*}
  \text{N}_\text{gelb}         &= \SI{0.61\pm0.03}{\volt} \\
  \text{N}_\text{rot}          &= \SI{1.07\pm0.10}{\volt} \\
  \text{N}_\text{grün}         &= \SI{0.70\pm0.03}{\volt} \\
  \text{N}_\text{violett}      &= \SI{1.22\pm0.06}{\volt} \\
  \text{N}_\text{ultraviolett} &= \SI{1.38\pm0.04}{\volt}
\end{align*}

Um nun das Verhältnis $\text{h} / \text{e}_0$ bestimmen zu können, werden die zuvor berechneten $\text{N}_\text{x}$ / $\text{U}_\text{g, x}$ gegen die Lichtfrequenz $\nu$ aufgetragen.
Die Lichtwellenlänge werden der Versuchsanleitung\cite[80]{sample} entnommen.
Die Werte sind in Tabelle \ref{tab:Lichtwellenlänge} zusammengefasst.
Der entsprechende Plot wurde in Abbildung \ref{fig:he0} dargestellt.

\begin{table}
  \centering
  \caption{Die untersuchten Spektrallinien des Lichtes.}
  \label{tab:Lichtwellenlänge}
  \begin{tabular}{c|c|c|l}
    \toprule
    $\lambda$ / $\si{\nano\metre}$ & $\nu$ / $\si{\tera\hertz}$ & $\text{U}_\text{g}$ / $\si{\volt}$ & Farbe \\
    \midrule
    671 & 467.69 & 1.07\pm0.10 & rot \\
    578 & 518.67 & 0.61\pm0.03 & gelb \\
    546 & 549.07 & 0.70\pm0.03 & grün \\
    405 & 740.23 & 1.22\pm0.06 & violett \\
    366 & 819.11 & 1.38\pm0.04 & ultraviolett \\
    \bottomrule
  \end{tabular}
\end{table}

\begin{figure}
  \centering
  \includegraphics{he0.pdf}
  \caption{Lineare Regressionen der einzelnen $\text{U}_\text{g}$ gegen die Lichtwellenlänge $\nu$.}
  \label{fig:he0}
\end{figure}

\begin{table}
  \centering
  \caption{Messwerte für die gelbe Spektrallinie bei $\SI{278}{\nano\metre}$}
  \label{tab:es}
  \begin{tabular}[t]{c|c}
   \toprule
     $U$ / $\si{\volt}$ & $I_\text{Gelb}$ / $\SI{100}{\pico\ampere}$ \\
     \midrule
     \csvreader[no head,
     late after line=\\,
     late after last line=\\\bottomrule,
     filter test={\ifnumless{\thecsvinputline}{20}}]%
     {data/messreihe2.csv}{}%
     {\csvcoli & \csvcolii }%
   \end{tabular}
  \begin{tabular}[t]{c|c}
   \toprule
      $U$ / $\si{\volt}$ & $I_\text{Gelb}$ / $\SI{100}{\pico\ampere}$ \\
    \midrule
    \csvreader[filter test={\ifnumgreater{\thecsvinputline}{19}},
    late after line=\\,
    late after last line=\\\bottomrule]%
    {data/messreihe2.csv}{}%
    {\csvcoli & \csvcolii}%
  \end{tabular}
\end{table}
