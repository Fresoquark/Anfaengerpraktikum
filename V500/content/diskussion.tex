\section{Diskussion}
\label{sec:Diskussion}

Betrachtet man die Daten der roten und gelben Spektrallinien in Abbildung \ref{fig:messreihe1} so sieht es so aus, als würde ein anderer Zusammenhang, als ein linearer bestehen.
So müssen bei der gelben Spektrallinie einige Messwerte aus dem Fit genommen werden, um eine vernünftige Ausgleichsrechnung durchführen zu können.
Hier hätte man einen kleineren Messbereich nutzen sollen.
Der Fit der roten Daten hingegen ist im Vergleich zu den restlichen Ausgleichsgeraden sehr flach.
Dies könnte darauf zurückzuführen sein, dass die rote Spektrallinie auf dem Schirm nicht, wie in der Anleitung dargestellt, stark sichtbar war, sondern eher schwach.
Dieser Umstand hat auch Auswirkungen auf die Bestimmung der h/e$_0$.
Wie in Abbildung \ref{fig:he0} erkennbar ist, passt der bestimmte Wert für U$_\text{g}$ nicht zu der gegebenen Lichtwellenlänge des roten Lichtes.
Sie müssen daher für eine bessere Regression ignoriert werden.
Es wird nun der bestimmte Wert für h/e$_0$ mit einem aus den Literaturwerten für h\cite{Planck} und e$_0$\cite{Ladung} errechneten Verhältnis verglichen.
\begin{align*}
  h &= \SI{6.63e-34}{\joule\per\second} \\
  e_0 &= \SI{1.60e-19}{\coulomb} \\
  \frac{h}{e_0} &= \SI{4.14e-15}{\joule\per\coulomb\per\second}
\end{align*}
Dies ergibt eine Abweichung von $\SI{37.12}{\percent}$ zwischen dem experimentell bestimmten Wert und dem Wert, der sich aus den Literaturwerten ergibt.
