\section{Auswertung}
\label{sec:Auswertung}

\subsection{\texorpdfstring{$\gamma$}{Gamma}-Strahlung}

Für die Nullmessung wurden folgende Daten ermittelt:

\begin{align*}
  \text{Messzeiteit: } t &= \SI{900}{\second} \\
  \text{Anzahl der Wechselwirkungen: } N &= (950 \pm 31) \\
  \text{Aktivität: } A_0 &= \SI{1.06 \pm 0.03}{\per\second}
\end{align*}

Die Fehler der Anzahl der Wechselwirkungen ergibt sich dadurch, dass diese statistisch nach der Poissonverteilung verteilt sind.
Der Fehler berechnet sich dann folgendermaßen:

\begin{equation}
  \Delta N = \sqrt{N}
\end{equation}

Die gemessenen Wechselwirkungen und die totalen Aktivitäten weisen somit ebenfalls einen Fehler auf.
Nach der gaußschen Fehlerfortpflanzung für $A_\text{total} = A - A_0$

\begin{gather}
    \Delta A_\text{total} = \frac{\partial A_\text{total}}{\partial A} \cdot \Delta A + \frac{\partial A_\text{total}}{\partial A_0} \cdot \Delta A_0 \\
    \Delta A_\text{total} = \Delta A - \Delta A_0
\end{gather}

wird der Fehler für die totale Aktivität berechnet.
Die so berechneten Werte sind für Zink in Tabelle \ref{tab:zink1} und für Blei in Tabelle \ref{tab:blei1} aufgetragen.

\begin{table}
  \centering
  \caption{Messwerte zur Bestimmung des Absorptionskoeffizienten von Zink}
  \label{tab:zink1}
  \begin{tabular}{c c c c}
    \toprule
    {Dicke d / $\si{\milli\metre}$} & {Zeit t / $\si{\second}$} & {Zählrate N} & {Aktivität (A - A$_0$) / $\si{\per\second}$} \\
    \midrule
    20 & 180 & 11022 \pm 105 & 60.2  \pm 0.4 \\
    18 & 160 & 10602 \pm 103 & 65.2  \pm 0.5 \\
    16 & 150 & 10898 \pm 104 & 71.6  \pm 0.5 \\
    14 & 140 & 11194 \pm 106 & 78.9  \pm 0.5 \\
    12 & 140 & 11933 \pm 109 & 84.2  \pm 0.6 \\
    10 & 120 & 11561 \pm 108 & 95.3  \pm 0.6 \\
    8  & 100 & 9970  \pm 100 & 98.6  \pm 0.7 \\
    6  & 90  & 10478 \pm 102 & 115.4 \pm 0.8 \\
    4  & 80  & 9716  \pm 99  & 120.4 \pm 0.8 \\
    2  & 80  & 10718 \pm 104 & 132.9 \pm 0.9 \\
    0  & 100 & 15013 \pm 123 & 149.1 \pm 0.9 \\
    \bottomrule
  \end{tabular}
\end{table}

\begin{table}
  \centering
  \caption{Messwerte zur Bestimmung des Absorptionskoeffizienten von Blei}
  \label{tab:blei1}
  \begin{tabular}{c c c c}
    \toprule
    {Dicke d / $\si{\milli\metre}$} & {Zeit t / $\si{\second}$} & {Zählrate N} & {Aktivität (A - A$_0$) / $\si{\per\second}$} \\
    \midrule
    50 & 700 & 1245  \pm 35  & 0.7   \pm 0.0 \\
    40 & 500 & 1334  \pm 37  & 1.6   \pm 0.0 \\
    30 & 350 & 2500  \pm 50  & 6.1   \pm 0.1 \\
    20 & 220 & 3839  \pm 62  & 16.4  \pm 0.1 \\
    15 & 170 & 5189  \pm 72  & 29.5  \pm 0.2 \\
    12 & 150 & 5784  \pm 76  & 37.5  \pm 0.3 \\
    10 & 130 & 6233  \pm 79  & 46.9  \pm 0.4 \\
    5  & 100 & 8110  \pm 90  & 80.0  \pm 0.6 \\
    4  & 90  & 7703  \pm 88  & 84.5  \pm 0.6 \\
    3  & 70  & 7324  \pm 86  & 103.6 \pm 0.8 \\
    2  & 60  & 7144  \pm 85  & 118.0 \pm 0.9 \\
    1  & 60  & 7787  \pm 88  & 128.7 \pm 1.0 \\
    0  & 100 & 15067 \pm 123 & 149.6 \pm 0.9 \\
    \bottomrule
  \end{tabular}
\end{table}

Zur Bestimmung der Absorptionskoeffizienten werden die Aktivitäten halblogarithmisch gegen die Dicke des Absorbermaterials aufgetragen.
Dies wird anschließend über lineare Regression gefittet.
Dies ist für Zink in Abbildung \ref{fig:zink1} und für Blei in Abbildung \ref{fig:blei1} dargestellt.

\begin{figure}
  \centering
  \includegraphics{build/zink.pdf}
  \caption{Die lineare Regression zur Bestimmung des Absorptionskoeffizienten von Zink.}
  \label{fig:zink1}
\end{figure}

\begin{figure}
  \centering
  \includegraphics{build/blei.pdf}
  \caption{Die lineare Regression zur Bestimmung des Absorptionskoeffizienten von Blei.}
  \label{fig:blei1}
\end{figure}
