\section{Auswertung}
\label{sec:Auswertung}
\subsection{Bestimmung der maximalen Energie eines \texorpdfstring{$\beta$}{beta}-Strahlers}
Die Messdaten befinden sich in Tabelle \ref{tab:beta}.
\begin{table}
  \centering
  \caption{Messwerte zur Absorption von Betastrahlung bei Aluminium}
  \label{tab:beta}
  \begin{tabular}[t]{c@{} c c}
   \toprule
    Messzeit / $\si{\second}\;\;$ & Counts & Dicke / $\si{\micro\metre}$ \\
     \midrule
     \csvreader[no head,
     late after line=\\,
     late after last line=\\\bottomrule]%
     {data/aluminium.csv}{}%
     {$\SI{\csvcoli}{}$ & $\SI{\csvcolii}{}$ & $\SI{\csvcoliii}{}$}%
   \end{tabular}
 \end{table}
 Die Hintergrundstrahlung wurde zu $0,631 \text{Counts}/\si{\second}$ bestimmt. Die gemessenen Counts werden durch die Messzeiten dividiert und dann die Hintergrundstrahlung
 abgezogen. Dann wird das Ergebnis logarhitmisch gegen die Massenbelegung R, welche das Produkt der Dicke des Absobers und der aus \cite{dichte} entnommenen Dichte von
 Aluminium ist, aufgetragen. Dies ist in Abbildung \ref{fig:beta} zu sehen. Zur Ermittlung der maximalen Reichweite $R_\text{max}$ der $\beta$-Strahlung werden nun die beiden linearen
 Anteile der Kurve mittels SciPy mit $y=ax+b$ linear gefittet.
 \begin{figure}
   \centering
   \includegraphics{aluminium.pdf}
   \caption{Grafische Darstellung der Messdaten mit linearen Fits.}
   \label{fig:beta}
 \end{figure}
 Der Schnittpunkt der beiden Geraden ist die maximale Reichweite und wird folgendermaßen ermittelt:
 \begin{equation*}
   R_\text{max}= \frac{b_2 - b_1}{a_1 - a_2} .
 \end{equation*}
 Der zugehörige Fehler wird mit
 \begin{equation*}
  \Delta R_\text{max} = \sqrt{\frac{\Delta_{a_{1}}^{2} \left(- b_{1} + b_{2}\right)^{2}}{\left(a_{1} - a_{2}\right)^{4}} + \frac{\Delta_{a_{2}}^{2} \left(- b_{1} + b_{2}\right)^{2}}{\left(a_{1} - a_{2}\right)^{4}} + \frac{\Delta_{b_{1}}^{2}}{\left(a_{1} - a_{2}\right)^{2}} + \frac{\Delta_{b_{2}}^{2}}{\left(a_{1} - a_{2}\right)^{2}}}
 \end{equation*}
ermittelt.
So wurden die Parameter und die maximale Reichweite zu folgenden Werten bestimmt:
\begin{align*}
  a_1 &= \SI{-110.59\pm5.83}{\centi\metre\squared\per\gram} \\
  a_2 &= \SI{1.58\pm6.70}{\centi\metre\squared\per\gram} \\
  b_1 &= \num{6.39\pm0.25} \\
  b_2 &= \num{-2.76\pm0.72} \\
  R_\text{max} &= \SI{0.082\pm0.009}{\gram\per\centi\metre\squared} .
\end{align*}
Nun wird mit der Gleichung \eqref{eqn:betaenergie} aus $R_\text{max}$ die maximale Energie der $\beta$-Strahlung ermittelt.
Der zugehörige Fehler wird mit
\begin{equation*}
  \Delta E_\text{max}= \sqrt{\frac{\Delta_{R_{\text{max}}}^{2} \cdot {1.92}^{2} \left(R_{\text{max}} + \frac{{0.22}}{2}\right)^{2}}{R_{\text{max}}^{2} + R_{\text{max}} {0.22}}}
\end{equation*}
berechnet.
So wurde folgende maximale Energie berechnet:
\begin{equation*}
  E_\text{max} = \SI{301\pm22}{\kilo\electronvolt}.
\end{equation*}
