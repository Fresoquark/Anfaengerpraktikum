\section{Auswertung}
\label{sec:Auswertung}
\subsection{Bestimmung der maximalen Energie eines \texorpdfstring{$\beta$}{beta}-Strahlers}
Die Messdaten befinden sich in Tabelle \ref{tab:beta}.
\begin{table}
  \centering
  \caption{Messwerte zur Absorption von Betastrahlung bei Aluminium}
  \label{tab:beta}
  \begin{tabular}[t]{c@{} c c}
   \toprule
    Messzeit / $\si{\second}\;\;$ & Counts & Dicke / $\si{\micro\metre}$ \\
     \midrule
     \csvreader[no head,
     late after line=\\,
     late after last line=\\\bottomrule]%
     {data/aluminium.csv}{}%
     {$\SI{\csvcoli}{}$ & $\SI{\csvcolii}{}$ & $\SI{\csvcoliii}{}$}%
   \end{tabular}
 \end{table}
 Die Hintergrundstrahlung wurde zu $0,631 \text{Counts}/\si{\second}$ bestimmt. Die gemessenen Counts werden durch die Messzeiten dividiert und dann die Hintergrundstrahlung
 abgezogen. Dann wird das Ergebnis logarhitmisch gegen die Massenbelegung R, welche das Produkt der Dicke des Absobers und der aus \cite{dichte} entnommenen Dichte von
 Aluminium ist, aufgetragen. Dies ist in Abbildung \ref{fig:beta} zu sehen. Zur Ermittlung der maximalen Reichweite $R_\text{max}$ der $\beta$-Strahlung werden nun die beiden linearen
 Anteile der Kurve mittels SciPy mit $y=ax+b$ linear gefittet.
 \begin{figure}
   \centering
   \includegraphics{aluminium.pdf}
   \caption{Grafische Darstellung der Messdaten mit linearen Fits.}
   \label{fig:beta}
 \end{figure}
 Der Schnittpunkt der beiden Geraden ist die maximale Reichweite und wird folgendermaßen ermittelt:
 \begin{equation*}
   R_\text{max}= \frac{b_2 - b_1}{a_1 - a_2} .
 \end{equation*}
 Der zugehörige Fehler wird mit
 \begin{equation*}
  \Delta R_\text{max} = \sqrt{\frac{\Delta_{a_{1}}^{2} \left(- b_{1} + b_{2}\right)^{2}}{\left(a_{1} - a_{2}\right)^{4}} + \frac{\Delta_{a_{2}}^{2} \left(- b_{1} + b_{2}\right)^{2}}{\left(a_{1} - a_{2}\right)^{4}} + \frac{\Delta_{b_{1}}^{2}}{\left(a_{1} - a_{2}\right)^{2}} + \frac{\Delta_{b_{2}}^{2}}{\left(a_{1} - a_{2}\right)^{2}}}
 \end{equation*}
ermittelt.
So wurden die Parameter und die maximale Reichweite zu folgenden Werten bestimmt:
\begin{align*}
  a_1 &= \SI{-110.59\pm5.83}{\centi\metre\squared\per\gram} \\
  a_2 &= \SI{1.58\pm6.70}{\centi\metre\squared\per\gram} \\
  b_1 &= \num{6.39\pm0.25} \\
  b_2 &= \num{-2.76\pm0.72} \\
  R_\text{max} &= \SI{0.082\pm0.009}{\gram\per\centi\metre\squared} .
\end{align*}
Nun wird mit der Gleichung \eqref{eqn:betaenergie} aus $R_\text{max}$ die maximale Energie der $\beta$-Strahlung ermittelt.
Der zugehörige Fehler wird mit
\begin{equation*}
  \Delta E_\text{max}= \sqrt{\frac{\Delta_{R_{\text{max}}}^{2} \cdot {1.92}^{2} \left(R_{\text{max}} + \frac{{0.22}}{2}\right)^{2}}{R_{\text{max}}^{2} + R_{\text{max}} {0.22}}}
\end{equation*}
berechnet.
So wurde folgende maximale Energie berechnet:
\begin{equation*}
  E_\text{max} = \SI{301\pm22}{\kilo\electronvolt}.
\end{equation*}

\subsection{Bestimmung des Absorptionskoeffizienten der \texorpdfstring{$\gamma$}{Gamma}-Strahlung für Zink und Blei}

Für die Nullmessung wurden folgende Daten ermittelt:

\begin{align*}
  \text{Messzeiteit: } t &= \SI{900}{\second} \\
  \text{Anzahl der Wechselwirkungen: } N &= (950 \pm 31) \\
  \text{Aktivität: } A_0 &= \SI{1.06 \pm 0.03}{\per\second}
\end{align*}

Die Fehler der Anzahl der Wechselwirkungen ergibt sich dadurch, dass diese statistisch nach der Poissonverteilung verteilt sind.
Der Fehler berechnet sich dann folgendermaßen:

\begin{equation}
  \Delta N = \sqrt{N}
\end{equation}

Die gemessenen Wechselwirkungen und die totalen Aktivitäten weisen somit ebenfalls einen Fehler auf.
Nach der gaußschen Fehlerfortpflanzung für $A_\text{total} = A - A_0$ wird der Fehler für die totale Aktivität nach

\begin{gather}
    \Delta A_\text{total} = \frac{\partial A_\text{total}}{\partial A} \cdot \Delta A + \frac{\partial A_\text{total}}{\partial A_0} \cdot \Delta A_0 \\
    \Delta A_\text{total} = \Delta A - \Delta A_0
\end{gather}

berechnet.
Die so berechneten Werte sind für Zink in Tabelle \ref{tab:zink1} und für Blei in Tabelle \ref{tab:blei1} aufgetragen.

\begin{table}
  \centering
  \caption{Messwerte zur Bestimmung des Absorptionskoeffizienten von Zink}
  \label{tab:zink1}
  \begin{tabular}{c c c c}
    \toprule
    {Dicke d / $\si{\milli\metre}$} & {Zeit t / $\si{\second}$} & {Zählrate N} & {Aktivität (A - A$_0$) / $\si{\per\second}$} \\
    \midrule
    20 & 180 & 11022 \pm 105 & 60.2  \pm 0.4 \\
    18 & 160 & 10602 \pm 103 & 65.2  \pm 0.5 \\
    16 & 150 & 10898 \pm 104 & 71.6  \pm 0.5 \\
    14 & 140 & 11194 \pm 106 & 78.9  \pm 0.5 \\
    12 & 140 & 11933 \pm 109 & 84.2  \pm 0.6 \\
    10 & 120 & 11561 \pm 108 & 95.3  \pm 0.6 \\
    8  & 100 & 9970  \pm 100 & 98.6  \pm 0.7 \\
    6  & 90  & 10478 \pm 102 & 115.4 \pm 0.8 \\
    4  & 80  & 9716  \pm 99  & 120.4 \pm 0.8 \\
    2  & 80  & 10718 \pm 104 & 132.9 \pm 0.9 \\
    0  & 100 & 15013 \pm 123 & 149.1 \pm 0.9 \\
    \bottomrule
  \end{tabular}
\end{table}

\begin{table}
  \centering
  \caption{Messwerte zur Bestimmung des Absorptionskoeffizienten von Blei}
  \label{tab:blei1}
  \begin{tabular}{c c c c}
    \toprule
    {Dicke d / $\si{\milli\metre}$} & {Zeit t / $\si{\second}$} & {Zählrate N} & {Aktivität (A - A$_0$) / $\si{\per\second}$} \\
    \midrule
    50 & 700 & 1245  \pm 35  & 0.7   \pm 0.0 \\
    40 & 500 & 1334  \pm 37  & 1.6   \pm 0.0 \\
    30 & 350 & 2500  \pm 50  & 6.1   \pm 0.1 \\
    20 & 220 & 3839  \pm 62  & 16.4  \pm 0.1 \\
    15 & 170 & 5189  \pm 72  & 29.5  \pm 0.2 \\
    12 & 150 & 5784  \pm 76  & 37.5  \pm 0.3 \\
    10 & 130 & 6233  \pm 79  & 46.9  \pm 0.4 \\
    5  & 100 & 8110  \pm 90  & 80.0  \pm 0.6 \\
    4  & 90  & 7703  \pm 88  & 84.5  \pm 0.6 \\
    3  & 70  & 7324  \pm 86  & 103.6 \pm 0.8 \\
    2  & 60  & 7144  \pm 85  & 118.0 \pm 0.9 \\
    1  & 60  & 7787  \pm 88  & 128.7 \pm 1.0 \\
    0  & 100 & 15067 \pm 123 & 149.6 \pm 0.9 \\
    \bottomrule
  \end{tabular}
\end{table}

Zur Bestimmung der Absorptionskoeffizienten werden die Aktivitäten halblogarithmisch gegen die Dicke des Absorbermaterials aufgetragen.
Dies wird anschließend über lineare Regression gefittet.
Die Regression erfolgt durch Python.
Dies ist für Zink in Abbildung \ref{fig:zink1} und für Blei in Abbildung \ref{fig:blei1} dargestellt.
In der Ausgleichsrechnung ist die Steigung dem Absorptionskoeffizienten und der y-Achsenabschnitt der Anfangsaktivität zugeordnet.
Es ergeben sich für Zink folgende Werte:

\begin{align*}
  \text{Absorptionskoeffizient: } \mu &= \SI{45.2 \pm 1.2}{\per\metre} \\
  \text{Anfangsaktivität: } A_0 &= \SI{147.0 \pm 1.4}{\per\second}
\end{align*}

Der Theoriewert des Compton-Absorptionskoeffizienten $\mu_\text{com}$ wird berechnet, indem der Compton-Wirkungsquerschnitt \ref{eqn:sigmacom} mit der Teilchenanzhal pro Volumen \ref{eqn:n} multipliziert wird.
Es ergibt sich daher:

\begin{align*}
  \mu_\text{com, Zink} = \SI{49.7}{\per\metre}.
\end{align*}

Die erforderlichen Werte des Molvolumens wurden \cite{zink} entnommen.
Dieser Wert ist etwas höher als der gemessene Wert, liegt aber noch im Toleranzbereich.
Es wird somit lediglich der Comptoneffekt eine Rolle spielen.

Für Blei ergeben sich folgende Werte:

\begin{align*}
  \text{Absorptionskoeffizient: } \mu &= \SI{115.6 \pm 3.8}{\per\metre} \\
  \text{Anfangsaktivität: } A_0 &= \SI{146.0 \pm 2.1}{\per\second}
\end{align*}

Der theoretische Compton-Absorptionskoeffizienten $\mu_\text{com}$ berechnet sich nun zu:

\begin{align*}
  \mu_\text{com, Blei} = \SI{68.1}{\per\metre}
\end{align*}

Die erforderlichen Werte des Molvolumens wurden \cite{blei} entnommen.
Dieser Wert liegt wesentlich unter dem gemessenen Wert.
Es ist daher anzunehmen, dass zusätzlich noch der Photoeffekt beteiligit ist.

\begin{figure}
  \centering
  \includegraphics{build/zink.pdf}
  \caption{Die lineare Regression zur Bestimmung des Absorptionskoeffizienten von Zink.}
  \label{fig:zink1}
\end{figure}

\begin{figure}
  \centering
  \includegraphics{build/blei.pdf}
  \caption{Die lineare Regression zur Bestimmung des Absorptionskoeffizienten von Blei.}
  \label{fig:blei1}
\end{figure}
