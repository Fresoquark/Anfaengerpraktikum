\section{Diskussion}
\label{sec:Diskussion}
Folgende Werte und Abweichungen von den Literaturwerten wurden bestimmt:
\begin{align*}
  \mu_\text{Zink,gemessen} &= \SI{45.2 \pm 1.2}{\per\metre} & \mu_\text{com, Zink} &= \SI{49.7}{\per\metre} \\
  & \Rightarrow \text{Relative Abweichung} = \SI{9.05}{\percent} \\
  \mu_\text{Blei,gemessen} &= \SI{115.6 \pm 3.8}{\per\metre} & \mu_\text{com, Blei} &=  \SI{68.1}{\per\metre} \\
  & \Rightarrow \text{Relative Abweichung} = \SI{69.75}{\percent} \\
  E_\text{max,gemessen} &=  \SI{312\pm22}{\kilo\electronvolt} & E_\text{Literatur} &=  \SI{293}{\kilo\electronvolt} \\
  & \Rightarrow \text{Relative Abweichung} = \SI{6.48}{\percent} 
\end{align*}
Der Literautrwert für die Energie eines Technetium-99 Strahlers wurde \cite{99Tc} entnommen.
Bei der $\gamma$-Absorption weicht der gemessene Absorptionskoeffizient nur sehr wenig von dem
theoretischen Compton-Absorptionskoeffizienten ab, was qualitativ zu erwarten war, da Zink ein Metall mit einem relativ leichten Kern ist. So lässt sich der
Comptoneffekt als vorherrschender Absorptionsprozess bei Zink mit hinreichender Ganuigkeit verifizieren. Beim Blei dagegen eine starke Abweichung erkennbar,
was vermutlich darauf zurück geht, dass Blei einen deutlich schwereren Kern hat, und somit der Photoeffekt einen größeren Einfluss hat. Darüber wie groß dieser ist
oder ob der Photoeffekt der vorherrschende Absorptionsprozess ist lässt sich anhand dieses Versuchs keine Aussage treffen. Die maximale Energie der $\beta$-Strahlung zeigt
nur eine geringe Abweichung vom Literaturwert, was entweder eine Messungenauigkeit ist, jedoch auch dadurch zu erklären ist, dass in der Quelle nicht unbedingt die maximale
sondern warscheinlich eher die mittlere Energie der Strahlung angegeben ist.
