\section{Auswertung}
\label{sec:Auswertung}

In der nachfolgenden Auflistung sind die gemessenen Werte für den Abstand zwischen den Spalten und dem verwendeten Photoelement $L$,
der Wellenlänge des verwendeten He-Ne-Lasers $\lambda$,
sowie des Dunkelstromes $I_D$ zusammengetragen:

\begin{align*}
  L &= \SI{99.5}{\centi\metre} \\
  \lambda &= \SI{635}{\nano\metre} \\
  I_D &= \SI{21}{\nano\ampere}
\end{align*}

\subsection{Einzelspalt}

Um die Spaltbreite des Einzelspaltes berechnen zu können, wird eine Ausgleichsrechnung nach Formel \ref{eqn:Einzelspalt} vorgenommen.
Die der Messung zu Grunde liegenden Daten können Tabelle \ref{tab:es} entnommen werden.

\begin{table}
  \centering
  \caption{Einzelspalt}
  \label{tab:es}
  \begin{tabular}[t]{c|c}
   \toprule
     $x \, / \, \si{\milli\metre}$ & $I \, / \, \si{\micro\ampere}$ \\
     \midrule
     \csvreader[no head,
     late after line=\\,
     late after last line=\\\bottomrule,
     filter test={\ifnumless{\thecsvinputline}{32}}]%
     {data/es.csv}{}%
     {\csvcoli & \csvcolii }%
   \end{tabular}
  \begin{tabular}[t]{c|c}
   \toprule
    $x \, / \, \si{\milli\metre}$ & $I \, / \, \si{\micro\ampere}$ \\\midrule
    \csvreader[filter test={\ifnumgreater{\thecsvinputline}{31}},
    late after line=\\,
    late after last line=\\\bottomrule]%
    {data/es.csv}{}%
    {\csvcoli & \csvcolii}%
  \end{tabular}
\end{table}

Der Winkel $\phi$ kann nicht direkt gemessen werden.
Er wird durch

\begin{equation}
  \phi \approx \frac{x - x_{\text{max}}}{L}
\end{equation}

angenähert.
Der Fit wird mittels SciPy durchgeführt und ist in Abbildung \ref{fig:Einzelspalt} dargestellt.
Der Dunkelstrom $I_D = \SI{21}{\nano\ampere}$ wird von den Daten der Tabelle abgezogen.
Für die so gefundenen Parameter ergeben sich folgende Werte:

\begin{align*}
  \Delta \phi &= \SI{3.35 \pm 0.64e-4}{\radian} \\
  b_{\text{Einzelspalt}} &= \SI{0.022 \pm 0.001}{\milli\metre} \\
  b_{\text{Einzelspalt, Hersteller}} &= \SI{0.075}{\milli\metre}
\end{align*}

Es ergibt sich eine Abweichung von $\SI{-70.67}{\percent}$ für die gemessene Breite b des Einzelspaltes und den Herstellerangaben.

\begin{figure}
  \centering
  \includegraphics{es.pdf}
  \caption{Die Messwerte des Einzelspaltes, sowie der entsprechende Fit}
  \label{fig:Einzelspalt}
\end{figure}

\subsection{Die Doppelspalte}

\begin{table}
  \centering
  \caption{Doppelspalt 1}
  \label{tab:ds1}
  \begin{tabular}[t]{c|c}
   \toprule
     $x \, / \, \si{\milli\metre}$ & $I \, / \, \si{\micro\ampere}$ \\
     \midrule
     \csvreader[no head,
     late after line=\\,
     late after last line=\\\bottomrule,
     filter test={\ifnumless{\thecsvinputline}{31}}]%
     {data/ds1.csv}{}%
     {\csvcoli & \csvcolii }%
   \end{tabular}
  \begin{tabular}[t]{c|c}
   \toprule
    $x \, / \, \si{\milli\metre}$ & $I \, / \, \si{\micro\ampere}$ \\\midrule
    \csvreader[filter test={\ifnumgreater{\thecsvinputline}{30}},
    late after line=\\,
    late after last line=\\\bottomrule]%
    {data/ds1.csv}{}%
    {\csvcoli & \csvcolii}%
  \end{tabular}
\end{table}
\begin{table}
  \centering
  \caption{Doppelspalt 2}
  \label{tab:ds2}
  \begin{tabular}[t]{c|c}
   \toprule
     $x \, / \, \si{\milli\metre}$ & $I \, / \, \si{\micro\ampere}$ \\
     \midrule
     \csvreader[no head,
     late after line=\\,
     late after last line=\\\bottomrule,
     filter test={\ifnumless{\thecsvinputline}{27}}]%
     {data/ds2.csv}{}%
     {\csvcoli & \csvcolii }%
   \end{tabular}
  \begin{tabular}[t]{c|c}
   \toprule
    $x \, / \, \si{\milli\metre}$ & $I \, / \, \si{\micro\ampere}$ \\\midrule
    \csvreader[filter test={\ifnumgreater{\thecsvinputline}{26}},
    late after line=\\,
    late after last line=\\\bottomrule]%
    {data/ds2.csv}{}%
    {\csvcoli & \csvcolii}%
  \end{tabular}
\end{table}
